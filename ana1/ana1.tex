% Created 2016-11-10 Do 03:34
\documentclass[a4paper]{scrartcl}
\usepackage[utf8]{inputenc}
\usepackage[T1]{fontenc}
\usepackage{fixltx2e}
\usepackage{graphicx}
\usepackage{longtable}
\usepackage{float}
\usepackage{wrapfig}
\usepackage{rotating}
\usepackage[normalem]{ulem}
\usepackage{amsmath}
\usepackage{textcomp}
\usepackage{marvosym}
\usepackage{wasysym}
\usepackage{amssymb}
\usepackage{hyperref}
\tolerance=1000
\usepackage{siunitx}%
\usepackage{fontspec}%
\sisetup{load-configurations = abbrevations}%
\newcommand{\estimates}{\overset{\scriptscriptstyle\wedge}{=}}%
\usepackage{mathtools}%
\DeclarePairedDelimiter\abs{\lvert}{\rvert}%
\DeclarePairedDelimiter\norm{\lVert}{\rVert}%
\DeclareMathOperator{\Exists}{\exists}%
\DeclareMathOperator{\Forall}{\forall}%
\DeclareMathOperator{\sgn}{sgn}
\def\colvec#1{\left(\vcenter{\halign{\hfil$##$\hfil\cr \colvecA#1;;}}\right)}
\def\colvecA#1;{\if;#1;\else #1\cr \expandafter \colvecA \fi}
\usepackage{xparse}% http://ctan.org/pkg/xparse
\NewDocumentCommand{\overarrow}{O{=} O{\uparrow} m}{%
\overset{\makebox[0pt]{\begin{tabular}{@{}c@{}}#3\\[0pt]\ensuremath{#2}\end{tabular}}}{#1}
}
\NewDocumentCommand{\underarrow}{O{=} O{\downarrow} m}{%
\underset{\makebox[0pt]{\begin{tabular}{@{}c@{}}\ensuremath{#2}\\[0pt]#3\end{tabular}}}{#1}
}
\newcommand{\ubar}[1]{\text{\b{$#1$}}}
\makeatletter
\newcommand{\pushright}[1]{\ifmeasuring@#1\else\omit\hfill$\displaystyle#1$\fi\ignorespaces}
\newcommand{\pushleft}[1]{\ifmeasuring@#1\else\omit$\displaystyle#1$\hfill\fi\ignorespaces}
\makeatother
\newcommand{\I}{\ensuremath{\imath}}%
\author{Robin Heinemann}
\date{\today}
\title{Analysis I (Marciniak-Czochra)}
\hypersetup{
  pdfkeywords={},
  pdfsubject={},
  pdfcreator={Emacs 25.1.1 (Org mode 8.2.10)}}
\begin{document}

\maketitle
\tableofcontents


\section{Einleitung}
\label{sec-1}
Webseite www.biostruct.uni-heidelberg.de/Analysis1.php
Klausurzulassung: 50\%
Klausur 18.2.2017 9-12Uhr
\section{Mengen und Zahlen}
\label{sec-2}
\subsection{Logische Regeln und Zeichen}
\label{sec-2-1}
\subsubsection{Quantoren}
\label{sec-2-1-1}
\begin{center}
\begin{tabular}{ll}
$\Forall x$ & für alle $x$\\
$\exists x$ & es gibt (mindestens) ein $x$\\
$\exists! x$ & es gibt genau ein $x$\\
\end{tabular}
\end{center}
\subsubsection{Hinreichend und Notwendig}
\label{sec-2-1-2}
\begin{itemize}
\item $A\Rightarrow B$: wenn $A$ gilt, gilt auch $B$, $A$ ist \textbf{hinreichend} für $B$, daraus folgt: $B$ ist \textbf{notwendig} für $A$, Ungültigkeit von $B$ impliziert die Ungültigkeit von $A$ ($\neg B \Rightarrow \neg A$)
\item $A \Leftrightarrow B$: $A$ gilt, genau dann, wenn $B$ gilt
\end{itemize}
\subsubsection{Beweistypen}
\label{sec-2-1-3}
\paragraph{Direkter Schluss}
\label{sec-2-1-3-1}
$A\Rightarrow B$
\subparagraph{Beispiel}
\label{sec-2-1-3-1-1}
$m$ gerade Zahl $\Rightarrow$ $m^2$ gerade Zahl
\begin{enumerate}
\item Beweis
\label{sec-2-1-3-1-1-1}
$m$ gerade $\Rightarrow \exists n\in\mathbb{N}~\text{sodass}~m = 2n \Rightarrow m^2 = 4n^2 = 2k,~\text{wobei}~k=2n^2\in\mathbb{N} \square$
\end{enumerate}
\paragraph{Beweis der Transponerten (der Kontraposition)}
\label{sec-2-1-3-2}
Zum Beweis $A\Rightarrow B$ zeigt man $\neg B\Rightarrow \neg A~(A\Rightarrow B)\Leftrightarrow (\neg B) \Rightarrow (\neg A)$
\subparagraph{Beispiel}
\label{sec-2-1-3-2-1}
Sei $m\in\mathbb{N}$, dann gilt $m^2~\text{gerade}~\Rightarrow m~\text{gerade}$
\begin{enumerate}
\item Beweis
\label{sec-2-1-3-2-1-1}
Wir zeigen: $m$ ist ungerade $\Rightarrow m^2$ ungerade
\[\exists n\in\mathbb{N}:~m=2n+1\Rightarrow m^2 = (2n+1)^2 = 2k+1, k=2n^2 + 2n\in\mathbb{N}\Rightarrow m^2~\text{ungerade} \square\]
\end{enumerate}
\paragraph{Indirekter Schluss ( Beweis durch Wiederspruch)}
\label{sec-2-1-3-3}
Man nimmt an, dass $A\Rightarrow B$ nicht gilt, das heißt $A \wedge \neg B$ und zeigt, dass dann für eine Aussage $C$ gelten muss $C\Rightarrow \neg C$, also ein Wiederspruch
\subparagraph{Beispiel}
\label{sec-2-1-3-3-1}
$\not\exists q\in\mathbb{Q}: a^2 = 2$
\begin{enumerate}
\item Beweis
\label{sec-2-1-3-3-1-1}
Wir nehmen an, dass $\exists a\in\mathbb{Q}: a^2=2$ Dann folgt:
$\exists b,c\in\mathbb{Z}$ teilfremd (ohne Einschränkung, denn sonst kürzen soweit wie möglich) mit $a=\frac{b}{c}$
Falls \[a^2=2\Rightarrow (\frac{b}{c})^2=2=\frac{b^2}{c^2}=2 \Rightarrow b^2 = 2c^2 \Rightarrow b^2~\text{gerade}~\Rightarrow b~\text{ist gerade (schon gezeigt)}\] \[\Rightarrow\exists d\in\mathbb{N}~\text{sodass}~b=2d\Rightarrow b^2=4d^2\]
Außerdem $b^2=2c^2\Rightarrow 2c^2=4d^2\Rightarrow c^2=2d^2\Rightarrow c$ ist auch gerade. Also müssen $b$ und $c$ beide gerade sein, also nicht teilerfremd, damit haben wir einen Widerspruch hergeleitet $\square$
\end{enumerate}
\subsubsection{Summenzeichen und Produktzeichen}
\label{sec-2-1-4}
\paragraph{Summenzeichen}
\label{sec-2-1-4-1}
Wir definieren für $m > 0$ \[\sum_{k=m}^m a_k := a_m + \ldots + a_n\] falls $n\geq m$
\[\sum_{k=m}^n a_k := 0\] falls $n < m$ (sogennante leere Summe)
\paragraph{Produktzeichen}
\label{sec-2-1-4-2}
\[\prod_{k=m}^n a_k := \begin{cases} a_m \cdot ... \cdot a_n & \text{falls}~n\geq m\\ 1 & \text{falls}~n<m~\text{(sog. leeres Produkt)}\end{cases}\]
\subsection{Mengen}
\label{sec-2-2}
\subsubsection{Definition}
\label{sec-2-2-1}
(Georg cantor 1885) Unter einer \uline{Menge} verstehen wir jede Zusammenfassung $M$ von bestimmten wohlunterschiedenen Objekten (welche die Elemente von $M$ genannt werden), zu einem Ganzen
$M$ dadurch ist charakterisiert, dass von jedem vorliegendem Objekt $x$ feststeht, ab gilt
\begin{itemize}
\item $x\in M$ (x Element von M)
\item x \textlnot{}$\in$ M (x kein Element von M)
\end{itemize}
\[M = \{x_1, x_2, \ldots, x_n\}\]
\[M=\{x\mid A(x)\} \rightarrow~\text{eine Menge}~M\text{für die}~x\in\ M \Leftrightarrow A(x)\]
\subsubsection{Mengenrelationen}
\label{sec-2-2-2}
\begin{itemize}
\item Mengeninklusion $A\subseteq M$ ($A$ ist eine Teilmenge von $M$)
\[\Forall x: (x\in A \Rightarrow x\in M)\] ,zum Beispiel $\mathbb{N} \subseteq \mathbb{Z}$
\item \[A = B \Leftrightarrow \Forall x: (x\in A \Leftrightarrow x\in B)\]
\item \[A \subset M~\text{(strikte Teilmenge)}~\Leftrightarrow A\subset M \wedge A \neq M\]
\item \[\emptyset:~\text{leere Menge}~\not\exists x: x\in\emptyset\]. Wir setzen fest, dass $\emptyset$ eine Teilmenge jeder Menge ist. Zum Beispiel \[\{x\in\mathbb{R}: x^2 + 1 = 0\}\]
\item Durchschnitt \[A\cap B := \{x\mid x\in A \wedge x\in B\}\]
\item Vereinigung \[A\cup B := \{x \mid x\in A \vee x\in B\}\]
\item Differenz (auch Komplement von $B$ in $A$) \[A\setminus B := \{x\mid x\in A \wedge x\not\in B\} := C_a B~\text{(auch $B^c$)}\]
\end{itemize}
\subsubsection{Potenzmenge}
\label{sec-2-2-3}
Potenzmenge $A$
\[\mathcal{P}(A) := \{B\mid B\subseteq A\}\]
Alle Teilmengen von $A$
\paragraph{Beispiel}
\label{sec-2-2-3-1}
\[\mathcal{P}(\{1,2\}) = \{\{1\}, \{2\}, \{1,2\}, \emptyset\}\]
\subsubsection{Familien von Mengen}
\label{sec-2-2-4}
Sei $I$ eine Indexmenge, $I \subseteq \mathbb{N}, (A_i)_{i\in I}$ eine Familie von Mengen $A$
\paragraph{Durchschnitt von $A$}
\label{sec-2-2-4-1}
\[\cap_{i\in I} = \{x\mid \Forall_{i\in I} x\in A_i\}\]
\paragraph{Vereinigung}
\label{sec-2-2-4-2}
\[\cup_{i\in I} = \{x\mid\exists i\in I: x\in A_i\}\]
\subsubsection{Rechenregeln}
\label{sec-2-2-5}
$A,B,C,D$ seien Mengen
\begin{itemize}
\item $\emptyset \subseteq A$
\item $A\subseteq A$ \hfill Reflexivität
\item $A\subseteq B, B\subseteq C \Rightarrow A\subseteq C$ \hfill Transitivität
\item $A\cap B = B\cap A$ \\ $A\cup B = B\cup A$ \hfill Kommutativität
\item $(A\cap B)\cap C = A\cap (B\cap C)$ \\ $(A\cup B) \cup C = A\cup (B\cup C)$ \hfill Assoziativität
\item $A\cap (B\cup C) =(A\cap B) \cup (A\cap C)$ \\ $A\cup (B\cap C) =(A\cup B) \cap (A\cup C)$
\item Eigenschaften der Komplementbildung: \\
      Seien $A,B \subseteq D (C_D A: = D\setminus A)$, dann gilt \[C_D (C_D A) = A\] \[C_D(A\cap B) = C_D A \cup C_D B\] \[C_D(A\cup B) = C_D A \cap C_D B\]
\begin{itemize}
\item Beweis:
\[x\in C_D(A\cap B) \Leftrightarrow x\in D \wedge (x\not\in (A\cap B)) \Leftrightarrow x\in D \wedge (x\not\in A \vee x\not\in B)\] \[\Leftrightarrow (x\in D\wedge x\not\in A) \vee (x\in D \wedge x\not\in B)\] \[\Leftrightarrow (x\in D\setminus A) \vee (x\in D\setminus B) \Leftrightarrow x\in D\setminus(A\cup B)~\square\]
\item Bemerkung: Komplement kann man auch mit $A^c$ bezeichnen
\end{itemize}
\end{itemize}
\subsubsection{geordneter Tupel}
\label{sec-2-2-6}
Sei $x_1, x_2, \ldots, x_n$ (nicht notwendig verschiedene) Objekte. Ein geordneter n-Tupel \[(x_1,x_2,\ldots,x_n) = (y_1,\ldots,y_n) \Leftrightarrow x_1 = y_1, \ldots, x_n = y_n\]
Beachte:
\[\{x_1, \ldots, x_n\} = \{y_i,\ldots,y_n\}\not\implies x_1 = y_1, \ldots, x_n = y_n\]
\subsubsection{Kartesisches Produkt}
\label{sec-2-2-7}
Seien \[A_1\times A_2\times \ldots \times A_n = \{(x_1,x_2,\ldots,x_n)\mid x_j \in A_j j\in\mathbb{N}, j \leq n\}\]
\paragraph{Beispiel}
\label{sec-2-2-7-1}
\begin{itemize}
\item \[\mathbb{Z}^2 = \mathbb{Z}\times \mathbb{Z}\]
\item $R^n$ n-dimensionaler Raum von reellen Zahlen
\end{itemize}
\subsubsection{Äquivalenzrelation}
\label{sec-2-2-8}
Eine Äquivalenzrelation auf eine Menge $A$ ist eine Beziehung zwischen ihren Elementen (Bezeichnung: $a \sim b$), sodass
\begin{itemize}
\item Für jede zwei $a,b\in A$ gilt entweder $a\sim b \vee a\not\sim b$
\item $a\sim a$ \hfill Reflexivität
\item $a\sim b \Rightarrow b\sim a$ \hfill Symmetrie
\item $a \sim b, b \sim c \Rightarrow a \sim c$ \hfill Transitivität
\end{itemize}
Mit Hilfe einer Äquivalenzrelation lassen sich die Elemente einer Menge in sogenannte Äquivalenzklassen einordnen: $[a]:\{b\in A\mid b\sim a\}$
\subsection{Relationen und Abbildungen}
\label{sec-2-3}
\subsubsection{Relationen}
\label{sec-2-3-1}
Unter einer \textbf{Relation} verstehen wir eine Teilmenge $R\subseteq X\times Y$ wobei $X, Y$ Mengen sind. Für $x\in X$ definieren wir, das \textbf{Bild} von $x$ unter $R$
\[R(X) := \{y\in Y \mid (x,y) \in R\}\]
und *Definitionsbereiche von $R$ (bezüglich $X$)
\[D(R):= \{x\in X\mid R(x)\neq\emptyset\}\]
\subsubsection{Graph der Abbildung}
\label{sec-2-3-2}
$R\subseteq X\times Y$ heißt Graph der Abbildung (Funktion) \[f:X\rightarrow Y \Leftrightarrow D(R) = X, \Forall x\in X: R(x) = \{f(x)\}\]
also enthält $R(x)$ genau ein Element. \\
    $X$ heißt Definitionsbereich von $f$ \\
    $Y$ heißt Werte- oder Bildbereich von $f$ (Bild) \\
    $x\in X$ heißt Argument \\
    $f(x)\in Y$ heißt Wert von $f$ an der Stelle x
\paragraph{Beispiel}
\label{sec-2-3-2-1}
$f: \mathbb{R}\rightarrow\mathbb{R}, x\rightarrow x^2$ dann ist der Graph von $f = \{(x,y)\in\mathbb{R}^2, y=x^2\}$
\subparagraph{Bemerkung}
\label{sec-2-3-2-1-1}
\[M^{*}(x) = \{(x,y)\in\mathbb{R}^2;x=y^2\} = \{(x,y)\in\mathbb{R}^2: x \geq 0, y=\sqrt{x} \vee y = -\sqrt{x}\}\]
Ist kein Graph einer Funktion $\mathbb{R}\rightarrow\mathbb{R}$, denn $M^{ *}(x) = \{\sqrt{x},-\sqrt{x}, x\geq 0\}$
$f$ heißt
\begin{itemize}
\item surjektiv, wenn gilt $f(X) = Y$
\item injectiv, $\Forall x_1,x_2\in X: f(x_1) = f(x_2) \Rightarrow x_1 = x_2$
\item bijektiv, wenn $f$ surjektiv und injectiv ist
\end{itemize}
\subsubsection{Umkehrabbildung}
\label{sec-2-3-3}
Sei die Abbildung $f: X\rightarrow Y$ bijektiv. Dann definieren wir die Umkehrabbildung $f^{-1}:Y\rightarrow X$ durch $y\rightarrow x\in X$, eindeutig bestimmt durch $y = f(x)$
\paragraph{Bemerkung}
\label{sec-2-3-3-1}
\[(x,y) \in~\text{Graph }f\Leftrightarrow (y,x)\in~\text{Graph }f^{-1}\]
\subsubsection{Komposition}
\label{sec-2-3-4}
Seien $f:X\rightarrow Y, g:Y\rightarrow Z$ Abbildungen. Die Komposition von $g$ und $f$ \[g\circ f: X\rightarrow Z~\text{ist durch}~x\rightarrow g(f(x))~\text{definiert}\]
\subsubsection{Identitäts Abbildung}
\label{sec-2-3-5}
Für jede Menge $X$ definieren wir die identische Abbildung \[I_d(A) = I_A: A\rightarrow A,~\text{durch}~x\rightarrow x\]
\paragraph{Beispiel}
\label{sec-2-3-5-1}
\begin{itemize}
\item \[\{(x,y)\in\mathbb{R}^2\mid x^2 + y^2 = 1\} = S^1\] \[S^{n-1} := \{(x_1 \ldots x_n) \in \mathbb{R}^n; \sum_{i = 1}^n x_i^2 = 1\}\] $(n - 1)$ dimensionale sphere in $\mathbb{R}^n$
\item Seien $X,Y$ Mengen, $M\subseteq X\times Y, f:M\rightarrow X$ \\ $f$ heißt Projektion, $f$ surjektiv \[f(M) = \{x\mid \exists y \in Y : (x,y) \in M\} = X\]
\end{itemize}
\subsubsection{Homomorphe Abbildungen}
\label{sec-2-3-6}
Existieren auf Mengen $X$ und $Y$ mit gewissen Operationen $\oplus_x$ bzw. $\oplus_y$ (zum Beispiel Addition, Ordungsrelation), ho heißt die Abbildung $f:X\to Y$ homomorph (strukturerhaltend), wenn gilt $\Forall x_1,x_2 \in X f(x_1\oplus_x x_2) = f(x_1)\oplus_y f(x_2)$
Eine bijektive Homomorphie heißt Isomorphisumus, beziehungsweise $X\approx Y$ (äquivalent, isomorph)
\subsection{Natürliche Zahlen}
\label{sec-2-4}
$\mathbb{N} = \{1,2,3,\ldots\},~\mathbb{N}_0 := \mathbb{N}\cup \{0\}$
\subsubsection{Peanosche Axiomensystem der natürlichen Zahlen}
\label{sec-2-4-1}
\begin{enumerate}
\item Die Zahl $1$ ist eine natürliche Zahl $1\in\mathbb{N}$
\item Zu jeder natürlichen Zahl $n$, gibt es genau einen "Nachfolger" $n' (=: n+1)$
\item Die Zahl 1 ist kein Nachfolger einer natürlichen Zahl
\item $n' = m' \Rightarrow n = m$
\item Enthält eine Teilmenge  $M \subseteq \mathbb{N}$ die Zahl $1$ und von jedem $n\in m$ auch den Nachfolger $n'$ ist $M = \mathbb{N}$
\end{enumerate}
Bemerkung: \\
    Mit Hilfe der Axiome lassen sich auf $\mathbb{N}$ Addition ($+$), Multiplikation ($\cdot$) und Ordung ($\leq$) einführen.
Wir definieren: \\
    $1' = 2, 2' = 3, \ldots$
$n + 1 := m'$
$n + m' := (n+m)';~n\cdot m' := n m + n$
Man kann zeigen, dass jede Menge, welche die Peano Axiome erfüllt isomorph bezüglich Multiplikation und Addition zu $\mathbb{N}$ ist
Wir definieren $n < m \Leftrightarrow \exists x\in \mathbb{N}: x + m = m$
\subsubsection{Vollständige Induktion}
\label{sec-2-4-2}
\paragraph{Induktionsprinzip}
\label{sec-2-4-2-1}
Es seien die folgende Schritte vollzogen:
\begin{enumerate}
\item Induktionsverankerung (Induktionsanfang): Die Aussage $A(1)$ gilt
\item Induktionsschluss: Ist für ein $n\in\mathbb{N}~A(n)$ gültig, so folgt auch die Gültigkeit von $A(n+1)$
\end{enumerate}
Dann sind alle Aussagen $A(n),n\in\mathbb{N}$ gültig.
\paragraph{Beweis:}
\label{sec-2-4-2-2}
Wir definieren die Tailmenge $M\subseteq\mathbb{N},~M:=\{n\in \mathbb{N}\mid A(N)~\text{ist gültig}\}$
Die Induktionsverankerung besagt, dass $1\in M$ und die Induktionsannahme $n\in M\Rightarrow n + 1 \in M$. Folglich ist nach dem 5. Axiom von Peano $M = \mathbb{N}\hfill\square$
\paragraph{Beispiel 1}
\label{sec-2-4-2-3}
Zu Beweisen: \[\Forall n\in\mathbb{N} \sum_{i = 1}^n i^2 = \frac{n(n+1)(2n+1)}{6}\]
\subparagraph{Beweis}
\label{sec-2-4-2-3-1}
\begin{enumerate}
\item Induktionsverankerung: $1^2 = \frac{1}{6}\cdot 1\cdot 2\cdot 3$
\item Annahme: $A(n)$ gültig für $n\in\mathbb{N}: \sum_{i = 1}^n i^2 = \frac{n(n+1)(2n+1)}{6}$ \\
         Zu zeigen $A(n + 1): 1^2 + \ldots + (n+1)^2 = \frac{1}{6} (n+1)(n+2)(2n+3)$
         \[1^2 + \ldots + n^2 + (n+1)^2 = \frac{1}{2} n(n+1)(2n+1) + (n+1)^2 = (n+1)(\frac{1}{3}n^2 + \frac{1}{6}n + n + 1)\]
         \[= \frac{1}{6}(n+1)(2n^2+7n+6) = \frac{1}{6}(n+1)(2n+3)(n+2)\hfill\square\]
\end{enumerate}
\paragraph{Beispiel 2}
\label{sec-2-4-2-4}
Definition von Potenzen
\[x^0 := 1\]
\[\Forall n\in\mathbb{N} x^n := x^{n - 1}x\]
(iterative (rekursive) Definition) \\
     Auf $\mathbb{N}$ sind diese elementaren Operationene erklärt:
\begin{itemize}
\item Addition $a+b$
\item Multiplikation $a\cdot b$
\item (unter gewissen Vorraussetzungen):
\begin{itemize}
\item Subtraktion $a-b$
\item Division $\frac{a}{b}$
\end{itemize}
\end{itemize}
$\mathbb{N}$ ist bezüglich "$-$" oder "$/$" nicht vollständig, das heißt $n+x = m$ ist nicht lösbar in $\mathbb{N}$
Erweiterungen:
\begin{itemize}
\item Ganze Zahlen $\mathbb{Z}:=\{0; \pm, n\in\mathbb{N}\}$ \\
       Negative Zahl $(-n)$ ist definiert duch $n+(-n) = 0$
\item Rationale Zahlen $\mathbb{Q}~(b x = y)$
\end{itemize}
Man sagt, dass $(\mathbb{Q},+,\cdot)$ einen Körper bildet.
\subsubsection{Definition Körper}
\label{sec-2-4-3}
$\mathbb{K}$ sei eine Menge auf der Addition und Multiplikation sei. $\mathbb{K}$ heißt ein Körper, wenn die folgende Axiome erfüllt sind:
\begin{itemize}
\item Addition: $(\mathbb{K}, +)$ ist eine kummutative Gruppe, das heißt $\Forall a,b,c\in \mathbb{K}$:
\begin{enumerate}
\item \((a+b)+c = a+(b+c)\) \hfill Assoziativität
\item \(a+b = b+a\) \hfill Kommutativität
\item \(\exists! 0\in\mathbb{K}:a+0 = a\)\hfill Existenz des Nullelement
\item \(\exists x\in\mathbb{K}: a+x = 0\)\hfill Existstenz des Nagativen
\end{enumerate}
\item Multiplikation: $(\mathbb{K}\setminus\{0\},\cdot)$ ist eine kommutative Gruppte, das heißt $\Forall a,b,c\in\mathbb{K}$
\begin{enumerate}
\item \((a\cdot b)\cdot c = a\cdot(b\cdot c)\)\hfill Assozativität
\item \(a\cdot b = b\cdot a\)\hfill Kummutativität
\item \(\exists!1\in\mathbb{K}:a\cdot 1 = a\)\hfill Existenz des Einselement
\item Für \(a\neq 0, \exists! y\in\mathbb{K}:a\cdot y = 1\)\hfill Inverse
\end{enumerate}
\item Verträglichkeit
\begin{enumerate}
\item \(a\cdot (b + c) = (a\cdot b)+ (a\cdot c)\)\hfill Distributivität
\end{enumerate}
\end{itemize}
\paragraph{Satz}
\label{sec-2-4-3-1}
$(\mathbb{Q},+,\cdot)$ ist ein Körper. Definieren auf $\mathbb{Q}$ eine Ordnung "$\leq$" duch \[x\leq y \Leftrightarrow\exists m\in\mathbb{N}_0, n\in\mathbb{N}:y - x = \frac{m}{n}\]
dann ist auch diese Ordnung mit der Addition und Multiplikation in $\mathbb{Q}$ in folgendem Sinne verträglich (Axiom M0):
\begin{itemize}
\item \(a\leq b\Rightarrow a+c \leq b + c\)
\item \(0\leq a\wedge 0\leq b \Rightarrow 0\leq a\cdot b\)
\end{itemize}
\paragraph{Bemerkung}
\label{sec-2-4-3-2}
\[\{a\in\mathbb{Q}: a = \frac{r}{s},r\in\mathbb{N}_0,s\in\mathbb{N}\} =: \mathbb{Q}_+ (\mathbb{Q}_{\geq 0})\]
\subsection{Abzählbarkeit}
\label{sec-2-5}
\subsubsection{Abzählbarkeit von Mengen}
\label{sec-2-5-1}
Sei $A$ eine Menge\\
\begin{itemize}
\item $A$ heißt endlich mit $\abs{A} = n$ Elementen ist äquivalent zu
\[\abs{A} = \begin{cases} A = \emptyset &(n = 0) \\ \exists f:A\to \{1,\ldots,n\} &f~\text{bijektiv},n < \infty\end{cases}\]
\item $A$ heißt abzählbar unendlich genau dann wenn \[\exists f: A\to \mathbb{N}~\text{bijektiv}\]
\item $A$ heißt überabzählbar genau dann wenn: $A$ ist weder endlich oder abzählbar unendlich
\end{itemize}
\paragraph{Beispiel}
\label{sec-2-5-1-1}
$\mathbb{Z}$ ist abzählbar unendlich
\subparagraph{Beweis}
\label{sec-2-5-1-1-1}
Die Abbildung $f:\mathbb{Z}\to\mathbb{N}$
\[z\mapsto \begin{cases} 2z & z\geq 0\\ -2z - 1 & x < 0\end{cases}\]
\begin{itemize}
\item Surjektivität: zu zeigen $f(\mathbb{Z}) = \mathbb{N}$ \\
           Offenbar $f(\mathbb{Z}) \subseteq \mathbb{N}$. Wir zeigen $\mathbb{N} \subseteq f(\mathbb{Z})$. Sei $n\in\mathbb{N}$, finde $z\in\mathbb{Z}$ mit $f(z) = n$.
Man unterscheide:
\begin{itemize}
\item n gerade $\rightarrow$ Wähle $z=\frac{n}{2}$
\item n ungerade $\rightarrow$ $z=-\frac{n + 1}{2}$
\end{itemize}
\item Injektivität: Sei $z_1,z_2 \in\mathbb{Z}$ und $f(z_1) = f(z_2)$ \\
           ohne Beschränkung der Allgemeinheit $z_1 \leq z_2$. Entweder $z_1,z_2 \geq 0$ oder $z_1,z_2 < 0$, denn sonst währe $f(z_1)$ ungerade und $f(z_1)$ gerade \textbf{Wiederspruch}.
Falls
\begin{itemize}
\item $z_1,z-2 \geq 0 \Rightarrow 2z_1 = f(z_1) = f(z_2) = 2z_2 \Rightarrow z_1 = z_2$
\item $z_1,z-2 < 0 \Rightarrow -2z_1 -1 = f(z_1) = f(z_2) = -2z_2-1 \Rightarrow z_1 = z_2 \hfill\square$
\end{itemize}
\end{itemize}
\paragraph{Beispiel}
\label{sec-2-5-1-2}
\begin{itemize}
\item $\mathbb{N}^2 = \mathbb{N}\times\mathbb{N}$ abzählbar unendlich
\item $\mathbb{Q}$ abzählbar unendlich
\item $\mathbb{R}$ überabzählbar
\end{itemize}
\paragraph{Abzählbarkeit von $\mathbb{N}\times\mathbb{N}$}
\label{sec-2-5-1-3}
\[(1,1) \to (1,2) \to (2,1) \to (2,2) \to (1,3) \to (2,3) \to (3,2) \to (3,1)\]
\paragraph{Korollar 1.30}
\label{sec-2-5-1-4}
$M_1,M_2,\ldots,M_n$ abzählbar $\Rightarrow M_1 \times \ldots \times M_n$ abzählbar.
\subparagraph{Beweis}
\label{sec-2-5-1-4-1}
Durch vollständige Induktion $M_1\times(M_2\times\ldots \times M_n)\approx \mathbb{N}\times\mathbb{N}\approx\mathbb{N}$
\paragraph{Satz}
\label{sec-2-5-1-5}
Die Menge aller Folgen $f:\mathbb{N}\to\{0,1\}$ ist überabzählbar. (Zum Beispiel: $1,0,0,0,\ldots, \underarrow[1]{\text{k-te Stelle}},\ldots,0,\ldots$)
\subparagraph{Beweis}
\label{sec-2-5-1-5-1}
$M$ ist unendlich, denn die Folgen $f_k:0,,\ldots,0,1,0,\ldots$ sind parrweise verschieden. Angenommen $M$ wäre abzählbar. Sei $f_1,f_2, \ldots$ eine Abzählung mit $f_k = ({z_{kn}}_{n\in \mathbb{N}})$.
\[\begin{matrix}1 & 0 & 0 & \ldots \\ 0 & 1 & \ldots \\ 0 & 0 & 0 & \ldots \\ 1 & 1 & 1 & 1 & \ldots \end{matrix}\]
$f:0 0 1 0$ Man setze $f=(z_n)_{n\in\mathbb{N}}$ mit \[z_n := \begin{cases} 1 & z_{nn} = 0 \\ 0 & z_{nn} = 1\end{cases}\]
Dann $f\in M$, aber $f\neq f_k \Forall k\in\mathbb{N}$. Also ist $M$ nicht abzählbar. ("Cantorsche Diagonalverfahren").
\subsection{Ordnung}
\label{sec-2-6}
\subsubsection{Definition}
\label{sec-2-6-1}
Sei $A$ eine Menge. Relation $R\subseteq A\times A$ heißt Teilordnung (Halbordnung) auf $A$, wenn $\Forall y,x,z\in A$ gilt:
\begin{enumerate}
\item $x\leq x$ \hfill (Reflexivität)
\item $x\leq y \wedge y\leq x \Rightarrow x = y$ \hfill (Symmetrie)
\item $x\leq y \wedge y\leq z \Rightarrow x\leq z$ \hfill (Transitivität)
\end{enumerate}
Wenn außerdem noch $\Forall x,y\in A$ gilt:
\begin{enumerate}
\setcounter{enumi}{3}
\item $x\leq y \vee y\leq x$ \hfill (Vergleichbarkeit je zweier Elemente)
\end{enumerate}
so heißt $R$ (totale) Ordung auf $A$. \$(A,$\le$) heißt teilweise beziehungsweise (total) geordnete Menge.
\paragraph{Beispiel}
\label{sec-2-6-1-1}
\begin{enumerate}
\item $(\mathbb{Q},\leq)$ mit der üblichen Ordnung ist eine total geordnete Menge
\item Wir definieren auf der Potenzmenge $\mathcal{P}(A)$ einer Menge $A$ eine Teilordnung "$\leq$": \[B\leq C \Leftrightarrow B \subseteq C\Forall B,C\in \mathcal{P}(A)\] \\
        \textbf{Beweis}: 1. - 3. sind trivial, 4. geht nicht (keine Totalordung). Wähle $B,C\in \mathcal{P}(a), B,C\neq \emptyset, B\cap C = \emptyset$. Dann gilt weder $B\subseteq C$ noch $C\subseteq B\hfill\square$
\item Sei $F:=\{f\mid f:A\to\mathbb{R}\}$ für eine Menge $A\subseteq \mathbb{R}$. Wir definieren $f\leq g \Leftrightarrow \Forall x\in A: f(x) \leq g(x)$ \\
        (1.) - (3.) trivial, 4. gilt nicht. Falls $A$ mehr als ein Element hat, gibt es eine Funktion, die nicht miteinander verglichen werden können.
\end{enumerate}
\subsection{Maximum und Minimum einer Menge}
\label{sec-2-7}
\subsubsection{Definition}
\label{sec-2-7-1}
Sei $(A,\leq)$ eine teilweise geordnete Menge, $a\in A$ \\
    Maximum:
\[a = \max A \Leftrightarrow \Forall x\in A: x\leq a\]
Minimum:
\[a = \max A \Leftrightarrow \Forall x\in A: a\leq x\]
\subsubsection{Bemerkung}
\label{sec-2-7-2}
Durch die Aussagen ist $a$ eindeutig bestimmt, denn seien:
\[a_1,a_2\in A:\Forall x\in A \begin{cases}x\leq a_1 \\ x\leq a_2 \end{cases} \Rightarrow \begin{cases} a_2 \leq a_1 \\ a_1 \leq a_2 \end{cases} \xRightarrow{\text{Symmetrie}} a_1 = a_2 \]
\subsection{Schranken}
\label{sec-2-8}
Sei $(A,\leq)$ eine (total geordnete) Menge, $B\subseteq A$
\begin{enumerate}
\item $S\in A$ heißt obere Schranke zu $B \Leftrightarrow \Forall x\in B: x\leq S$ \\
      $S\in A$ heißt untere Schranke zu $B \Leftrightarrow \Forall x\in B: S\leq x$
\item $\bar{S}(B):= \{S\in A \mid S~\text{S ist untere Schranke zu}~B\}$ \\
      $\ubar{S}(B):= \{S\in A \mid S~\text{S ist obere Schranke zu}~B\}$
\item Existiert $g:=\min \ubar{S}(B)$ beziehungsweise $g:=\max \bar{S}$ so sagen wir: \\
      $g = \sup B$ (kleinste obere Schranke, \uline{supremum}, obere "Grenze" von $B$ in $A$)
$g = \inf B$ (größte obere Schranke, \uline{infimum}, untere "Grenze" von $B$ in $A$)
\end{enumerate}
\subsubsection{Bemerkung}
\label{sec-2-8-1}
\begin{enumerate}
\item Existiert $\max B = \bar{b}$, so folt $\sup B = \bar{b}$, denn $\bar{b} \in \ubar{S}(B)$ nach Definition.
\[s\in \ubar{S}(B) \Rightarrow \bar{b} \leq s,~\text{da}~\bar{b}\in B\]
Ebeso gilt: $\exists\min B = \ubar{b} \Rightarrow \inf B = \ubar{b}$
\end{enumerate}
\subsubsection{Beispiel}
\label{sec-2-8-2}
\begin{enumerate}
\item $B = \{\frac{1}{n}\mid n\in\mathbb{N}\}, A = \mathbb{R},~(1, \frac{1}{2},\ldots)$
\begin{itemize}
\item Es gilt $1\in B, \Forall n\in\mathbb{N}$ gilt $\frac{1}{n} \leq 1$, daher folgt $\max B = \sup B = 1$
\item Sei $s\leq 0$, dann gilt $\Forall n\in\mathbb{N}: s\leq \frac{1}{n}$, also $s\in \bar{S}(B)$ \\
         Sei $s > 0 \Rightarrow s > \frac{1}{n} \Leftrightarrow n > \frac{1}{s}$, also $s\not\in\bar{S}(B)$ \\
         Es folgt $\bar{S}(B) = \{x\in\mathbb{R}\mid s\leq 0\}$ insbesondere $0\in\bar{S}(B)$ \\
         Ferner gilt $\Forall s\in \bar{S}(B):s\leq 0 \Rightarrow \ubar{0} = \max \bar{S}(B) = \inf B$
\end{itemize}
\item $A = \mathbb{Q}, B = \{x\in\mathbb{Q} : 0 \leq x \wedge x^2 \leq 2\}$. Es gilt $0 = \min B = \inf B$, aber $\sup B$ existiert nicht in $\mathbb{Q}$
\end{enumerate}
\subsection{Reelle Zahlen}
\label{sec-2-9}
$x^2 = 2$ hat keine Lösungen in $\mathbb{Q}$. Allerdings können wir $\sqrt{2}$ "beliebig gut" durch $y\in \mathbb{Q}$ approximieren, das heißt $\Forall \varepsilon > 0\exists y\in\mathbb{Q}:2 - \varepsilon \leq y^2 \leq 2 + \varepsilon$
Das motiviert die folgende Vorstellung:
\begin{enumerate}
\item $\mathbb{Q}$ ist "unvollständig"
\item $\mathbb{Q}$ ist "dicht" in $\mathbb{R}$
\end{enumerate}
\subsubsection{Vollständigkeitsaxiom (Archimedes)}
\label{sec-2-9-1}
Jede nach oben (unten) beschränkte Teilmenge hat ein Supremum oder Infimum.
\subsubsection{Axiomatischer Standpunkt}
\label{sec-2-9-2}
Es gibt eine Menge $\mathbb{R}$ (genannt Menge der reellen Zahlen) mit Addition, Multiplikation, Ordung, die die Definition eines Körper und das Vollständigkeitsaxiom erfüllt und $(\mathbb{R},+,\cdot)$ mit "$\leq$" eine Ordung bildet.
\subsubsection{Bemerkung}
\label{sec-2-9-3}
\begin{enumerate}
\item Bis auf Isomorphie gibt es höchstens ein solches $\mathbb{R}$, das heißt $\tilde{\mathbb{R}}$ ein weiteres System der reellen Zahlen ist, dann $\exists$ bijektive Abbildung $f:\mathbb{R}\to\tilde{\mathbb{R}}$ die bezüglich Additoin, Multiplikation, Ordung eine Homomorphie ist.
\[\Forall x,y\in \mathbb{R}:\]
\[f(x+y) = f(x) + f(y)\]
\[f(x y) = f(x) f(y)\]
\[x\leq y \Rightarrow f(x) \leq f(y)\]
\item $\mathbb{N}$ (und damit auch $\mathbb{Z},\mathbb{Q}$) lassen sich durch injektive Homomorphismus $g:\mathbb{N}\to\mathbb{R}$ in $\mathbb{R}$ einbetten
\[g(\tilde{0}_{\in\mathbb{N}}) = 0_{\in\mathbb{R}}\]
\[g(\tilde{n}_{\in\mathbb{N}} + 1) = g(n_{\in\mathbb{R}}) + 1\]
\[g(1_{\in\mathbb{N}}) = 1_{\in\mathbb{R}}\]
\end{enumerate}
\subsubsection{Konstruktiver Standpunkt}
\label{sec-2-9-4}
Wir können $\mathbb{R}$ ausgehend von $\mathbb{Q}$ konstruieren.
\paragraph{Methode der Abschnitte}
\label{sec-2-9-4-1}
Jede reelle Zahl wird charakterisiert durch ein "rechts offenes, unbeschränktes Interval", dessen "rechte Grenze" die Zahl erstellt.
\[\mathbb{R}:=\{A\subseteq \mathbb{Q}\begin{cases}A\neq\emptyset \\ x\in A, y\leq x\Rightarrow y\in A \\ \Forall x\in A\exists y\in A, x<y\end{cases}\]
\paragraph{Mehtode der Cauchy-Folgen}
\label{sec-2-9-4-2}
Jede reelle Zahl wird charaktierisiert als "Grenzwert" eine Klasser äquivalenter "Cauchy Folgen" aus $\mathbb{Q}$ (später)
\subsubsection{Definition 1.37}
\label{sec-2-9-5}
\begin{itemize}
\item \[x\in \mathbb{R}~\text{heißt}~\begin{cases}\text{positiv} & 0 < x \\ \text{nichtnegativ} & 0\leq x \\ \text{negativ} & x < 0 \\ \text{nichtpositiv} & x\geq 0 \end{cases}\]
\item Die Betragsfunktion $\abs \cdot:\mathbb{R} \to \mathbb{R}$ wird definiert durch $\abs{x} = \max \{x,-x\} = \begin{cases} x & x \geq 0 \\ -x & x < 0\end{cases}$
\item Die Vorzeichen- oder Signumfunktion \[\sgn:\mathbb{R}\to\mathbb{R},\sgn{x} = \begin{cases}\frac{x}{\abs{x}} & x \neq 0 \\ 0 & x = 0\end{cases} = \begin{cases} 1 & x > 0 \\ -1 & x < 0 \\ 0 & x = 0\end{cases}\]
\end{itemize}
\subsubsection{Satz 1.38}
\label{sec-2-9-6}
\begin{enumerate}
\item $\abs{x y} = \abs{x} \abs{y}$
\item $\abs{x + y} \leq \abs{x} + \abs{y}$ \\
       \textbf{Beweis:} \\
\begin{align}
\abs{x + y}^2 &= (x+y)^2 = x^2 + 2x y + y^2 = \abs{x}^2 + 2xy + \abs{y}^2 \\
&\leq \abs{x}^2 + 2\abs{x y} + \abs{y}^2 = \abs{x}^2 + 2\abs{x}\abs{y} + \abs{y^2} \\
&= (\abs{x} + \abs{y})^2 \Rightarrow \abs{x + y} \leq \abs{\abs{x} + \abs{y}} = \abs{x} + \abs{y} \tag{$\square$}
\end{align}
\item $\abs{x + y} = \abs{x} + \abs{y} \Leftrightarrow x y \geq 0$
\end{enumerate}
\subsubsection{Satz 1.39}
\label{sec-2-9-7}
\begin{enumerate}
\item $\abs{\abs{x} - \abs{y}} \leq \abs{x - y}$ \\
       \textbf{Beweis:} \\
\begin{align}
\abs{x} &= \abs{x - y + y} \leq \abs{x - y} + \abs{y} \Rightarrow \abs{x} - \abs{y} \leq \abs{x - y} \\
\abs{y} &= \abs{y - x + x} \leq \abs{y - x} + \abs{x} \Rightarrow \abs{y} - \abs{x} \leq \abs{x - y} \\
\abs{\abs{x} - \abs{y}} &= \max \{\abs{x} - \abs{y},\abs{y} - \abs{x}\} \leq \abs{x - y} \tag{$\square$}
\end{align}
\item \[\abs{x - y} \leq \varepsilon \Leftrightarrow \begin{cases} x - \varepsilon \leq y \leq x + \varepsilon \\ y - \varepsilon \leq x \leq y + \varepsilon \end{cases}\]
       \textbf{Beweis:} \\
\begin{align}
\abs{x - y} = \max\{x - y , y - x\} \leq \varepsilon \Leftrightarrow \begin{cases} x - y \leq \varepsilon \\ y - x \leq \varepsilon\end{cases} \Leftrightarrow \begin{cases} x \leq y + \varepsilon \\ y - x \leq \varepsilon \end{cases} \Leftrightarrow y - \varepsilon \leq x \leq y + \varepsilon
\end{align}
Vertausche $x$ und $y$ $\Rightarrow$ $x - \varepsilon \leq x + \varepsilon \hfill \square$
\end{enumerate}
\subsubsection{Definition 1.40}
\label{sec-2-9-8}
Sei $a,b\in\mathbb{R},a\leq b$
\begin{itemize}
\item $[a,b]:=\{x\in\mathbb{R}: a\leq x \leq b\}$ \hfill abgeschlossenes Intervall
\item $(a,b):= \{x\in\mathbb{R}: a < x < b\} = ]a,b[$ \hfill offenes Intervall
\item $[a,b) := \{x\in\mathbb{R}:a\leq x < b\}$ \hfill rechts-halboffenes Intervall
\item $(a,b]:=\{x\in\mathbb{R}:a<x\leq b\}$ \hfill links-halboffenes Intervall
\item $\varepsilon > 0, I_\varepsilon (x) := (x -\varepsilon,x + \varepsilon) = \{y\in\mathbb{R}:\abs{x - y} < \varepsilon = B_\varepsilon (x) (\text{Kugel})\}$
\end{itemize}
\subsubsection{Lemma 1.41}
\label{sec-2-9-9}
Es gilt $y\in I_\varepsilon (x) \Rightarrow \exists \delta > 0: I_\delta (y) \subseteq I_\varepsilon (x)$
\paragraph{Beweis}
\label{sec-2-9-9-1}
Sei $y\in I_\varepsilon (x) \Rightarrow \abs{x - y}  < \varepsilon \Leftrightarrow \varepsilon - \abs{x - y} > 0$
Wähle $0 < \delta < \varepsilon - \abs{x - y}$. Es ist nun zu zeigen $I_\delta (y) \subseteq I_\varepsilon (x)$, das heißt
$z\in I_\delta(y) \Rightarrow z\in I_\varepsilon(x)$. Es gilt
\begin{align}
&z\in I_\delta (y) \Rightarrow \abs{z - y}  <\delta \\
\Rightarrow &\abs{z - x} = \abs{z - y + y - x} \leq \abs{z - y} + \abs{y - x} \leq \delta + \abs{x - y} < \varepsilon \\
\Rightarrow &z\in I_\varepsilon (x) \tag{$\square$}
\end{align}
\subsubsection{Definition 1.42}
\label{sec-2-9-10}
$A,B$ seien geordnete Mengen, $f:A\to B$ heißt:
\begin{itemize}
\item monoton $\begin{cases} \text{wachsed} & x \leq y \Rightarrow f(x) \leq f(y)  \\ \text{fallend} & x \leq y \Rightarrow  f(x) \leq f(y) \end{cases}$
\item streng monoton $\begin{cases} \text{wachsend} & x < y \Rightarrow f(x) < f(y) \\ \text{fallend} & x < y \Rightarrow f(x) > f(y) \end{cases}$
\end{itemize}
\paragraph{Beispiel 1.43}
\label{sec-2-9-10-1}
$\mathbb{R}_+\setminus \{0\} \to \mathbb{R}_+\setminus\{0\}, x\mapsto x^n$ ist streng monoton wachsend $\Forall n\in\mathbb{N}$
\subparagraph{Beweis}
\label{sec-2-9-10-1-1}
Induktion + Axiom M0 $\hfill\square$
\subsubsection{Lemma 1.44}
\label{sec-2-9-11}
Sei $M,N \subseteq \mathbb{R}, f:M\to N$ streng monoton und bijektiv. Dann ist $f^{-1}$ streng monoton.
\paragraph{Beweis}
\label{sec-2-9-11-1}
Wir betrachten den Fall $f$ streng monoton wachsend. Seien \$y$_{\text{1}}$,y$_{\text{2}}$ $\in$ N, y$_{\text{1}}$ < y$_{\text{2}}$,x$_{\text{1}}$ = f$^{\text{-1}}$(y$_{\text{1}}$), x$_{\text{2}}$ = f$^{\text{-1}}$(y$_{\text{2}}$). \\
     Behauptung $x_1 < x_2$ (sonst wäre \$x$_{\text{1}}$ $\ge$ x$_{\text{2}}$). \\
     Falls $x_1 > x_2 \xRightarrow{\text{streng monoton}} f(x_2) > f(x_2)$ \textbf{Widerspruch} zu $y_1 < y_2$ \\
     Falls $x_1 = x_2 \Rightarrow y_1 = y_2$ \textbf{Widerspruch} zur Annahme $y_1 < y_2 \hfill \square$
\subsubsection{Definition 1.45 Produktzeichen}
\label{sec-2-9-12}
Für $a\in\mathbb{R},n\in\mathbb{N}$ definieren wir $a^n := \prod_{j=1}^n a$ und für $a\in\mathbb{R}\setminus\{0\},n\in\mathbb{N}$ $a^{-n} := \frac{1}{a^n}$.
\subsubsection{Satz 1.46}
\label{sec-2-9-13}
Es gilt $\Forall a,b\in\mathbb{R}$ (beziehungsweise $\mathbb{R}\setminus \{0\}$),$n,m\in\mathbb{N}_0$ (beziehungsweise $\mathbb{Z}$)
\begin{enumerate}
\item $a^n a^m = a^{n+m}$
\item $(a^n)^m$ = a$^{\text{n m}}$\$
\item $(ab)^m = a^m b^m$
\end{enumerate}
\paragraph{Beweis}
\label{sec-2-9-13-1}
Zunächst f+r $n,m\in\mathbb{N}_0$ durch Indukton nach $n$, dann für $n,m\in\mathbb{Z}$ (mit Hilfe der Definition von $a^{-n}$)
\subsubsection{Definition 1.47}
\label{sec-2-9-14}
Sei $n,k\in\mathbb{N}_0$ \[\binom{n}{k}:=\prod_{j=1}^k \frac{n -j + 1}{j}\]
\subsubsection{Lemma 1.48}
\label{sec-2-9-15}
Sei $k,n\in\mathbb{N}_0$
\begin{enumerate}
\item $\binom{n}{k} = 0$ für $k > n$ \\
       $\binom{n}{k} = \frac{n!}{k!(n -k)!} = \binom{n}{n - k}$ für $k\leq n$
\item $\binom{n}{k} = \binom{n - 1}{k - 1} + \binom{n - 1}{k}$ für $1 \leq k \leq n$
\end{enumerate}
\subsubsection{Satz 1.49}
\label{sec-2-9-16}
$\Forall n\in\mathbb{N}_0,\Forall x,y\in\mathbb{R}$ gilt
\[(x + y)^n = \sum_{j = 0}^n \binom{n}{j}x^{n - j}y^j\]
\paragraph{Beweis}
\label{sec-2-9-16-1}
Induktion:
\begin{itemize}
\item Induktionsanfang: $n = 0,(x+y)^0 = 1,\binom{0}{j}x^0y^0 = 1$ nach Definition
\item Induktionsschritt $n \to n + 1:$
\begin{align}
&(x + y)^{n + 1} = (x +y)(x+y)^n \\
\xRightarrow{\text{Induktonsvoraussetung}} &(x + y)\sum_{j = 0}^n \binom{n}{j}x^{n - j} y^j \\
= &\sum_{j=0}^n \binom{n}{j}x^{n - j + 1}y^j + \sum_{j = 0}^n \binom{n}{j} x^{n - j} y^{j + 1} \\
= &\binom{n}{0}x^{n + 1} + \sum_{j = 1}^n \binom{n}{j}x^{n + 1 - j}y^j + \underbrace{\sum_{i = 1}^n \binom{n}{i - 1} x^{n - i + 1} y^i}_{\text{Substitution $i:= j + 1$}} + \binom{n}{n}y^{n + 1} \\
= &x^{n + 1} + \sum_{j = 1}^n \underbrace{(\binom{n}{j} + \binom{n}{j - 1})}_{\binom{n + 1}{j}\text{nach Lemma 1.48}} x^{n + 1 - j} y^j + y^{n + 1} \\
= &\sum{j = 0}^{n + 1} \binom{n + 1}{j}x^{n + 1 - j}y^j\tag{$\square$}
\end{align}
\end{itemize}
\subsubsection{Folgerung 1.50}
\label{sec-2-9-17}
\begin{enumerate}
\item $\sum_{j = 0}^n \binom{n}{j} = 2^n$
\item $\sum_{j = 0}^n \binom{n}{j} (-1)^j = \begin{cases}0 & n \neq 0 \\ 1 & n = 0\end{cases}$
\end{enumerate}
\paragraph{Beweis:}
\label{sec-2-9-17-1}
Setze in Binomische Formel $x = 1, y = 1$ beziehungsweise $y = -1 \hfill\square$
\subsubsection{Lemma 1.51}
\label{sec-2-9-18}
Sei $m\in R$ nach oben (beziehungsweise nach unten) beschränkt \\
    Dann gilt
\begin{enumerate}
\item $s = \sup M \Leftrightarrow \Forall \varepsilon > 0 \exists x\in M: s - \varepsilon < x (\geq s)$
\item $l=\inf M \Leftrightarrow \Forall \varepsilon > 0 \exists x\in M: (l \leq) x < l + \varepsilon$
\end{enumerate}
\paragraph{Beweis}
\label{sec-2-9-18-1}
Wir beweisen 1. \\
     $s\neq \sup M\Leftrightarrow s$ ist nicht die kleinste obere Schranke von $m$ $\Leftrightarrow$ es gibt eine kleinere obere Schranke $s' = s - \varepsilon$ von $M$ $\Leftrightarrow$
nicht $\Forall \varepsilon > 0\exists x\in M: x > s - \varepsilon \hfill \square$
\subsubsection{Lemma 1.52}
\label{sec-2-9-19}
$\mathbb{N}$ ist unbeschränkt in $\mathbb{R}$
\paragraph{Beweis}
\label{sec-2-9-19-1}
sonst $\exists x = \sup \mathbb{N}$ (nach Vollständigkeits Axiom), $x$ kleinste obere Schranke $\xRightarrow{\text{[[Lemma 1.51]]}} \varepsilon= \frac{1}{2} \exists m_o \in \mathbb{N}:x - \frac{1}{2} < m_0 \Rightarrow m_0 + 1 \in \mathbb{N},m_0 + 1 > x + \frac{1}{2} > x$
$\Rightarrow$ $x$ inst nicht die obere Schranke von $\mathbb{N}\hfill\square$
\subsubsection{Lemma 1.53 (Bernoullische Ungleichung)}
\label{sec-2-9-20}
\label{Lemma-1.53}
\[\Forall x\in [-1,\infty),n\in\mathbb{N}_0: (1 + x)^n \geq 1 + n x\]
\paragraph{Beweis}
\label{sec-2-9-20-1}
Beweis durch Induktion:
\begin{itemize}
\item \textbf{IA}: $n = 0$ klar
\item \textbf{IS}:
\begin{align}
n\to n + 1: (1 + x)^{n + 1} &= (1 + x)^n(1 + x) \\
&\geq (1 + n x) (1 + x) = 1 + nx^2 + (n + 1) x \\
&\geq 1 + (n + 1) x~\text{da $x^2 \geq 0$} \tag{\text{$\square$}}
\end{align}
\end{itemize}
\subsubsection{Folgerung 1.54}
\label{sec-2-9-21}
\begin{enumerate}
\item Sei $y\in(1,\infty)$. Dann gilt $\Forall c > 0 \exists n_0 \in \mathbb{N},\Forall n\geq n_0 y^n \in (c,\infty)$ ("Konvergenz" von $y^n$ gegen 0)
\item Sei $y \in (-1,1)$. Dann gilt $\Forall \varepsilon > 0\exists n_0 \in \mathbb{N}\Forall n\geq n_0:y^n \in I_\varepsilon (0)$ ("Konvergenz" $y^n$ gegen 0)
\end{enumerate}
\paragraph{Beweis}
\label{sec-2-9-21-1}
\begin{enumerate}
\item \label{1541} Für $x = y - 1 > 0$ gilt dann nach \ref{Lemma-1.53} \[\underbrace{(1 + x)^n}_y \geq 1 + n x \Rightarrow y^n > n x\]
        Nach \ref{sec-2-9-19} existiert für $c > 0$ ein $n_0 \in \mathbb{N}$ mit $n_0 > \frac{c}{x} \Rightarrow$
        \[\Forall n\geq n_0: y^n > n x \geq n_0 x \geq \frac{c}{x} x = c \Rightarrow \Forall n\geq n_0: y^n\in(c,\infty)\]
\item Für  $x = \frac{1}{\abs{y}} > 1 \xRightarrow{\text{nach [[1541]] mit } c = \frac{1}{\varepsilon}}$
        \[\Forall \varepsilon > 0\exists n_0\in\mathbb{N}\Forall n\geq n_0: x^n > \frac{1}{\varepsilon}\]
        \[\Rightarrow \frac{1}{\abs{y^n}} > \frac{1}{\varepsilon} \Rightarrow \abs{y^n} < \varepsilon \hfill \square\]
\end{enumerate}
\subsubsection{Satz 1.55 (Existenz der m-ten Wurzel)}
\label{sec-2-9-22}
\[\Forall m\in\mathbb{N}, a\in[a,\infty)~\text{gilt}~\exists ! x\in [0,\infty): x^m = a\]
\paragraph{Beweis (Skizze 1, 2)}
\label{sec-2-9-22-1}
Wir geben ein Iterationsverfahren
\[p_3 (x) = m\]
\[a_3 x^3 + a_2 x^2 + a_1 x + a_0, a_3 > 0\]
Ohne Beschränkung der Allgemeinheit $a > 0,m\geq 2$, $x$ muss die Gleichung $x^m -a = 0$ lösen, das heißt Nullstelle der Funktion $f:[0,\infty) \to \mathbb{R},x\mapsto x^m - a$ suchen.
Diese approximieren wir nach dem \textbf{Newton Verfahren} \\
     $x_0$ sodass $x_0^m - a \geq 0$
\[x_n - x_{n + 1} = \frac{f(x_n)}{f'(x_n)} \Leftarrow \frac{f(x_n)}{x_n - x_{n + 1}} = f'(x_n)\]
\[x_{n+1} := \underbrace{x_n - \frac{f(x_n)}{f'(x_n)}}_{F(x_n)} = x_n - \frac{x_n^m - a}{m x_n^{m - 1}}\]
\[= x_n(1 - \frac{1}{m}(1 - \frac{a}{x_n^m}))\]
Hoffnung: $x_n \to x^*$

\texttt{Skizze 3}

Sei $x_0^m > a$. Wir zeigen
\begin{enumerate}
\item \label{1.55.1} $x_n > 0$
\item \label{1.55.2} $x_n^m \geq a$
\item $x_{n + 1} \leq x_n$
\end{enumerate}
\textbf{Beweis:}
\begin{enumerate}
\item Induktion
\item Induktion
\begin{itemize}
\item $n = 0, x_0^m \geq \Rightarrow x_0 > 0$, da $a > 0,x_0\geq 0$
\item $n\to n + 1$ \[x_n > 0, x_n^m\geq a \Rightarrow x_{n + 1} = x_n(1 - \frac{1}{m}(1 - \frac{a}{x_n^m})) \geq 0\]
          weil \[x_{n + 1}^n = \underbrace{x_n^m}_{\geq 0} (1 - \frac{1}{m}(1 - \frac{a}{x_n^m}))^m \underbrace{\geq}_{\text{Bernoulli}} x_n^m(1 - \frac{1}{m}(1 - \frac{a}{x_n^m}))  = 0\]
          $\Rightarrow$ $x_{n + 1} > 0$, da $a > 0$
\end{itemize}
\item Nach \ref{1.55.2}: \[x_n^m \geq a \Rightarrow 0 \leq 1 - \frac{1}{m}(1 - \frac{1}{x_n^m}) \leq 1\]
        Nach \ref{1.55.1}: \[x_m > 0 \Rightarrow x_{n + 1} = x_n(1 - \frac{1}{m}(1-\frac{a}{x_n^m})) < x_n\]
        Wegen \ref{1.55.1} ist $M = \{x_n:n\in\mathbb{N}_0\}$ nach unten beschränkt $\Rightarrow$
        \[x:= \inf M~\text{existiert}\]
        Wir wollen zeigen, dass $x^m = a$. Es gilt \[x \leq x_{n + 1} = (1 - \frac{1}{m})x_n + \frac{1}{m}\frac{a}{x^{m -1}_n}\]
        \[\leq (1 - \frac{1}{m})x_n + \frac{a}{m}\sup \{\frac{1}{x_n^{m - 1}\mid x\in\mathbb{N}_0}\} \]
\item \label{1.55.4} Es gilt nach nach \ref{1.56.2}
\[a\leq \inf \{x_n^m \mid n\in \mathbb{N}_0\} = (\inf \{x_n \mid n\in \mathbb{N}_0\})^m = x^m\]
und damit $x > 0$ \\
        Ferner gilt
\[y = \sup \{\frac{1}{x_n^{m - 1}} \mid n\in\mathbb{N}_0\} = \inf \{x_n^{m - 1}\mid x \in \mathbb{N}_0\}^{-1}\]
mit \ref{sec-2-9-23} \[= (\frac{1}{\inf\{x_n \mid n\in\mathbb{N}_0\}})^{m - 1} = \frac{1}{x^{m -1}} \Rightarrow a y\leq \frac{a}{x^{m - 1}}\]
\item \label{1.55.5} Von oben wissen wir, dass $x \leq a y$
        \[\Rightarrow x\leq a y  \leq \frac{a}{x^{m -1}} \Rightarrow x^m \leq a\]
\end{enumerate}

Aus \ref{1.55.4} und \ref{1.55.5} folgt $x^m = a\hfill\square$

\subsubsection{Lemma 1.56}
\label{sec-2-9-23}
\begin{enumerate}
\item Seien für $n\in \mathbb{N}_0:y_n > 0$ und $\inf \{x_n\mid x\in\mathbb{N}_0\} > 0$ \\
       Dann gilt \[\sup \{\frac{1}{y_n} \mid n\in \mathbb{N}_0\} = \frac{1}{\inf \{y_n \mid n\in\mathbb{N}_0\}}\]
\item \label{1.56.2} Seien für $n\in\mathbb{N}_0,y_n > 0, k\in\mathbb{N}_0$. Dann gilt:
\[\inf \{y_n^k \mid n\in\mathbb{N}_0\} = (\inf \{y_n\mid n\in\mathbb{N}_0\})^k\]
\end{enumerate}
(ohne Beweis)

\section{Komplexe Zahlen}
\label{sec-3}
\textbf{Motivation:} $x^2 + 1 = 0$ nicht lösbar in $\mathbb{R}$ \\
   Wir betracheten die Menge der Paare $\{x,y\} = \mathbb{R}\times\mathbb{R}$ auf denen die Addition und Multiplikation wie folgt definiert ist:
\begin{itemize}
\item \label{KA} (KA) $\{x_1,y_1\} + \{x_2,y_2\} = \{x_1 + x_2, y_2 + y_2\}$
\item \label{KM} (KM) \$\{x$_{\text{1}}$,y$_{\text{1$\backslash$}}$\} $\cdot$ \{x$_{\text{2}}$,y$_{\text{2$\backslash$}}$\} = \{x$_{\text{1}}$ x$_{\text{2}}$ - y$_{\text{1}}$ y$_{\text{2}}$, x$_{\text{1}}$ y$_{\text{2}}$ + x$_{\text{2}}$ y$_{\text{1$\backslash$}}$\}
\end{itemize}
\subsection{Komplexer Zahlkörper}
\label{sec-3-1}
\begin{enumerate}
\item Die Menge der Paare $z = \{x,y\} \in \mathbb{R}\times\mathbb{R}$ mit Addition \ref{KA} und Multiplikation \ref{KM} bildet den Körper $\mathbb{C}$ der \textbf{komplexen Zahlen} mit den neutralen Elementn $\{0,0\}$ und $\{1,0\}$
\item Die Gleichung $z^2 + \{1,0\} = \{0,0\}$ hat in $\mathbb{C}$ zwei Lösungen, welche mit $\I:= \{0,\pm 1\}$ bezeichnet werden
\item Der Körper $\mathbb{R}$ ist mit der Abbildung $x\in\mathbb{R}:x\mapsto\{x,0\}\in\mathbb{C}$ isomorph zu einem Unterkörper von $\mathbb{C}$
\end{enumerate}
\subsubsection{Beweis}
\label{sec-3-1-1}
\begin{enumerate}
\item Die Gültigkeit des Kommutativitäts-, Assoziativs-, und Distributibitätsgesetzes verifiziert man durch Nachrechenen. \\
       Neutrale Elemente: Wir lösen die Gleichung $a + z = \{0,0\}$ für beliebige gegebene $a\in\mathbb{C},a=\{a_1,a_2\}$
       \[\Rightarrow z = \{-a_1, -a_2\}\]
       \[a\cdot z = \{1,0\}\]
       \[z = \frac{1}{a}:=\{\frac{a_1}{a_1^2 + a_2^2},-\frac{a_2}{a_1^2 + a_2^2}\},~\text{weil}~a\cdot\frac{1}{a}\]
       \[\text{weil}~a\frac{1}{a}=\{a_1\frac{a_1}{a_1^2 + a_2^2} + \frac{a_2^2}{a_1^2 + a_2^2},\frac{a_1 a_2}{a_1^2 + a_2^2} - \frac{a_2 a_1}{a_1^2 + a_2^2}\}\]
\item $i:= \{0,1\}$ hat die Eigenschaft
\[1 + \I^2 = \{1,0\} + \{0^2 - 1^2, 0\} = \{0,0\} \Rightarrow 1+\I^2 = 0\]
Ähnlich $1 + (-\I)^2 = 0$
\item Die Zuordnung $x\in\mathbb{R}:x\mapsto\{x,0\}\in\mathbb{C}$ bildet $\mathbb{R}$ bijektiv auf eine Untermenge von $\mathbb{C}$ ab, welche bezüglich der komplexen Addition und Multiplikation wieder ein Körper ist $\hfill\square$
\end{enumerate}
\subsection{Notation}
\label{sec-3-2}
$z = \{x,y\} =: x + \I y,~x,y\in\mathbb{R}$
\begin{itemize}
\item $x$ ist Realteil $x = \Re{z}$
\item $y$ ist Imaginärteil $x = \Im{z}$
\end{itemize}
\[z_1 + z_2 = (x_1 + \I y_1) + (x_2 + \I y_2) = \underbrace{x_1 + x_2}_{\Re (z_1 + z_2)} + \I\underbrace{(y_1 + y_2)}_{\Im (z_1 + z_2)}\]
\[z_1 z_2 = (x_1 + \I y_1)(x_1 + \I y_2) = x_1 x_2 + \I y_1 x_2 + \I y_2 x_1 + (\I y_1)(\I y_2) = \underbrace{x_1 x_2 - y_1 y_2}_{\Re (z_1 z_2)} + \I\underbrace{(x_1y_2 + y_1 x_2)}_{\Im (z_1,z_2)}\]
\subsection{{\bfseries\sffamily TODO} Graphische Darstellung}
\label{sec-3-3}
\subsection{Bemerkung}
\label{sec-3-4}
Die reellen Zahlen  sind durch $\Im z = 0$ charakterisiert.
\[z_1 = z_2 \Rightarrow x_1 + \I y_i = x_2 + \I y_2 \Leftrightarrow x_1 = x_2,y_1 = y_2\]
\subsection{Korollar 1.59}
\label{sec-3-5}
Jede quadratische Gleichung
\[z^2+p z + q = 0,~p,q\in\mathbb{R}\]
besitzt in $\mathbb{C}$ genau zwei Lösungen
\[z_1,2 = \begin{cases} -\frac{1}{2}\pm \frac{1}{2}\sqrt{p^2 - 4q} & p^2 \geq 4q \\ -\frac{1}{2}\pm \I\frac{1}{2}\sqrt{\abs{p^2 - 4q}} & p^2 - 4q < 0 \end{cases} \]
\subsection{Fundamentalsatz der Algebra}
\label{sec-3-6}
Jede algebraische Gleichung der Form \[z^n + \sum_{i = 0}^{n - 1} a_i z^i = 0\]
hat in $\mathbb{C}$ mindestens eine Lösung. Beweis $\rightarrow$ Funktionstheorie
\subsection{Betrag}
\label{sec-3-7}
Für komplese Zahlen lässt sich ein Absolutbetrag definieren
\[r = \abs{z} = \sqrt{x^2 + y^2}\]
Damit:
\begin{align}
x &= r\cos{\alpha}
y &= r\sin{\alpha}
z &= x + \I y = r(\cos{alpha} + \I\sin{\alpha})
\end{align}
\subsection{Konjugation}
\label{sec-3-8}
Zu einem $z = x + \I y\in\mathbb{C}$ definieren wir eine konjugierte komplexe Zahl \[\bar z = x - \I y \in\mathbb{C}\]
Dann gilt \[\abs{z}^2 = x^2 + y^2 = z\bar z\]
Aus der Definition:
\begin{itemize}
\item $\overline{z_1 + z_2} = \overline{z_1} + \overline{z_2}$
\item $\overline{z_1 * z_2} = \overline{z_1} * \overline{z_2}$
\item $x = \frac{z + \bar z}{2}$
\item $y = \frac{z - \bar z}{2\I}$
\end{itemize}
\section{Folgen}
\label{sec-4}
Eine Folge von reellen Zahlen wird gegeben durch eine Abbildung \[\mathbb{N}_0 \to \mathbb{R},n\mapsto x_n\]
Wir bezeichnen die Folge auch mit $(x_n)_{n\in\mathbb{N}_0}$

Topologische Struktur auf Mengen.
\begin{itemize}
\item Abstände in $\mathbb{R}^1$ Betrag $\abs{x - y}$ $\xrightarrow{\text{Verallgemeinerung}}$ Norm / Metrik
\item Umgebung in $\mathbb{R}^1$ $\varepsilon$-Intervall $\xrightarrow{\text{Verallgemeinerung}}$ Kugel Umgebung
\end{itemize}

Wir betrachten Folgen $\mathbb{N}\to\mathbb{R}, n\mapsto a_n$ (oder $\mathbb{C}$)
\subsection{Definition 2.1 Konverenz}
\label{sec-4-1}
Wir sagen, dass die Folge $(a_n)_{n\in\mathbb{N}}$ in $\mathbb{K}$ ($\mathbb{R}$ oder $\mathbb{C}$) gegen den Grenzwert (oder Limes) $a\in\mathbb{K}$ konvergiert
\[a_n \xrightarrow{n\to\infty} a~(a=\lim_{n\to\infty} a_n)\]
wenn für beliebiges $\varepsilon > 0$ von einem $n_\varepsilon \in\mathbb{N}$ an gilt
\[\abs{a_n - a} < \varepsilon,n\geq n_\varepsilon\]
\[\Leftrightarrow \Forall\varepsilon > 0\exists n\varepsilon \in\mathbb{N}:\Forall n\geq n_\varepsilon a_n \in I_\varepsilon(a)\]
\subsection{Folgerung 2.2}
\label{sec-4-2}
Sei $(a_n)_{n\in\mathbb{N}}$ eine monoton wachsende beziehungsweise fallende Folge reeller Zahlen $M=\{a_n\mid n\in\mathbb{N}\}$ und sei nach oben beziehungsweise unten beschränkt. Dann gilt \[a_n\to\sup M, a_n\to\inf M\]
Beweis $\to$ Übungen
\subsection{Definition 2.3 Cauchy Folgen}
\label{sec-4-3}
Eine Folge $(a_n)_{n\in\mathbb{N}}$ heißt Cauchy-Folge wenn:
\[\Forall\varepsilon > 0\exists n_\varepsilon \in\mathbb{N}\Forall n,m \geq n_\varepsilon: \abs{a_n - a_m} < \varepsilon \]
(Cauchy Kriterium)
\subsection{Definition 2.4 Teilfolge}
\label{sec-4-4}
Eine Teilfolge einer gegebenen Folge $(a_n)_{n\in\mathbb{N}}$ ist eine Auswahl $(a_{n_k})_{k\in\mathbb{N}}$, wobei $a_{n_k}$ auch die Glieder von $(a_n)_{n\in\mathbb{N}}$ sind
\subsubsection{Beispiel 2.5}
\label{sec-4-4-1}
\[a_n = \frac{1}{m}\] ist eine Cauchy-Folge. Für ein $\varepsilon > 0$ wählen wir $n_\varepsilon$ so dass $n_\varepsilon > \frac{1}{\varepsilon}$. Für beliebiges $n\geq m > N$
\[\abs{a_m - a_n} = \abs{\frac{1}{m} - \frac{1}{n}} = \frac{n - m}{m n} \leq \frac{n}{m n} = \frac{1}{m} < \frac{1}{n_\varepsilon} < \varepsilon\hfill\square\]
% Emacs 25.1.1 (Org mode 8.2.10)
\end{document}