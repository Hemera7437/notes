% Created 2016-10-19 Mi 10:48
\documentclass[11pt]{article}
\usepackage[utf8]{inputenc}
\usepackage[T1]{fontenc}
\usepackage{fixltx2e}
\usepackage{graphicx}
\usepackage{longtable}
\usepackage{float}
\usepackage{wrapfig}
\usepackage{rotating}
\usepackage[normalem]{ulem}
\usepackage{amsmath}
\usepackage{textcomp}
\usepackage{marvosym}
\usepackage{wasysym}
\usepackage{amssymb}
\usepackage{hyperref}
\tolerance=1000
\usepackage{siunitx}%
\usepackage{fontspec}%
\sisetup{load-configurations = abbrevations}%
\newcommand{\estimates}{\overset{\scriptscriptstyle\wedge}{=}}%
\usepackage{mathtools}%
\DeclarePairedDelimiter\abs{\lvert}{\rvert}%
\DeclarePairedDelimiter\norm{\lVert}{\rVert}%
\DeclareMathOperator{\Exists}{\exists}%
\DeclareMathOperator{\Forall}{\forall}%
\author{Robin Heinemann}
\date{\today}
\title{Analysis I (Marciniak-Czochra)}
\hypersetup{
  pdfkeywords={},
  pdfsubject={},
  pdfcreator={Emacs 25.1.1 (Org mode 8.2.10)}}
\begin{document}

\maketitle
\tableofcontents


\section{Einleitung}
\label{sec-1}
Webseite www.biostruct.uni-heidelberg.de/Analysis1.php
Klausurzulassung: 50\%
Klausur 18.2.2017 9-12Uhr
\section{Mengen und Zahlen}
\label{sec-2}
\subsection{Logische Regeln und Zeichen}
\label{sec-2-1}
\subsubsection{Quantoren}
\label{sec-2-1-1}
\begin{center}
\begin{tabular}{ll}
$\Forall x$ & für alle $x$\\
$\exists x$ & es gibt (mindestens) ein $x$\\
$\exists! x$ & es gibt genau ein $x$\\
\end{tabular}
\end{center}
\subsubsection{Hinreichend und Notwendig}
\label{sec-2-1-2}
\begin{itemize}
\item $A\Rightarrow B$: wenn $A$ gilt, gilt auch $B$, $A$ ist \textbf{hinreichend} für $B$, daraus folgt: $B$ ist \textbf{notwendig} für $A$, Ungültigkeit von $B$ impliziert die Ungültigkeit von $A$ ($\neg B \Rightarrow \neg A$)
\item $A \Leftrightarrow B$: $A$ gilt, genau dann, wenn $B$ gilt
\end{itemize}
\subsubsection{Beweistypen}
\label{sec-2-1-3}
\begin{enumerate}
\item Direkter Schluss
\label{sec-2-1-3-1}
$A\Rightarrow B$
\begin{enumerate}
\item Beispiel
\label{sec-2-1-3-1-1}
$m$ gerade Zahl $\Rightarrow$ $m^2$ gerade Zahl
\begin{enumerate}
\item Beweis
\label{sec-2-1-3-1-1-1}
$m$ gerade $\Rightarrow \exists n\in\mathbb{N}~\text{sodass}~m = 2n \Rightarrow m^2 = 4n^2 = 2k,~\text{wobei}~k=2n^2\in\mathbb{N} \square$
\end{enumerate}
\end{enumerate}
\item Beweis der Transponerten (der Kontraposition)
\label{sec-2-1-3-2}
Zum Beweis $A\Rightarrow B$ zeigt man $\neg B\Rightarrow \neg A~(A\Rightarrow B)\Leftrightarrow (\neg B) \Rightarrow (\neg A)$
\begin{enumerate}
\item Beispiel
\label{sec-2-1-3-2-1}
Sei $m\in\mathbb{N}$, dann gilt $m^2~\text{gerade}~\Rightarrow m~\text{gerade}$
\begin{enumerate}
\item Beweis
\label{sec-2-1-3-2-1-1}
Wir zeigen: $m$ ist ungerade $\Rightarrow m^2$ ungerade
\[\exists n\in\mathbb{N}:~m=2n+1\Rightarrow m^2 = (2n+1)^2 = 2k+1, k=2n^2 + 2n\in\mathbb{N}\Rightarrow m^2~\text{ungerade} \square\]
\end{enumerate}
\end{enumerate}
\item Indirekter Schluss ( Beweis durch Wiederspruch)
\label{sec-2-1-3-3}
Man nimmt an, dass $A\Rightarrow B$ nicht gilt, das heißt $A \wedge \neg B$ und zeigt, dass dann für eine Aussage $C$ gelten muss $C\Rightarrow \neg C$, also ein Wiederspruch
\begin{enumerate}
\item Beispiel
\label{sec-2-1-3-3-1}
$\not\exists q\in\mathbb{Q}: a^2 = 2$
\begin{enumerate}
\item Beweis
\label{sec-2-1-3-3-1-1}
Wir nehmen an, dass $\exists a\in\mathbb{Q}: a^2=2$ Dann folgt:
$\exists b,c\in\mathbb{Z}$ teilfremd (ohne Einschränkung, denn sonst kürzen soweit wie möglich) mit $a=\frac{b}{c}$
Falls \[a^2=2\Rightarrow (\frac{b}{c})^2=2=\frac{b^2}{c^2}=2 \Rightarrow b^2 = 2c^2 \Rightarrow b^2~\text{gerade}~\Rightarrow b~\text{ist gerade (schon gezeight)}~\Rightarrow\exists d\in\mathbb{N}~\text{sodass}~b=2d\Rightarrow b^2=4d^2\]
Außerdem $b^2=2c^2\Rightarrow 2c^2=4d^2\Rightarrow c^2=2d^2\Rightarrow c$ ist auch gerade. Also müssen $b$ und $c$ beide gerade sein, also nicht teilerfremd, damit haben wir einen Widerspruch hergeleitet $\square$
\end{enumerate}
\end{enumerate}
\end{enumerate}
\subsubsection{Summenzeichen und Produktzeichen}
\label{sec-2-1-4}
\begin{enumerate}
\item Summenzeichen
\label{sec-2-1-4-1}
Wir definieren für $m > 0$ \[\sum_{k=m}^m a_k := a_m + \ldots + a_n\] falls $n\geq m$
\[\sum_{k=m}^n a_k := 0\] falls $n < m$ (sogennante leere Summe)
\item Produktzeichen
\label{sec-2-1-4-2}
\[\prod_{k=m}^n a_k := \begin{cases} a_m \cdot ... \cdot a_n & \text{falls}~n\geq m\\ 1 & \text{falls}~n<m~\text{(sog. leeres Produkt)}\end{cases}\]
\end{enumerate}
\subsection{Mengen}
\label{sec-2-2}
\subsubsection{Definition}
\label{sec-2-2-1}
(Georg cantor 1885) Unger einer \uline{Menge} verstehen wir jede Zusammenfassung $M$ von bestimmten wohlunterschiedenen Objekten (welche die Elemente von $M$ genannt werden), zu einem Ganzen
$M$ dadurch ist charakterisiert, dass von jedem vorliegendem Objekt $x$ feststeht, ab gilt
\begin{itemize}
\item $x\in M$ (x Element von M)
\item x \textlnot{}$\in$ M (x kein Element von M)
\end{itemize}
\[M = \{x_1, x_2, \ldots, x_n\}\]
\[M=\{x\mid A(x)\} \rightarrow~\text{eine Menge}~M\text{für die}~x\in\ M \Leftrightarrow A(x)\]
\subsubsection{Mengenrelationen}
\label{sec-2-2-2}
\begin{itemize}
\item Mengeninklusion $A\subset M$ ($A$ ist eine Teilmenge von $M$)
\[\Forall x: (x\in A \Rightarrow x\in M)\], zum Beispiel $\mathbb{N} \subset \mathbb{Z}$
\end{itemize}
% Emacs 25.1.1 (Org mode 8.2.10)
\end{document}