% Created 2016-10-20 Do 11:20
\documentclass[11pt]{article}
\usepackage[utf8]{inputenc}
\usepackage[T1]{fontenc}
\usepackage{fixltx2e}
\usepackage{graphicx}
\usepackage{longtable}
\usepackage{float}
\usepackage{wrapfig}
\usepackage{rotating}
\usepackage[normalem]{ulem}
\usepackage{amsmath}
\usepackage{textcomp}
\usepackage{marvosym}
\usepackage{wasysym}
\usepackage{amssymb}
\usepackage{hyperref}
\tolerance=1000
\usepackage{siunitx}
\usepackage{fontspec}
\sisetup{load-configurations = abbrevations}
\newcommand{\estimates}{\overset{\scriptscriptstyle\wedge}{=}}
\usepackage{mathtools}
\DeclarePairedDelimiter\abs{\lvert}{\rvert}%
\DeclarePairedDelimiter\norm{\lVert}{\rVert}%
\DeclareMathOperator{\Exists}{\exists}
\DeclareMathOperator{\Forall}{\forall}
\def\colvec#1{\left(\vcenter{\halign{\hfil$##$\hfil\cr \colvecA#1;;}}\right)}
\def\colvecA#1;{\if;#1;\else #1\cr \expandafter \colvecA \fi}
\author{Robin Heinemann}
\date{\today}
\title{Lineare Algebra (Vogel)}
\hypersetup{
  pdfkeywords={},
  pdfsubject={},
  pdfcreator={Emacs 25.1.1 (Org mode 8.2.10)}}
\begin{document}

\maketitle
\tableofcontents


\section{Einleitung}
\label{sec-1}
Übungsblätter/Lösungen:
jew. Donnerstag / folgender Donnerstag
Abgabe Donnerstag 9:30
50\% der Übungsblätter
\subsection{Plenarübung}
\label{sec-1-1}
Aufgeteilt
\subsection{Moodle}
\label{sec-1-2}
Passwort: vektorraumhomomorphismus
\subsection{Klausur}
\label{sec-1-3}
24.02.2017
\section{Grundlagen}
\label{sec-2}
\subsection{Naive Aussagenlogik}
\label{sec-2-1}
naive Logik: wir vewenden die sprachliche Vorstellung ($\neq$ mathematische Logik: eigne Vorlesung)
Eine Aussage ist ein festehender Satz, dem genau einer der Wahrheitswerte "wahr" oder "falsch" zugeordnet werden kann.
Aus einfachen Aussagen kann man durch logische Verknüpfungen kompliziertere Aussagen bilden.
Angabe der Wahrheitswertes der zusammengesetzten Aussage erfolgt duch Wahrheitstafeln (liefern den Wahrheitswert der zusammengesetzten Aussage, aus dem Wahrheitswert der einzelnen Aussagen).
Im folgenden seien $A$ und $B$ Aussagen.
\begin{itemize}
\item Negation (NICHT-Verknüpfung)
\begin{itemize}
\item Symbol: \$$\neg{}$
\item Wahrheitstafel:
\begin{center}
\begin{tabular}{ll}
$A$ & $\neg A$\\
\hline
w & f\\
f & w\\
\end{tabular}
\end{center}
\item Beispiel: $A$: 7 ist eine Primzahl (w)
$\neg A$: 7 ist keine Primzahl (f)
\end{itemize}

\item Konjunktion (UND-Verknüpfung)
\begin{itemize}
\item Symbol $\wedge$
\item Wahrheitstafel:
\begin{center}
\begin{tabular}{lll}
$A$ & $B$ & $A\wedge B$\\
\hline
w & w & w\\
w & f & f\\
f & w & f\\
f & f & f\\
\end{tabular}
\end{center}
\end{itemize}

\item Disjunktion (ODER-Verknüpfung)
\begin{itemize}
\item Symbol: $\vee$
\item Wahrheitstafel:
\begin{center}
\begin{tabular}{lll}
$A$ & $B$ & $A\vee B$\\
\hline
w & w & w\\
w & f & w\\
f & w & w\\
f & f & f\\
\end{tabular}
\end{center}
\item exklusives oder: $(A\vee B) \wedge (\neg(A\wedge B))$
\end{itemize}
\item Beispiel $A$: 7 ist eine Primzahl (w), $B$: 5 ist gerade (f)
\begin{itemize}
\item $A\wedge B$ 7 ist eine Primzahl und 5 ist gerade (f)
\item $A\vee B$ 7 ist eine Primzahl oder 5 ist gerade (w)
\end{itemize}

\item Implikation (WENN-DANN-Verknüpfung)
\begin{itemize}
\item Symbol: $\Rightarrow$
\item Wahrheitstafel:
\begin{center}
\begin{tabular}{lll}
$A$ & $B$ & $A\Rightarrow B$\\
w & w & w\\
w & f & f\\
f & w & w\\
f & f & w\\
\end{tabular}
\end{center}
\item Sprechweise: $A$ impliziert $B$, aus $A$ folgt $B$, $A$ ist eine hinreichende Bedingung für $B$ (ist $A\Rightarrow B$ wahr, dann folgt aus $A$ wahr, $B$ ist wahr), $B$ ist eine notwendige Bedingung für $A$ (ist $A\Rightarrow B$ wahr, dann kann $A$ nur dann wahr sein, wenn Aussage $B$ wahr ist)
\item Beispiel Es seinen $m,n\in\mathbb{N}$
\begin{itemize}
\item $A$: m ist gerade
\item $B$: $mn$ ist gerade
\item Dann gilt $\Forall m,n \in\mathbb{N}~A\Rightarrow B~\text{wahr}$ \\
         Fallunterscheidung:
\begin{itemize}
\item $m$ gerade, $n$ gerade, dann ist $A$ wahr, $B$ wahr, d.h. $A\Rightarrow B$ wahr
\item $m$ gerade, $n$ ungerade, dann ist $A$ wahr, $B$ wahr, d.h. $A\Rightarrow B$ wahr
\item $m$ ungerade, $n$ gerade, dann ist $A$ falsch, $B$ wahr, d.h. $A\Rightarrow B$ wahr
\item $m$ ungerade, $n$ ungerade, dann ist $A$ falsch, $B$ falsh, d.h. $A\Rightarrow B$ wahr
\end{itemize}
\end{itemize}
\end{itemize}
\item Äquivalenz (GENAU-DANN-WENN-Verknüpfung)
\begin{itemize}
\item Symbol $\Leftrightarrow$
\item Wahrheitstafel:
\begin{center}
\begin{tabular}{lll}
$A$ & $B$ & $A\Leftrightarrow$ B\\
\hline
w & w & w\\
w & f & f\\
f & w & f\\
f & f & w\\
\end{tabular}
\end{center}
\item Sprechweise: $A$ gilt genau dann, wenn $B$ gilt, $A$ ist hinreichend und notwendig für $B$ \\
       Die Aussagen $A\Leftrightarrow B$ und $(A\Rightarrow B)\wedge (B\Rightarrow A)$ sind gleichbedeutend:
\begin{center}
\begin{tabular}{llllll}
$A$ & $B$ & $A\Leftrightarrow B$ & $A\Rightarrow B$ & $B\Rightarrow A$ & $(A\Rightarrow B)\wedge (B\Rightarrow A)$\\
\hline
w & w & w & w & w & w\\
w & f & f & f & w & f\\
f & w & f & w & f & f\\
f & w & f & w & f & f\\
f & f & w & w & w & w\\
\end{tabular}
\end{center}
\item Beispiel: Es sei $n$ eine ganze Zahl \\
       $A:~n-2>1$ \\
       $B:~n>3$ \\
       $\Forall n\in\mathbb{N}~\text{gilt}~A\Leftrightarrow B$
       $C:~n>0$ \\
       $D:~n^2>0$ \\
       Für $n=-1$ ist die Äquivalenz $C\Leftrightarrow$ falsch ($C$ falsch, $D$ wahr) \\
       Für alle ganzen Zahlen $n$ gilt zumindest die Implikation $C\Rightarrow D$
\end{itemize}
\end{itemize}
\subsection{Beweis}
\label{sec-2-2}
Mathematische Sätze, Bemerkungen, Folgerungen, etc. sind meistens in Form wahrer Implikationen formuliert
\subsubsection{beweisen}
\label{sec-2-2-1}
Begründen warum diese Implikation wahr ist
\subsubsection{Beweismethoden for diese Implikation $A\Rightarrow B$}
\label{sec-2-2-2}
\begin{itemize}
\item direkter Beweis ($A\Rightarrow B$)
\item Beweis durch Kontraposition ($\neq B \Rightarrow \neg A$)
\item Widerspruchbeweis ($\neg (A\wedge \neg B)$)
\end{itemize}
Diese sind äquivalent zueinander
\begin{center}
\begin{tabular}{lllllll}
$A$ & $B$ & $\neg A$ & $\neg B$ & $A\Rightarrow B$ & $\neg B \Rightarrow \neg A$ & $\neg (A \wedge \neg B)$\\
\hline
w & w & f & f & w & w & w\\
w & f & f & w & f & f & f\\
f & w & w & f & w & w & w\\
f & f & w & w & w & w & w\\
\end{tabular}
\end{center}
\begin{enumerate}
\item Beispiel
\label{sec-2-2-2-1}
$m,n$ natürliche Zahlen \\
     \[A:~m^2 < n^2\]
\[B:~m < n\]
Wir wollen zeigen, dass $A\Rightarrow B$ für alle natürlichen Zahlen $m,n$ wahr ist
\begin{itemize}
\item direkter Beweis: \\
       \[A:~m^2 < n^2 \Rightarrow 0 < n^2 - m^2 \Rightarrow 0 < (n-m)\underbrace{(n+m)}_{>0} \Rightarrow 0 < n-m \Rightarrow m<n\]
\item Beweis durch Kontraposition: \\
       \[\neg B:~m \geq n \Rightarrow m^2\geq n m \wedge m n \geq n^2 \Rightarrow m^2 \geq n^2 \Rightarrow \neg A\]
\item Beweis durch Widerspruch: \\
       \[A\wedge \neg B \Rightarrow m^2 < n^2 \wedge n\leq m \Rightarrow m^2 < n^2 \wedge m n \leq m^2 \wedge n^2 \leq m n \Rightarrow m n \leq m^2 < n^2 \leq m n\]
       Wiederspruch
\end{itemize}
\end{enumerate}
\subsection{Existenz- und Allquantor}
\label{sec-2-3}
\subsubsection{Existenzquantor}
\label{sec-2-3-1}
\$A(x) Aussage, die von Variable x abhängt \\
    $\exists x:~A(x)$ ist gleichbedeutend mit "Es existiert ein x, für das $A(x)$ wahr ist" (hierbei ist "existiert ein x" im Sinne von "existiert mindestens ein x" zu verstehen) \\
    Beispiel:
\[\exists n\in\mathbb{N}:~n>5\quad\text{(w)}\]
$\exists !x:~A(x)$ ist gleichbedeutend mit "Es existiert genau ein x, für dass $A(x)$ wahr ist" $\backslash$
\subsubsection{Allquantor}
\label{sec-2-3-2}
$\Forall x:~A(x)$ ist gleichbedeutend mit "Für alle x ist A(x) wahr"
Beispiel:
\[\Forall n\in\mathbb{N}: 4n~\text{ist gerade}\]
\subsubsection{Negation von Existenz- und Allquantor}
\label{sec-2-3-3}
\[\neg(\exists x:~A(x)) \Leftrightarrow \Forall x:~\neg A(x)\]
\[\neg(\Forall x:~A(x)) \Leftrightarrow \exists x:~\neg A(x)\]
\subsubsection{Spezielle Beweistechniken für Existenz und Allaussagen}
\label{sec-2-3-4}
\begin{itemize}
\item Angabe eines Beispiel, um zu zeigen, dass deine Existenzaussage wahr ist. \\
      Beispiel:
\[\exists n\in\mathbb{N}:~n>5 \text{ist wahr, denn für $n = 7$ ist die Aussage $n > 5$ wahr}\]
\item Angabe eines Gegenbeispiel, um zu zeigen, dass eine Allausage falsch ist. \\
      Beispiel:
\[\Forall n\in\mathbb{N}:~n\leq 5 \text{ist flasch, dann für $n=7$ ist die Aussage $n\leq 5$ falsch}\]
\end{itemize}
\subsection{Naive Mengenlehre}
\label{sec-2-4}
Mengenbegriff nach Cantor: \\
   Eine Menge ist eine Zusammenfassung von bestimmten, wohlunterschiedenen Objekten userer Anschauung oder useres Denkens (die Elemente genannt werden) zu einem Ganzen

\subsubsection{Schreibweise}
\label{sec-2-4-1}
\begin{itemize}
\item $x\in M$, falls $x$ ein Element von $M$ ist
\item $x\not\in M$, falls $x$ kein Element von $M$ ist
\item $M=N$, falls $M$ und $N$ die gleichen Elemente besitzen, $M\subseteq N \wedge N\subseteq M$
\end{itemize}

\subsubsection{Angabe von Mengen}
\label{sec-2-4-2}
\begin{itemize}
\item Reihenfolge ist unrelevant (\$\{1,2,3\}=\{1,3,2\})
\item Elemente sind wohlunterschieden $\{1,2,2\} = \{1,2\}$
\item Auflisten der Elemente $M=\{a,b,c,\ldots\}$
\item Beschreibung der Elemente durch Eigenschaften: $M=\{x\mid E(x)\}$ \\
     (Elemente x, für die E(x) wahr)
\begin{itemize}
\item Beispiel:
\[\{2,4,6,8\} = \{x\mid x\in\mathbb{N}, x~\text{gerade}, 1 < x < 10\}\]
\end{itemize}
\end{itemize}
\subsubsection{leere Menge}
\label{sec-2-4-3}
Die leere Menge $\emptyset$ enthält keine Elemente
\begin{enumerate}
\item Beispiel
\label{sec-2-4-3-1}
\[\{x\mid x\in\mathbb{N}, x < -5\} = \emptyset\]
\end{enumerate}

\subsubsection{Zahlenbereiche}
\label{sec-2-4-4}
Menge der natürlichen Zahlen:
\[\mathbb{N} := \{1,2,3,\ldots\}\]

Menge der natürlichen Zahlen mit Null:
\[\mathbb{N}_0 := \{0, 1,2,3,\ldots\}\]

Menge der Ganzen Zahlen:
\[\mathbb{Z} := \{0,1,-1,2,-2\}\]

Menge der rationalen Zahlen:
\[\mathbb{Q} := \{\frac{m}{n} \mid m\in\mathbb{Z}, n\in\mathbb{N}\}\]

Menge der reellen Zahlen: $\mathbb{R}$

\subsubsection{Teilmenge}
\label{sec-2-4-5}
$A,B$ seien Mengen. \\
    $A$ heißt Teilmenge von $B~(A\subseteq B) \xLeftrightarrow{\text{Def.}} \Forall x\in A: x\in B$
$A$ heißt echte Teilmenge von $B~(A\subset B) \xLeftrightarrow{\text{Def.}} A\subseteq B \wedge A\neq B$
\begin{enumerate}
\item Anmerkung
\label{sec-2-4-5-1}
Offenbar gilt für Mengen $A,B$:
\[A=B \Leftrightarrow A\subseteq B \wedge B\subseteq A\]
$\emptyset$ ist Teilmenge jeder Menge

\item Beipspiel
\label{sec-2-4-5-2}
\[\mathbb{N}\subset\mathbb{N}_0\subset\mathbb{Z}\subset\mathbb{Q}\]
\end{enumerate}

\subsubsection{Durschnitt}
\label{sec-2-4-6}
\[A \cap B := \{x\mid x\in A \wedge x\in B\}\]
\begin{enumerate}
\item Beispiel
\label{sec-2-4-6-1}
\[A=\{2,3,5,7\}, B=\{3,4,6,7\}, A\cap B = \{3,7\}\]
\end{enumerate}

\subsubsection{Vereinigung}
\label{sec-2-4-7}
\[A\cup B := \{x\mid x\in A \vee x\in B\}\]
\begin{enumerate}
\item Beispiel
\label{sec-2-4-7-1}
\[A=\{2,3,5,7\}, B=\{3,4,6,7\}, A\cup B = \{2,3,4,5,6,7\}\]
\end{enumerate}

\subsubsection{Differenz}
\label{sec-2-4-8}
\[A\setminus B := \{x\mid x\in A \wedge x\not\in B\}\]
Im Fall $B\subseteq A$ nennt man $A\setminus B$ auch das Komplement von $B$ in $A$ und schreibt $\mathcal{c}_A(B) = A\setminus B$
\begin{enumerate}
\item Beispiel
\label{sec-2-4-8-1}
\[A=\{2,3,5,7\}, B=\{3,4,6,7\}, A\setminus B = \{2,5\}\]
\end{enumerate}
\subsubsection{Bemerkung zu Vereinigung und Durschnitt}
\label{sec-2-4-9}
$A,B$ seien zwei Mengen. Dann gilt \[A\cap (B\cup C) = (A\cap B) \cup (A\cap C)\]
\begin{enumerate}
\item Beweis
\label{sec-2-4-9-1}
\[A\cap(B\cup C) \subseteq (A\cap B) \cup (A\cap C)\]
\[A\cap(B\cup C) \supseteq (A\cap B) \cup (A\cap C)\]
"$\subseteq$" Sei $x\in A \cap (B\cup C)$. Dann ist $x\in A \wedge x\in B\cup C$
\begin{itemize}
\item 1. Fall: $x\in A \wedge x\in B$
       \[\Rightarrow x\in A\cap B \Rightarrow x \in (A\cap B) \cup (A\cap C)\]
\item 2. Fall $x\in A \wedge x\in C$
       \[\Rightarrow x\in A\cap C \Rightarrow x\in (A\cap B)\cup(A\cap C)\]
\end{itemize}
Damit ist "$\subseteq$" gezeigt.
"$\supseteq$" Sei \$x$\in$ (A$\cap$ B) $\cup$ (A$\cap$ C)
\[\Rightarrow x\in A\cap B \vee x\in A\cap C \\ \Rightarrow (x\in A \wedge x\in B) \vee (x\in A \wedge x\in C) \\ \Rightarrow x\in A \wedge (x\in B\vee x\in C) \\ \Rightarrow x\in A \wedge x\in B\cup C \\ \Rightarrow x\in A\cap (B\cup C)\]
Damit ist "$\supseteq$" gezeigt.
\end{enumerate}
\subsubsection{Bemerkung zu Äquivalenz von Mengen}
\label{sec-2-4-10}
Seien $A,B$ Mengen, dann sind äquivalent:
\begin{enumerate}
\item $A\cup B = B$
\item $A\subseteq B$
\end{enumerate}
\begin{enumerate}
\item Beweis
\label{sec-2-4-10-1}
Wir zeigen 1) $\Rightarrow$ 2) und 2) $\Rightarrow$ 1.
\[1) \Rightarrow 2):~\text{Es gelte}~A\cup B = B,~\text{zu zeigen ist}~A\subseteq B \\ \text{Sei}~x\in A \Rightarrow x\in A \wedge x \in B \Rightarrow x\in A\cup B = B\]
\[2) \Rightarrow 1):~\text{Es gelte}~A\subseteq B\text{, zu zeigen ist}~A\cup B = B \]
"$\subseteq$": Sei $x\in A\cup B \Rightarrow x\in A \vee x\in B \xRightarrow{A\subseteq B} x\in B$
"$\supseteq$": $B\subseteq A\cup B$ klar
\end{enumerate}
\subsubsection{Kartesisches Produkt}
\label{sec-2-4-11}
Seien $A,B$ Mengen
\[A\times B := \{(a,b)\mid a\in A, b\in B\}\]
heipt das kartesische Produkt von $A$ und $B$. Hierbei ist $(a,b) = (a',b') \xLeftrightarrow{\text{Def}} a = a' \wedge b = b'$ a = a' $\wedge$ b = b'\$

\begin{enumerate}
\item Beispiel
\label{sec-2-4-11-1}
\begin{itemize}
\item \[\{1,2\}\times \{1,3,4\} = \{(1,1),(1,3),(1,4),(2,1),(2,3),(2,4)\}\]
\item \[\mathbb{R}\times\mathbb{R}=\{(x,y)|mid x,y \in \mathbb{R}\} = \mathbb{R}^2\]
\end{itemize}
\end{enumerate}
\subsubsection{Potenzmenge}
\label{sec-2-4-12}
$A$ sei eine Menge
\[\mathcal{P} (A) := \{M\mid M\subseteq A\}\]
heißt die Potenzmenge von $A$
\begin{enumerate}
\item Beispiel
\label{sec-2-4-12-1}
\[\mathcal{P} (\{1,2,3\}) =  \{\emptyset, \{1\}, \{2\},\{3\},\{1,2\},\{1,3\},\{2,3\}\{1,2,3,4\}\}\]
\end{enumerate}
\subsubsection{Kardinalität}
\label{sec-2-4-13}
$M$ sei eine Menge. Wir setzen
\[\abs{M} := \begin{cases} n & \text{falls $M$ eine endliche Menge ist und $n$ Elemente enthält} \\ \infty & \text{falls $M$ nicht endlich ist} \end{cases}\]
$\abs{M}$ heißt Kardinalität von A
\begin{enumerate}
\item Beispiel
\label{sec-2-4-13-1}
\begin{itemize}
\item $\abs{\{7,11,16\}} = 3$
\item $\abs{\mathbb{N}} = \infty$
\end{itemize}
\end{enumerate}
\subsubsection{Bemerkung zu natürlichen Zahlen}
\label{sec-2-4-14}
Für die natürlichen Zahlen gilt das Induktionsaxiom
Ist $M\subseteq N$ eine Teilmenge, für die gilt:
\[1\in M \wedge \Forall n\in M : n\in M \Rightarrow n+1 \in M\]
dann ist $M = \mathbb{N}$
\subsubsection{Prinzip der vollständigen Induktion}
\label{sec-2-4-15}
Für jedes $n\in \mathbb{N}$ sei eine Aussage $A(n)$ gegeben. Die Aussagen $A(N)$ gelten für alle $n\in\mathbb{N}$, wenn man folgendes zeigen kann: \\
\begin{itemize}
\item (IA) $A(1)$ ist wahr
\item (IS) Für jedes $n\in\mathbb{N}$ gilt: $A(n) \Rightarrow A(n+1)$
\end{itemize}
Der Schritt (IA) heißt Induktionsanfang, die Implikation $A(n) \Rightarrow A(n+1)$ heißt Induktionsschritt
\begin{enumerate}
\item Beweis
\label{sec-2-4-15-1}
Setze $M := \{n\in \mathbb{N}\mid A(n)~\text{ist wahr}\}$
Wegen (IA) ist $1\in M$, wegen (IS) gilt: \$n$\in$ M $\Rightarrow$ n+1 $\in$ M\$\\
     Nach Induktionsaxiom folgt $M = \mathbb{N}$, das heißt $A(n)$ ist wahr für alle $n\in \mathbb{N}$
\item Beispiel
\label{sec-2-4-15-2}
Für $n\in\mathbb{N}$ sei $A(n)$ die Aussage: $1+\ldots + n = \frac{n(n+1)}{2}$
Wir zeigen: $A(n)$ ist wahr für alle $n\in \mathbb{N}$, und zwar durch vollständige Induktion
\begin{itemize}
\item (IA) $A(1)$ ist wahr, denn $1 = \frac{1(1+1)}{2}$
\item (IS) zu zeigen: $A(n) \Rightarrow A(n+1)$ \\
       Es gelte $A(n)$, das heißt $1+\ldots+n = \frac{n(n+1)}{2}$ ist wahr \[\Rightarrow 1 + \ldots + n + (n + 1) = \frac{n(n+1)}{2} + (n+1) =  \frac{n(n+1) + 2(n+1)}{2} = \frac{(n+1)(n+2)}{2} \square\]
\end{itemize}
\end{enumerate}
% Emacs 25.1.1 (Org mode 8.2.10)
\end{document}