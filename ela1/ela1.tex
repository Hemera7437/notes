% Created 2016-11-11 Fr 13:59
\documentclass[a4paper]{scrartcl}
\usepackage[utf8]{inputenc}
\usepackage[T1]{fontenc}
\usepackage{fixltx2e}
\usepackage{graphicx}
\usepackage{longtable}
\usepackage{float}
\usepackage{wrapfig}
\usepackage{rotating}
\usepackage[normalem]{ulem}
\usepackage{amsmath}
\usepackage{textcomp}
\usepackage{marvosym}
\usepackage{wasysym}
\usepackage{amssymb}
\usepackage{hyperref}
\tolerance=1000
\usepackage{siunitx}
\usepackage{fontspec}
\sisetup{load-configurations = abbrevations}
\newcommand{\estimates}{\overset{\scriptscriptstyle\wedge}{=}}
\usepackage{mathtools}
\DeclarePairedDelimiter\abs{\lvert}{\rvert}%
\DeclarePairedDelimiter\norm{\lVert}{\rVert}%
\DeclareMathOperator{\Exists}{\exists}
\DeclareMathOperator{\Forall}{\forall}
\def\colvec#1{\left(\vcenter{\halign{\hfil$##$\hfil\cr \colvecA#1;;}}\right)}
\def\colvecA#1;{\if;#1;\else #1\cr \expandafter \colvecA \fi}
\usepackage{stmaryrd}
\author{Robin Heinemann}
\date{\today}
\title{Lineare Algebra (Vogel)}
\hypersetup{
  pdfkeywords={},
  pdfsubject={},
  pdfcreator={Emacs 25.1.1 (Org mode 8.2.10)}}
\begin{document}

\maketitle
\tableofcontents


\section{Einleitung}
\label{sec-1}
Übungsblätter/Lösungen:
jew. Donnerstag / folgender Donnerstag
Abgabe Donnerstag 9:30
50\% der Übungsblätter
\subsection{Plenarübung}
\label{sec-1-1}
Aufgeteilt
\subsection{Moodle}
\label{sec-1-2}
Passwort: vektorraumhomomorphismus
\subsection{Klausur}
\label{sec-1-3}
24.02.2017
\section{Grundlagen}
\label{sec-2}
\subsection{Naive Aussagenlogik}
\label{sec-2-1}
naive Logik: wir vewenden die sprachliche Vorstellung ($\neq$ mathematische Logik: eigne Vorlesung)
Eine Aussage ist ein festehender Satz, dem genau einer der Wahrheitswerte "wahr" oder "falsch" zugeordnet werden kann.
Aus einfachen Aussagen kann man durch logische Verknüpfungen kompliziertere Aussagen bilden.
Angabe der Wahrheitswertes der zusammengesetzten Aussage erfolgt duch Wahrheitstafeln (liefern den Wahrheitswert der zusammengesetzten Aussage, aus dem Wahrheitswert der einzelnen Aussagen).
Im folgenden seien $A$ und $B$ Aussagen.
\begin{itemize}
\item Negation (NICHT-Verknüpfung)
\begin{itemize}
\item Symbol: \$$\neg{}$
\item Wahrheitstafel:
\begin{center}
\begin{tabular}{ll}
$A$ & $\neg A$\\
\hline
w & f\\
f & w\\
\end{tabular}
\end{center}
\item Beispiel: $A$: 7 ist eine Primzahl (w)
$\neg A$: 7 ist keine Primzahl (f)
\end{itemize}

\item Konjunktion (UND-Verknüpfung)
\begin{itemize}
\item Symbol $\wedge$
\item Wahrheitstafel:
\begin{center}
\begin{tabular}{lll}
$A$ & $B$ & $A\wedge B$\\
\hline
w & w & w\\
w & f & f\\
f & w & f\\
f & f & f\\
\end{tabular}
\end{center}
\end{itemize}

\item Disjunktion (ODER-Verknüpfung)
\begin{itemize}
\item Symbol: $\vee$
\item Wahrheitstafel:
\begin{center}
\begin{tabular}{lll}
$A$ & $B$ & $A\vee B$\\
\hline
w & w & w\\
w & f & w\\
f & w & w\\
f & f & f\\
\end{tabular}
\end{center}
\item exklusives oder: $(A\vee B) \wedge (\neg(A\wedge B))$
\end{itemize}
\item Beispiel $A$: 7 ist eine Primzahl (w), $B$: 5 ist gerade (f)
\begin{itemize}
\item $A\wedge B$ 7 ist eine Primzahl und 5 ist gerade (f)
\item $A\vee B$ 7 ist eine Primzahl oder 5 ist gerade (w)
\end{itemize}

\item Implikation (WENN-DANN-Verknüpfung)
\begin{itemize}
\item Symbol: $\Rightarrow$
\item Wahrheitstafel:
\begin{center}
\begin{tabular}{lll}
$A$ & $B$ & $A\Rightarrow B$\\
\hline
w & w & w\\
w & f & f\\
f & w & w\\
f & f & w\\
\end{tabular}
\end{center}
\item Sprechweise: $A$ impliziert $B$, aus $A$ folgt $B$, $A$ ist eine hinreichende Bedingung für $B$ (ist $A\Rightarrow B$ wahr, dann folgt aus $A$ wahr, $B$ ist wahr), $B$ ist eine notwendige Bedingung für $A$ (ist $A\Rightarrow B$ wahr, dann kann $A$ nur dann wahr sein, wenn Aussage $B$ wahr ist)
\item Beispiel Es seinen $m,n\in\mathbb{N}$
\begin{itemize}
\item $A$: m ist gerade
\item $B$: $mn$ ist gerade
\item Dann gilt $\Forall m,n \in\mathbb{N}~A\Rightarrow B~\text{wahr}$ \\
         Fallunterscheidung:
\begin{itemize}
\item $m$ gerade, $n$ gerade, dann ist $A$ wahr, $B$ wahr, d.h. $A\Rightarrow B$ wahr
\item $m$ gerade, $n$ ungerade, dann ist $A$ wahr, $B$ wahr, d.h. $A\Rightarrow B$ wahr
\item $m$ ungerade, $n$ gerade, dann ist $A$ falsch, $B$ wahr, d.h. $A\Rightarrow B$ wahr
\item $m$ ungerade, $n$ ungerade, dann ist $A$ falsch, $B$ falsh, d.h. $A\Rightarrow B$ wahr
\end{itemize}
\end{itemize}
\end{itemize}
\item Äquivalenz (GENAU-DANN-WENN-Verknüpfung)
\begin{itemize}
\item Symbol $\Leftrightarrow$
\item Wahrheitstafel:
\begin{center}
\begin{tabular}{lll}
$A$ & $B$ & $A\Leftrightarrow$ B\\
\hline
w & w & w\\
w & f & f\\
f & w & f\\
f & f & w\\
\end{tabular}
\end{center}
\item Sprechweise: $A$ gilt genau dann, wenn $B$ gilt, $A$ ist hinreichend und notwendig für $B$ \\
       Die Aussagen $A\Leftrightarrow B$ und $(A\Rightarrow B)\wedge (B\Rightarrow A)$ sind gleichbedeutend:
\begin{center}
\begin{tabular}{llllll}
$A$ & $B$ & $A\Leftrightarrow B$ & $A\Rightarrow B$ & $B\Rightarrow A$ & $(A\Rightarrow B)\wedge (B\Rightarrow A)$\\
\hline
w & w & w & w & w & w\\
w & f & f & f & w & f\\
f & w & f & w & f & f\\
f & w & f & w & f & f\\
f & f & w & w & w & w\\
\end{tabular}
\end{center}
\item Beispiel: Es sei $n$ eine ganze Zahl \\
       $A:~n-2>1$ \\
       $B:~n>3$ \\
       $\Forall n\in\mathbb{N}~\text{gilt}~A\Leftrightarrow B$
       $C:~n>0$ \\
       $D:~n^2>0$ \\
       Für $n=-1$ ist die Äquivalenz $C\Leftrightarrow$ falsch ($C$ falsch, $D$ wahr) \\
       Für alle ganzen Zahlen $n$ gilt zumindest die Implikation $C\Rightarrow D$
\end{itemize}
\end{itemize}
\subsection{Beweis}
\label{sec-2-2}
Mathematische Sätze, Bemerkungen, Folgerungen, etc. sind meistens in Form wahrer Implikationen formuliert
\subsubsection{beweisen}
\label{sec-2-2-1}
Begründen warum diese Implikation wahr ist
\subsubsection{Beweismethoden for diese Implikation $A\Rightarrow B$}
\label{sec-2-2-2}
\begin{itemize}
\item direkter Beweis ($A\Rightarrow B$)
\item Beweis durch Kontraposition ($\neq B \Rightarrow \neg A$)
\item Widerspruchbeweis ($\neg (A\wedge \neg B)$)
\end{itemize}
Diese sind äquivalent zueinander
\begin{center}
\begin{tabular}{lllllll}
$A$ & $B$ & $\neg A$ & $\neg B$ & $A\Rightarrow B$ & $\neg B \Rightarrow \neg A$ & $\neg (A \wedge \neg B)$\\
\hline
w & w & f & f & w & w & w\\
w & f & f & w & f & f & f\\
f & w & w & f & w & w & w\\
f & f & w & w & w & w & w\\
\end{tabular}
\end{center}
\paragraph{Beispiel}
\label{sec-2-2-2-1}
$m,n$ natürliche Zahlen \\
     \[A:~m^2 < n^2\]
\[B:~m < n\]
Wir wollen zeigen, dass $A\Rightarrow B$ für alle natürlichen Zahlen $m,n$ wahr ist
\begin{itemize}
\item direkter Beweis: \\
       \[A:~m^2 < n^2 \Rightarrow 0 < n^2 - m^2 \Rightarrow 0 < (n-m)\underbrace{(n+m)}_{>0} \Rightarrow 0 < n-m \Rightarrow m<n\]
\item Beweis durch Kontraposition: \\
       \[\neg B:~m \geq n \Rightarrow m^2\geq n m \wedge m n \geq n^2 \Rightarrow m^2 \geq n^2 \Rightarrow \neg A\]
\item Beweis durch Widerspruch: \\
       \[A\wedge \neg B \Rightarrow m^2 < n^2 \wedge n\leq m \Rightarrow m^2 < n^2 \wedge m n \leq m^2 \wedge n^2 \leq m n \Rightarrow m n \leq m^2 < n^2 \leq m n\]
       Wiederspruch
\end{itemize}
\subsection{Existenz- und Allquantor}
\label{sec-2-3}
\subsubsection{Existenzquantor}
\label{sec-2-3-1}
\$A(x) Aussage, die von Variable x abhängt \\
    $\exists x:~A(x)$ ist gleichbedeutend mit "Es existiert ein x, für das $A(x)$ wahr ist" (hierbei ist "existiert ein x" im Sinne von "existiert mindestens ein x" zu verstehen) \\
    Beispiel:
\[\exists n\in\mathbb{N}:~n>5\quad\text{(w)}\]
$\exists !x:~A(x)$ ist gleichbedeutend mit "Es existiert genau ein x, für dass $A(x)$ wahr ist" 
\subsubsection{Allquantor}
\label{sec-2-3-2}
$\Forall x:~A(x)$ ist gleichbedeutend mit "Für alle x ist A(x) wahr"
Beispiel:
\[\Forall n\in\mathbb{N}: 4n~\text{ist gerade}\]
\subsubsection{Negation von Existenz- und Allquantor}
\label{sec-2-3-3}
\[\neg(\exists x:~A(x)) \Leftrightarrow \Forall x:~\neg A(x)\]
\[\neg(\Forall x:~A(x)) \Leftrightarrow \exists x:~\neg A(x)\]
\subsubsection{Spezielle Beweistechniken für Existenz und Allaussagen}
\label{sec-2-3-4}
\begin{itemize}
\item Angabe eines Beispiel, um zu zeigen, dass deine Existenzaussage wahr ist. \\
      Beispiel:
\[\exists n\in\mathbb{N}:~n>5 \text{ist wahr, denn für $n = 7$ ist die Aussage $n > 5$ wahr}\]
\item Angabe eines Gegenbeispiel, um zu zeigen, dass eine Allausage falsch ist. \\
      Beispiel:
\[\Forall n\in\mathbb{N}:~n\leq 5 \text{ist flasch, dann für $n=7$ ist die Aussage $n\leq 5$ falsch}\]
\end{itemize}
\subsection{Naive Mengenlehre}
\label{sec-2-4}
Mengenbegriff nach Cantor: \\
   Eine Menge ist eine Zusammenfassung von bestimmten, wohlunterschiedenen Objekten userer Anschauung oder useres Denkens (die Elemente genannt werden) zu einem Ganzen

\subsubsection{Schreibweise}
\label{sec-2-4-1}
\begin{itemize}
\item $x\in M$, falls $x$ ein Element von $M$ ist
\item $x\not\in M$, falls $x$ kein Element von $M$ ist
\item $M=N$, falls $M$ und $N$ die gleichen Elemente besitzen, $M\subseteq N \wedge N\subseteq M$
\end{itemize}

\subsubsection{Angabe von Mengen}
\label{sec-2-4-2}
\begin{itemize}
\item Reihenfolge ist unrelevant (\$\{1,2,3\}=\{1,3,2\})
\item Elemente sind wohlunterschieden $\{1,2,2\} = \{1,2\}$
\item Auflisten der Elemente $M=\{a,b,c,\ldots\}$
\item Beschreibung der Elemente durch Eigenschaften: $M=\{x\mid E(x)\}$ \\
     (Elemente x, für die E(x) wahr)
\begin{itemize}
\item Beispiel:
\[\{2,4,6,8\} = \{x\mid x\in\mathbb{N}, x~\text{gerade}, 1 < x < 10\}\]
\end{itemize}
\end{itemize}
\subsubsection{leere Menge}
\label{sec-2-4-3}
Die leere Menge $\emptyset$ enthält keine Elemente
\paragraph{Beispiel}
\label{sec-2-4-3-1}
\[\{x\mid x\in\mathbb{N}, x < -5\} = \emptyset\]

\subsubsection{Zahlenbereiche}
\label{sec-2-4-4}
Menge der natürlichen Zahlen:
\[\mathbb{N} := \{1,2,3,\ldots\}\]

Menge der natürlichen Zahlen mit Null:
\[\mathbb{N}_0 := \{0, 1,2,3,\ldots\}\]

Menge der Ganzen Zahlen:
\[\mathbb{Z} := \{0,1,-1,2,-2\}\]

Menge der rationalen Zahlen:
\[\mathbb{Q} := \{\frac{m}{n} \mid m\in\mathbb{Z}, n\in\mathbb{N}\}\]

Menge der reellen Zahlen: $\mathbb{R}$

\subsubsection{Teilmenge}
\label{sec-2-4-5}
$A,B$ seien Mengen. \\
    $A$ heißt Teilmenge von $B~(A\subseteq B) \xLeftrightarrow{\text{Def.}} \Forall x\in A: x\in B$
$A$ heißt echte Teilmenge von $B~(A\subset B) \xLeftrightarrow{\text{Def.}} A\subseteq B \wedge A\neq B$
\paragraph{Anmerkung}
\label{sec-2-4-5-1}
Offenbar gilt für Mengen $A,B$:
\[A=B \Leftrightarrow A\subseteq B \wedge B\subseteq A\]
$\emptyset$ ist Teilmenge jeder Menge

\paragraph{Beipspiel}
\label{sec-2-4-5-2}
\[\mathbb{N}\subset\mathbb{N}_0\subset\mathbb{Z}\subset\mathbb{Q}\]

\subsubsection{Durschnitt}
\label{sec-2-4-6}
\[A \cap B := \{x\mid x\in A \wedge x\in B\}\]
\paragraph{Beispiel}
\label{sec-2-4-6-1}
\[A=\{2,3,5,7\}, B=\{3,4,6,7\}, A\cap B = \{3,7\}\]

\subsubsection{Vereinigung}
\label{sec-2-4-7}
\[A\cup B := \{x\mid x\in A \vee x\in B\}\]
\paragraph{Beispiel}
\label{sec-2-4-7-1}
\[A=\{2,3,5,7\}, B=\{3,4,6,7\}, A\cup B = \{2,3,4,5,6,7\}\]

\subsubsection{Differenz}
\label{sec-2-4-8}
\[A\setminus B := \{x\mid x\in A \wedge x\not\in B\}\]
Im Fall $B\subseteq A$ nennt man $A\setminus B$ auch das Komplement von $B$ in $A$ und schreibt $\mathcal{c}_A(B) = A\setminus B$
\paragraph{Beispiel}
\label{sec-2-4-8-1}
\[A=\{2,3,5,7\}, B=\{3,4,6,7\}, A\setminus B = \{2,5\}\]
\subsubsection{Bemerkung zu Vereinigung und Durschnitt}
\label{sec-2-4-9}
$A,B$ seien zwei Mengen. Dann gilt \[A\cap (B\cup C) = (A\cap B) \cup (A\cap C)\]
\paragraph{Beweis}
\label{sec-2-4-9-1}
\[A\cap(B\cup C) \subseteq (A\cap B) \cup (A\cap C)\]
\[A\cap(B\cup C) \supseteq (A\cap B) \cup (A\cap C)\]
"$\subseteq$" Sei $x\in A \cap (B\cup C)$. Dann ist $x\in A \wedge x\in B\cup C$
\begin{itemize}
\item 1. Fall: $x\in A \wedge x\in B$
       \[\Rightarrow x\in A\cap B \Rightarrow x \in (A\cap B) \cup (A\cap C)\]
\item 2. Fall $x\in A \wedge x\in C$
       \[\Rightarrow x\in A\cap C \Rightarrow x\in (A\cap B)\cup(A\cap C)\]
\end{itemize}
Damit ist "$\subseteq$" gezeigt.
"$\supseteq$" Sei \$x$\in$ (A$\cap$ B) $\cup$ (A$\cap$ C)
\[\Rightarrow x\in A\cap B \vee x\in A\cap C \\ \Rightarrow (x\in A \wedge x\in B) \vee (x\in A \wedge x\in C) \\ \Rightarrow x\in A \wedge (x\in B\vee x\in C) \\ \Rightarrow x\in A \wedge x\in B\cup C \\ \Rightarrow x\in A\cap (B\cup C)\]
Damit ist "$\supseteq$" gezeigt.
\subsubsection{Bemerkung zu Äquivalenz von Mengen}
\label{sec-2-4-10}
Seien $A,B$ Mengen, dann sind äquivalent:
\begin{enumerate}
\item $A\cup B = B$
\item $A\subseteq B$
\end{enumerate}
\paragraph{Beweis}
\label{sec-2-4-10-1}
Wir zeigen 1) $\Rightarrow$ 2) und 2) $\Rightarrow$ 1.
\[1) \Rightarrow 2):~\text{Es gelte}~A\cup B = B,~\text{zu zeigen ist}~A\subseteq B \\ \text{Sei}~x\in A \Rightarrow x\in A \wedge x \in B \Rightarrow x\in A\cup B = B\]
\[2) \Rightarrow 1):~\text{Es gelte}~A\subseteq B\text{, zu zeigen ist}~A\cup B = B \]
"$\subseteq$": Sei $x\in A\cup B \Rightarrow x\in A \vee x\in B \xRightarrow{A\subseteq B} x\in B$
"$\supseteq$": $B\subseteq A\cup B$ klar
\subsubsection{Kartesisches Produkt}
\label{sec-2-4-11}
Seien $A,B$ Mengen
\[A\times B := \{(a,b)\mid a\in A, b\in B\}\]
heipt das kartesische Produkt von $A$ und $B$. Hierbei ist $(a,b) = (a',b') \xLeftrightarrow{\text{Def}} a = a' \wedge b = b'$ a = a' $\wedge$ b = b'\$

\paragraph{Beispiel}
\label{sec-2-4-11-1}
\begin{itemize}
\item \[\{1,2\}\times \{1,3,4\} = \{(1,1),(1,3),(1,4),(2,1),(2,3),(2,4)\}\]
\item \[\mathbb{R}\times\mathbb{R}=\{(x,y)|mid x,y \in \mathbb{R}\} = \mathbb{R}^2\]
\end{itemize}
\subsubsection{Potenzmenge}
\label{sec-2-4-12}
$A$ sei eine Menge
\[\mathcal{P} (A) := \{M\mid M\subseteq A\}\]
heißt die Potenzmenge von $A$
\paragraph{Beispiel}
\label{sec-2-4-12-1}
\[\mathcal{P} (\{1,2,3\}) =  \{\emptyset, \{1\}, \{2\},\{3\},\{1,2\},\{1,3\},\{2,3\}\{1,2,3,4\}\}\]
\subsubsection{Kardinalität}
\label{sec-2-4-13}
$M$ sei eine Menge. Wir setzen
\[\abs{M} := \begin{cases} n & \text{falls $M$ eine endliche Menge ist und $n$ Elemente enthält} \\ \infty & \text{falls $M$ nicht endlich ist} \end{cases}\]
$\abs{M}$ heißt Kardinalität von A
\paragraph{Beispiel}
\label{sec-2-4-13-1}
\begin{itemize}
\item $\abs{\{7,11,16\}} = 3$
\item $\abs{\mathbb{N}} = \infty$
\end{itemize}
\subsubsection{Bemerkung zu natürlichen Zahlen}
\label{sec-2-4-14}
Für die natürlichen Zahlen gilt das Induktionsaxiom
Ist $M\subseteq N$ eine Teilmenge, für die gilt:
\[1\in M \wedge \Forall n\in M : n\in M \Rightarrow n+1 \in M\]
dann ist $M = \mathbb{N}$
\subsubsection{Prinzip der vollständigen Induktion}
\label{sec-2-4-15}
Für jedes $n\in \mathbb{N}$ sei eine Aussage $A(n)$ gegeben. Die Aussagen $A(N)$ gelten für alle $n\in\mathbb{N}$, wenn man folgendes zeigen kann: \\
\begin{itemize}
\item (IA) $A(1)$ ist wahr
\item (IS) Für jedes $n\in\mathbb{N}$ gilt: $A(n) \Rightarrow A(n+1)$
\end{itemize}
Der Schritt (IA) heißt Induktionsanfang, die Implikation $A(n) \Rightarrow A(n+1)$ heißt Induktionsschritt
\paragraph{Beweis}
\label{sec-2-4-15-1}
Setze $M := \{n\in \mathbb{N}\mid A(n)~\text{ist wahr}\}$
Wegen (IA) ist $1\in M$, wegen (IS) gilt: $n\in M \Rightarrow n+1 \in M$ \\
     Nach Induktionsaxiom folgt $M = \mathbb{N}$, das heißt $A(n)$ ist wahr für alle $n\in \mathbb{N}$
\paragraph{Beispiel}
\label{sec-2-4-15-2}
Für $n\in\mathbb{N}$ sei $A(n)$ die Aussage: $1+\ldots + n = \frac{n(n+1)}{2}$
Wir zeigen: $A(n)$ ist wahr für alle $n\in \mathbb{N}$, und zwar durch vollständige Induktion
\begin{itemize}
\item (IA) $A(1)$ ist wahr, denn $1 = \frac{1(1+1)}{2}$
\item (IS) zu zeigen: $A(n) \Rightarrow A(n+1)$ \\
       Es gelte $A(n)$, das heißt $1+\ldots+n = \frac{n(n+1)}{2}$ ist wahr \[\Rightarrow 1 + \ldots + n + (n + 1) = \frac{n(n+1)}{2} + (n+1) =  \frac{n(n+1) + 2(n+1)}{2} = \frac{(n+1)(n+2)}{2} \square\]
\end{itemize}
\subsection{Relationen}
\label{sec-2-5}
\subsubsection{Definiton}
\label{sec-2-5-1}
Eine Relation auf $M$ ist eine Teilmenge $R\subseteq M\times M$
Wir schreiben $a\sim b \xLeftrightarrow{\text{Def}} (a,b) \in R$ ("a steht in Relation zu b")

\begin{itemize}
\item anschaulich: eine Relation auf $M$ stellt eine "Beziehung" zwischen den Elementen von $M$ her.
\item Für $a,b \in M$ gilt entweder $a\sim b$ oder $a\not\sim b$, denn: entweder ist $(a,b) \in R$ oder $(a,b)\not\in R$
\end{itemize}
\paragraph{Anmerkung}
\label{sec-2-5-1-1}
Aufgrund der obigen Notation spricht man in der Regel von Relation "\$$\sim$" auf $M$ als von der Relation $R \subseteq M\times M$
\paragraph{Beispiel}
\label{sec-2-5-1-2}
\$M = \{1,2,3\}. Durch $R = \{(1,1), (1,2), (3,3) \subseteq M\times M\}$ ist eine Relation auf $M$ gegeben. Es gilt dann: $1\sim 1, 1\sim 2, 3\sim 3$ (aber zum Beispiel: $1\not\sim 3, 2\not\sim 1, 2\not\sim 2$)

\subsubsection{Eigenschaften von Relationen}
\label{sec-2-5-2}
$M$ Menge, $\sim$ Relation auf $M$ \\
    $\sim$ heißt:
\begin{itemize}
\item reflexiv $\xLeftrightarrow{\text{Def}}$ für alle $a\in M$ gilt $a\sim a$
\item symmetrisch $\xLeftrightarrow{\text{Def}}$ für alle $a,b\in M$ gilt: $a\sim b \Rightarrow b\sim a$
\item antisymmetrisch $\xLeftrightarrow{\text{Def}}$ für alle $a,b \in M$ gilt: $a\sim b \wedge b\sim a \Rightarrow a = b$
\item transitiv $\xLeftrightarrow{\text{Def}}$ für alle $a,b,c\in M$ gilt: $a\sim b \wedge b\sim v \Rightarrow a\sim c$
\item total $\xLeftrightarrow{\text{Def}}$ für alle $a,b\in M$ gilt: $a\sim b \vee b\sim a$
\end{itemize}
\paragraph{Beispiel}
\label{sec-2-5-2-1}
Sei $M$ die Menge der Studierenden in der LA1-Vorlesung
\begin{enumerate}
\item Für $a,b \in M$ sei $a\sim b \xLeftrightarrow{\text{Def}}$ $a$ hat den selben Vornamen wie $b$ \\
        $\sim$ reflexiv, symmetrisch, nicht antisymmetrisch, transitiv, nicht total
\item Für $a,b \in M$ sei $a\sim b \xLeftrightarrow{\text{Def}}$ Martrikelnummer von $a$ ist kleiner gleich als die Martrikelnummer von $b$ \\
        $\sim$ ist reflexiv, nicht symmetrisch, antisymmetrisch, transitiv, total
\item Für $a,b \in M$ sei $a\sim b \xLeftrightarrow{\text{Def}}$ $a$ sitzt auf dem Platz recht von $b$ \\
        $\sim$ ist nicht reflexiv, nicht symmetrisch, nicht antisymmetrisch, nicht transitiv, nicht total
\end{enumerate}
\subsubsection{Halbordnung / Totalordung}
\label{sec-2-5-3}
$\sim$ heißt
\begin{itemize}
\item Halbordnung auf $M\xLeftrightarrow{\text{Def}}~\sim$ ist reflexiv, antisymmetrisch und transitiv
\item Totalordung auf $M\xLeftrightarrow{\text{Def}}~\sim$ ist eine Halbordnung und $\sim$ ist total
\end{itemize}
In diesen Fällen sagt man auch: Das Tupel $(M,\sim)$ ist eine halbgeordnete, beziehungsweise totalgeordnete Menge.
\paragraph{Beispiel}
\label{sec-2-5-3-1}
\begin{enumerate}
\item $\leq$ auf $\mathbb{N}$ ist eine Totalordung
\item Sei $M = \mathcal{P}(\{1,2,3\})$. $\subseteq$ ist auf $M$ eine Halbordung, aber keine Totalordung (es ist zum Beispiel weder $\{1\} \subseteq \{3\}$ noch $\{3\}\subseteq \{\}$)
\end{enumerate}
\paragraph{Anmerkung}
\label{sec-2-5-3-2}
Wegen der Analogie zur $\leq$ auf $\mathbb{N}$ bezeichnen wir Halbordnungen in der Regel mit $\leq$
\subsubsection{Größtes / kleinstes Element}
\label{sec-2-5-4}
$(M, \leq)$ halbgeordnete Menge, $a\in M$ \\
    $a$ heißt ein
\begin{itemize}
\item größtes Element von $M\xLeftrightarrow{\text{Def}}$ Für alle $x\in M$ gilt $x\leq a$
\item kleinstes Element von $M\xLeftrightarrow{\text{Def}}$ Für alle $x\in M$ gilt $a\leq x$
\end{itemize}
\paragraph{Bemerkung}
\label{sec-2-5-4-1}
$(M,\leq)$ halbgeordnete Menge \\
     Dann gilt: Existiert in $M$ ein größtes (beziehungsweise kleinstes) Element, so ist dieses eindeutig bestimmt
\subparagraph{Beweis}
\label{sec-2-5-4-1-1}
Es seien $a,b\in M$ größte Elemente von $M$ \\
      $\Rightarrow x\leq a$ für alle $x\in M$, also auch $b\leq a$ \\
      Außerdem: $x \leq b$ für alle $x\in M$, also auch $a\leq b$ \\
      $\xRightarrow{\text{Antisymmetrie}} a = b$ \\
      Analog für kleinstes Element
\subparagraph{Anmerkung}
\label{sec-2-5-4-1-2}
Dies sagt nichts darüber aus, ob ein größtes (beziehungsweise kleinstes) Element in $M$ überhaupt existiert.
\paragraph{Beispiel}
\label{sec-2-5-4-2}
\begin{enumerate}
\item In $(\mathbb{N},\leq)$ ist 1 das kleinste Element, ein größtes Element gibt es nicht
\item $(\{\{1\},\{2\},\{3\},\{1,2\},\{1,3\},\{2,3\}\}, \subseteq)$ ist eine halbgeordnete Menge ohne kleinstes beziehungsweise größtes Element
\end{enumerate}
\subsubsection{maximales / minimales Element}
\label{sec-2-5-5}
$(M,\leq)$ halbgeordnete Menge, $a\in M$ \\
    $a$ heißt ein
\begin{itemize}
\item maximales Element von $M \xLeftrightarrow{\text{Def}}$ für alle $x\in M$ gilt: $a\leq x \Rightarrow a = x$
\item minmales Element von $M \xLeftrightarrow{\text{Def}}$ für alle $x\in M$ gilt: $x\leq a \Rightarrow a = x$
\end{itemize}
\paragraph{Beispiel}
\label{sec-2-5-5-1}
In $(\{\{1\},\{2\},\{3\},\{1,2\},\{1,3\},\{2,3\}\}, \subseteq)$ sind $\{1,2\},\{1,3\},\{2,3\}$ maximale Elemente und $\{1\},\{2\},\{3\}$ sind minimale Elemente.
\paragraph{Bemerkung}
\label{sec-2-5-5-2}
$(M,\leq)$ halbgeordnete Menge, $a\in M$ \\
     Dann gilt: Ist $a$ ein größtes (beziehungsweise kleinstes) Element von $M$, dann ist $a$ ein maximales (beziehungsweise minimales) Element von $M$.
\subparagraph{Beweis}
\label{sec-2-5-5-2-1}
Sei $a$ ein größtes Element von $M$. \\
      zu zeigen ist: Für alle $x\in M$ gilt $a\leq x \Rightarrow a = x$
Sei $x\in M$ mit $a\leq x$. Da $a$ größtes Element von $M$ ist, gilt auch $x\leq a$ \\
      $\xLeftrightarrow{\text{Antisymmetrie}} a = x$ \\
      Analog für kleinstes Element.
\subsubsection{Äquivalenzrelation}
\label{sec-2-5-6}
$M$ Menge, $\sim$ auf $M$ \\
    $\sim$ heißt Äquivalenzrelation $\xLeftrightarrow{\text{Def}}~\sim$ ist reflexiv, symmetrisch und transitiv.
In dem Fll sagen wir für $a\sim b$ auch $a$ ist äquivalent zu $b$. Für $a\in M$ heißt $[a]:=\{b\in M \mid b\sim a\}$ heißt die Äquivalentklasse von $a$.
Elemente aus $[a]$ nennt man Vertreter oder Repräsentanten von $a$
\paragraph{Beispiel}
\label{sec-2-5-6-1}
$M$ Menge aller Bürgerinnen und Bürger Deutschlands. \\
     Wir definieren für $a,b\in M$ $a\sim b \xLeftrightarrow{\text{Def}} a$ und $b$ sind im selben Jahr geboren. \\
     $\sim$ ist ein Äquivalenzrelation. \\
     Jerôme Boateng wurde 1988 geboren. \\
      $[\text{Jerôme Boateng}] = \{b\in M\mid b~\text{ist im selben Jahr geboren wie Jerôme Boateng}\} = \{b\in M\mid b~\text{wurde 1988 geboren}\}$
Weitere Vertreter von $[\text{Jerôme Boateng}]$ sind zum Beispiel Mesut Özil, Mats Hummels.
Es ist $[\text{Jerôme Boateng}] = [\text{Mesut Özil}] = [\text{Mats Hummels}]$.
Man sieht in diesem Beispiel: Die Menge $M$ zerfällt komplett in verschiedene Äquivalentzklassen:
\begin{itemize}
\item Jeder Bürger / jede Bürgerinn Detuschalnds ist in genau einer Äquivalenzklasse enthalten
\item Jede zwei Äquivalentklasse sind entweder gleich oder disjunkt (haben leeren Durchschnitt)
\end{itemize}
\paragraph{Bemerkung}
\label{sec-2-5-6-2}
$M$ Menge, $\sim$ Äquivalenzrelation auf $M$ \\
     Dann gilt:
\begin{enumerate}
\item Jedes Element von $M$ liegt in genau einer Äquivalenzklasse
\item Je zwei Äquivalenzklassen sind entweder gleich oder disjunkt
\end{enumerate}
Man sagt auch: Die Äquivalenzklassen bezüglich "$\sim$" bilden eine \textbf{Partition} von $M$.
\subparagraph{Beweis}
\label{sec-2-5-6-2-1}
\begin{enumerate}
\item Sei $a\in M$ \\
         zu zeigen: Es gibt genau eine Äquivalenzklassen, in der $a$ liegt
\begin{enumerate}
\item Es gibt eine Äquivalenzklasse, in der $a$ liegt, denn \$a$\in$ [a], denn $a\sim a$
\item Ist \$a$\in$[b] und a$\in$[c], dann ist [b]=[c] (d.h. $a$ liegt in höchstens einer Äquivalenzklasse) \\
            denn: Seien $b,c\in M$ mit $a\in[b]$ und $a\in[c]$
            $\Rightarrow a\sim b$ und $a\sim c \xRightarrow{\text{Symmetrie}} b\sim a$ und $a\sim c \xRightarrow{\text{Transitivität}} b\sim c$
            Behautptung $[b] =[c]$
            denn: "$\subseteq$" Sei $x\in [b] \Rightarrow x\sim b \xRightarrow{Transitivität}^{b\sim c} x\sim c \Rightarrow x\in [c]$
            denn: "$\supseteq$" Sei $x\in [c] \Rightarrow x\sim c \xRightarrow{Transitivität}^{c\sim b} x\sim b \Rightarrow x\in [b]$
\end{enumerate}
\item Sind $b,c\in M$ mit $[b] \cap [c] \neq \emptyset$, dann existiert ein \$a$\in$ [b]$\cap$ [c], und es folgt wie in 2.: \\
         $[b] = [c]$
         Für $b,c\in M$ gilt also entweder $[b]\cap[c] =\emptyset$ oder $[b] = [c]\hfill\square$
\end{enumerate}
\paragraph{Faktormenge}
\label{sec-2-5-6-3}
$M$ Menge, $\sim$ Äquivalenzrelation auf $M$
$M/\sim := \{[a]|a\in M\}$ (Menge der Äquivalenzklassen) heißt die Faktormenge (Quotientenmenge) von $M$ nach $\sim$
\subparagraph{Beispiel}
\label{sec-2-5-6-3-1}
\[M= \{1,2,3,-1,-2,-3\}\]
Für $a,b,c \in M$ setzen wir $a\sim b \xLeftrightarrow{\text{Def.}} \abs{x} = \abs{b}$
Das ist eine Äquivalenzrelation auf $M$
Es ist $[1] = \{1,-1\},[2]=\{2,-2\},[3]=\{3,-3\}$
Somit: $M/sim := \{[1],[2],[3]\} = \{\{1,-1\},\{2,-2\},\{3,-3\}\}$
\subparagraph{Anmerkung}
\label{sec-2-5-6-3-2}
Der Übergang zur Äquivalenzklassen soll (für eine jeweils gegebene Relation) irrelevante Informationen abstreifen.
\subsection{Abbildungen}
\label{sec-2-6}
\textbf{naive Definition}: \\
    Eine Abbildung $f$ von $M$ nach $N$ ist eine Vorschrift, die jedem $n\in M$ genau ein Element aus $N$ zuordnet, dieses wird mit $f(n)$ bezeichnet.
\textbf{Notation}: \\
    \[f:M\to N,m\mapsto f(m)\]

Zwei Abbildungen $f,g:M\to N$ sind gleich, wenn gilt $\Forall n\in M:f(n) = g(n)$
$M$ heißt die Definitionsmenge von $f$, $N$ heißt die Zielmenge von $f$
\subsubsection{Definition}
\label{sec-2-6-1}
Eine Abbildung $f$ von $M$ nach $N$ ist ein Tupel $(M,N,G_f)$, wobei $G_f$ eine Teilmenge von $M\times N$ mit der Eigenschaft ist, dass für jedes Element $m\in M$ genau ein Element $n\in N$ mit $(m,n) \in G_f$ existiert.
(für dieses Element $n$ schreiben wir auch $f(m)$). $G_f$ heißt der Graph von $f$.
\subsubsection{Beispiel}
\label{sec-2-6-2}
\begin{enumerate}
\item $f:\mathbb{R}\to\mathbb{R}, x\mapsto x^2$
\item $f:\mathbb{R}\to\mathbb{R}^2,x\mapsto (x,x+1)$
\item $M$ Menge, $id_M: M\to M,m\mapsto m$ heißt Identität (identische Abbildung) auf $M$
\item $I$,$M$ Mengen: Eine über $I$ indizierte Familie von Elementen von $M$ ist eine Abbildung: \\
       $m:I\to M,i\mapsto m(i) =: m_i$. Wir schreiben für die Familie auch kurz $(m_i)_{i\in I}$. $I$ heißt Indexmenge der Familie.
\item Spezialfall von 4.: $I = \mathbb{N},M = \mathbb{R}:~((m_i)_{i\in\mathbb{N}})$ nennt man auch Folge reeler Zahlen.
\end{enumerate}
\subsubsection{Anmerkung über den Begriff der Familie}
\label{sec-2-6-3}
Über den Begriff der Familie lassen sich diverse Konstruktionen aus der naiven Mengenlehre verallgemeinern.
Ist $(M_i)_{i\in I}$ eine Familie von Mengen, dann ist:
\[\cup_{i\in I} M_i:=\{x\mid\exists i\in I: x\in M_i\}\]
\[\cap_{i\in I}M_i := \{x\mid\Forall i\in I: x\in M_i\}\]
\[\prod_{i\in I}M_i := \{(x_i)_{i\in I}\mid \Forall i\in I: x_i \in M\}\]
\subsubsection{Bild}
\label{sec-2-6-4}
$m,N$ Mengen, $f:M\to n$ Abbildung. \\
    Sind $m\in M,n\in N$ mit $n = f(m)$ dann nennen wir $n$ ein \textbf{Bild} von $m$ unter $f$ und wir nennen $m$ ein \textbf{Urbild} von $n$ unter $f$.
\paragraph{Anmerkung}
\label{sec-2-6-4-1}
In obiger Situation ist das Bild von $m$ unter $f$ eindeutig bestimmt (nach der Definition einer Abbildung)
Urbilder sind im allgemeinen nicht eindeutig bestimmt, und im Allgemeinen besitzt nicht jedes Element aus $N$ ein Urbild.
\paragraph{Beispiel}
\label{sec-2-6-4-2}
$f:\mathbb{R}\to\mathbb{R},x\mapsto x^2$, dann ist $4=f(2) = f(-2)$, das heißt $2$ und $-2$ sind Urbilder von $4$, das Element $-5$ hat kein Urbild unter $f$, denn es existiert kein $x\in\mathbb{R}$ mit $x^2 = -5$
\paragraph{Definition}
\label{sec-2-6-4-3}
$M, N$ Mengen, $f:M\to N$ Abbildung, $A\subseteq M, B\subseteq N$ \\
     $f(A) := \{f(a)\mid a\in A\} \subseteq N$ heißt das Bild von $A$ unter $f$. \\
     $f^-1(B) := \{m\in M\mid f(m) \in B\} \subseteq M$ heißt das Urbild von $B$ unter $f$
\subparagraph{Beispiel}
\label{sec-2-6-4-3-1}
\[f:\mathbb{R}\to\mathbb{R},x\mapsto x^2\]
\[f(\{1,2,3\}) = \{1,4,9\}\]
\[f^-1(\{4,-5\}) = \{2,-2\}\]
\[f^-1(\{4\}) = \{2,-2\}\]
\[f^-1(\{-5\}) = \emptyset\]
\[f(\mathbb{R}) = {x^2\mid x\in \mathbb{R}} = \{x\in\mathbb{R}\mid x\geq 0\} =:\mathbb{R}_{\geq 0}\]
\subsubsection{Restriktion}
\label{sec-2-6-5}
$M,N$ Mengen, $f:M\to N$ Abbildung, $A\subseteq M$
\[f\mid_A:A\to N, m\mapsto f(m)\]
heißt die Restriktion von $f$ auf $A$.
Ist $B\subseteq N$ mit $f(A) \subseteq B$, dann setzen wir
\[f\mid_A^B: A\to B,m\mapsto f(m)\]
Ist $f(M) \subseteq B$ dann setzen wir:
\[f\mid^B := f\mid_M^B,M\to B, m\mapsto f(m)\]
\subsubsection{Komposition}
\label{sec-2-6-6}
$L,M,N$ Mengen, $f:L\to M,g:M\to N$ Abbildung \\
    \[g\circ f: L\to N, x\mapsto(g\circ f)(x):=g(f(x))\]
heißt die Komposition (Hintereinanderausführung) von $f$ und $g$
\paragraph{Beispiel}
\label{sec-2-6-6-1}
\[f:\mathbb{R}\to\mathbb{R},x\mapsto x^2, g:\mathbb{R}\to\mathbb{R}:x\mapsto x + 1\]
\[\Rightarrow g\circ f:\mathbb{R}\to\mathbb{R},x\mapsto g(f(x)) = g(x^2) = x^2 + 1\]
\paragraph{Assoziativität}
\label{sec-2-6-6-2}
$L,M,N,P$ Mengen, $f:L\to M, g:M\to N,h:n\to p$ \\
     Dann gilt
\[h\circ (g\circ f) = (h\circ g)\circ f\]
das heißt die Verknüpfung von Abbildungen ist assoziativ.
\subparagraph{Beweis}
\label{sec-2-6-6-2-1}
Für $x\in L ist$ \\
      \[(h\circ (g\circ f)) = h((g\circ f)(x)) = h(g(f(x))) = (h\circ g)(f(x)) = ((h\circ g)\circ f)(x)\hfill\square\]
\subsubsection{Eigenschaften von Abbildungen}
\label{sec-2-6-7}
$M,N$ Mengen, $f:M\to N$ Abbildung
\paragraph{Injektivität}
\label{sec-2-6-7-1}
$f$ heißt injektiv: \[\xLeftrightarrow{\text{Def}} \Forall m_1,m_2\in M: f(m_1) = f(m_2) \Rightarrow m_1 = m_2 \Leftrightarrow \Forall m_1,m_2\in M: m_1\neq m_2 \Rightarrow f(m_1)\neq f(m_2)\]
\paragraph{Surjektivität}
\label{sec-2-6-7-2}
$f$ heißt sujektiv:
\[\xLeftrightarrow{\text{Def}} \Forall n\in M :\exists m\in M: f(m) = n \Leftrightarrow f(M) = N\]
\paragraph{Bijektivität}
\label{sec-2-6-7-3}
$f$ heißt bijektiv: $\xLeftrightarrow{\text{Def}}$ $f$ ist injektiv und surjektiv
\paragraph{Beispiel}
\label{sec-2-6-7-4}
\begin{enumerate}
\item $f:\mathbb{R}\to\mathbb{R},x\mapsto x^2$ ist:
\begin{itemize}
\item nicht injektiv, denn $f(2) = f(-2)$, aber $2\neq -2$
\item nicht surjektiv, denn es existier kein $m\in\mathbb{R}$ mit $f(m) = -1$
\item nicht bijektiv
\end{itemize}
\item $f:\mathbb{R}_{\geq 0} \to \mathbb{R}, x\mapsto x^2$ ist:
\begin{itemize}
\item injektiv, denn für $m_1,m_2 \in\mathbb{R}_{\geq 0}$ gilt: $f(m_1) = f(m_2) \Rightarrow m_1^2 = m_2^2 \xRightarrow{m_1,m_2 > 0} m_1 = m_2$
\item nicht surjektiv, denn es existier kein $m\in\mathbb{R}_{\geq 0}$ mit $f(m) = -1$
\item nicht bijektiv
\end{itemize}
\item $f:\mathbb{R}_{\geq 0} \to \mathbb{R}_{\geq 0}, x\mapsto x^2$ ist:
\begin{itemize}
\item injektiv, denn für $m_1,m_2 \in\mathbb{R}_{\geq 0}$ gilt: $f(m_1) = f(m_2) \Rightarrow m_1^2 = m_2^2 \xRightarrow{m_1,m_2 > 0} m_1 = m_2$
\item surjektiv, denn für $m\in\mathbb{R}_{\geq 0}$ ist $f(\sqrt{m}) = (\sqrt{m})^2 = m$
\item bijektiv
\end{itemize}
\end{enumerate}
\paragraph{Bemerkung 4.12}
\label{sec-2-6-7-5}
$M,N$ Mengen, $f:M\to N, g:n\to M$ mit $g\circ f = id_M$
Dann ist $f$ injektiv und $g$ surjektiv.
\subparagraph{Beweis}
\label{sec-2-6-7-5-1}
\begin{enumerate}
\item $f$ ist injektiv, denn: \\
         Seien $m_1, m_2 \in M$ mit $f(m_1) = f(m_2) \Rightarrow g(f(m_1)) = g(f(m_2)) \Rightarrow (g\circ f)(m_1) = (g\circ f)(m_2) \Rightarrow id_m(m_1) = id_M(m_2)\Rightarrow m_1 = m_2$
\item $g$ ist surjektiv, denn: \\
         Sei $m\in M$ Dann ist $m=id_M(m) = (g\circ f)(m) = g(f(m))$
\end{enumerate}
\paragraph{Bemerkung}
\label{sec-2-6-7-6}
Sei $f:M\to N$, $N,M$ Mengen
Dann sind äquivalent:
\begin{enumerate}
\item $f$ ist bijektiv
\item Zu jedem $n\in N$ gibt es genau ein $m\in M$ mit $f(m) = n$
\item Es gibt genau eine Abbildung $g:N\to M$ mit $g\circ f = id_M$ und $f\circ g = id_N$
\end{enumerate}
In diesem Fall bezeichnen wir die Abbildung $g:N\to M$ aus 3. mit $f^{-1}$ und nennen $f^{-1}$ die Umkehrabbildung von $f$. Sie ist gegeben durch
\[f^{-1} : N\to M, n\mapsto~\text{Das eindeutig bestimmte Element $m\in M$ mit $f(m) = n$}\]
\subparagraph{Beweis}
\label{sec-2-6-7-6-1}
Statt 1. $\Leftrightarrow$ 2. und 2. $\Leftrightarrow$ 3. zeigen 1. $\Rightarrow$ 2. $\Rightarrow$ 3. $\Rightarrow$ 1.
\begin{itemize}
\item 1. $\Rightarrow$ 2. Sei $f$ bijektiv \\
        zz: Ist $n\in N$, dann existiert genau ein $m\in M$ mit $f(m) = n$ \\
\begin{itemize}
\item Existenz folg aus Surjektivität von $f$
\item Eindeutigkeit: Seien \$m$_{\text{1}}$,m$_{\text{2}}$ $\in$ M  mit \(f(m_1) = n, f(m_2) = n \Rightarrow f(m_1) = f(m_2) \xRightarrow{f injektiv} m_1 = m_2\)
\end{itemize}
\item 2. $\Rightarrow$ 3. Zu jedem $n\in M$ existiere genau ein $m\in M$ mit $f(m) = n$ \\
        zz: Ex existert genau eine Abbildung $g:N\to M$ mit $f\circ f = id_M$ und $f\circ g = id_N$
\begin{itemize}
\item Existenz: Wir definieren \(g:N\to M, n\mapsto~\text{das nach 2. eindeutig \\ 
		  bestimmte Element $m\in M$ mit $f(m) = n$}\) \\
Dann gilt für $m\in M$: \[(g\circ f)(m) = f(f(m)) = m,~text{das heißt}~ g\circ f = id_M\]
und für $n\in N$ ist $(f\circ g)(n) = f(g(n)) = n$ also $f\circ g = id_N$
\item Eindeutigkeit: Es seinen $g_1,g_2:N\to M$ mit $g_i \circ f = id_M, f\circ g_i = id_N$ für $i = 1,2$ \\
          \[\Rightarrow g_1 = g_1 \circ id_N = g_1 \circ (f\circ g_2) = (g_1 \circ f) \circ g_2 = id_M \circ g_2 = g_2\]
\end{itemize}
\item 3. $\Rightarrow$ 1. Wegen 3. existier $g:N\to M$ mit $g\circ f = id_M,f\circ g = id_N$ \\
        \[\xRightarrow{[[Bemerkung 4.12]]} f~\text{injektiv}~,f~\text{surjektiv}~\Rightarrow f~\text{bijektiv}\Rightarrow~\text{1.}\]
\end{itemize}
\subparagraph{Anmerkung}
\label{sec-2-6-7-6-2}
\begin{itemize}
\item Bitte stets aufpassen, ob mit $f^{-1}$ die Unmkerhabbildung (falls existent) oder das Bilden der Urbildmenge gemeint ist.
\item Im Beweis von 3. $\Rightarrow$ 1. haben wir die Eindeutigkeit von $g$ garnicht verwendet, das heißt wir haben sogar gezeigt: \\
        $f$ bijektiv $\Leftrightarrow$ 3.' Es existiert eine Abbildung $g:N\to M$ mit $f\circ g = id_N$ und $f\circ f = id_M$ Soch eine Abbildung $g$ ist in diesem Fall automatisch bestimmt.
\end{itemize}
\subparagraph{Beispiel}
\label{sec-2-6-7-6-3}
Im Beispiel vorher haben wir gesehen $f:\mathbb{R}_{\geq 0} \to \mathbb{R}_{\geq 0}, x\mapsto x^2$ ist bijektiv.
Die Umkehrabbildung ist gegeben durch $f^{-1}:\mathbb{R}_{\geq 0} \to \mathbb{R}_{\geq 0}, x\mapsto \sqrt{x}$
\paragraph{Bemerkung}
\label{sec-2-6-7-7}
$M,N$ Mengen, $f:M\to N$ Dann gilt:
\begin{enumerate}
\item $f$ injektiv $\Leftrightarrow$ Es existiert $g:N\to M$ mit $g\circ f = id_M$ \\
        \textbf{Beweis:}
\begin{itemize}
\item "$\Leftarrow$" folgt aus \ref{sec-2-6-7-5}
\item "$\Rightarrow$" Sei $f$ injektiv. Sein $x$ ein beliebiges Element aus $M$
          Wir definieren \[g:N\to M,n\mapsto \begin{cases} x & n\not\in f(M) \\ \text{das eindeutig bestimmte Element $m\in M$ mit $f(m) = n$} & n\in f(M) \end{cases}\]
          Für alle $m\in M$ ist dann $(g\circ f)(m) = g(f(m)) = m$ das geißt $g\circ f = id_M$
\end{itemize}
\item $f$ surjektiv $\Leftrightarrow$ Es existiert $g:N\to M$ mit $f\circ g = id_N$ \\
        \textbf{Beweis:}
\begin{itemize}
\item "$\Leftarrow$" folgt aus \ref{sec-2-6-7-5}
\item "$\Rightarrow$" Sei $f$ surjektiv. Für jedes Element $n\in N$ wählen wir ein Element $\tilde n\in f^{-1}(\{n\}) \neq \emptyset$ und sehen
$g:N\to M,n\mapsto \tilde n$. Dann ist $(f\circ g)(n) = f(g(n)) = n$ für alle $n\in N$ und das heißt $f\circ g = id_N \hfill\square$
\end{itemize}
\end{enumerate}
\subparagraph{Anmerkung}
\label{sec-2-6-7-7-1}
Das wir stets einen Auswahlprozess wie im Beweis von 2. "$\Rightarrow$" vornehmen können ist ein Axiom der Mengenlehre (erkennen wir als gültig an, ist jedoch nicht beweisbar), das \textbf{Auswahlaxiom}: \\
      Ist $I$ eine Indexmenge und $(A_i)_{i\in I}$ eine Familie von nichtleeren Mengen, dann gibt es eine Abbildung $\gamma:I\to \bigcup_{i\in I} A_i$ mit $\gamma (i) \in A_i$ für alle $i\in I$ (im obigen Beweis ist $I = N,A_n = f^{-1}(\{n\})$ für $n\in N$)
\paragraph{Bemerkung 4.16}
\label{sec-2-6-7-8}
$L,M,N$ Mengen, $f:L\to M, g:M\to N$ \\
     Dann gilt: $g,f$ beide injektiv (beziehungsweise surjektiv oder bijektiv) $\Rightarrow$ $g\circ f$ injektiv (beziehungsweise sujektiv oder bijektiv)
\paragraph{Definition 4.17}
\label{sec-2-6-7-9}
\paragraph{Bemerkung 4.19}
\label{sec-2-6-7-10}
$M,N$ endliche Mengen mit $\abs{M} = \abs{N},f:M\to N$ Dann sind äquivalent:
\begin{enumerate}
\item $f$ ist injektiv
\item $f$ ist surjektiv
\item $f$ ist bijektiv
\end{enumerate}
\subparagraph{Beweis}
\label{sec-2-6-7-10-1}
\begin{itemize}
\item 1. $\Rightarrow$ 2. Sei $f$ injektiv $\Rightarrow$ $\abs{f(M)} = \abs{M} = \abs{N}$ wegen $f(M) \subseteq N$ folgt $f(M) = N$ $\Rightarrow$ $f$ surjektiv
\item 2. $\Rightarrow$ 3. Sei $f$ sujektiv, das heißt $f(M) = N$ \\
        Annahme: $f$ ist nicht bijektiv $\Rightarrow$ $f$ nicht injektiv $\Rightarrow$ $\exists m_1,m_2\in M: m_1\neq m_2 \wedge f(M_1) = f(m_2) \Rightarrow \abs{f(M)} < \abs{M} = \abs{N}$ Wiederspruch zu $f(M) = N$
\item 3. $\Rightarrow$ 1. trivial
\end{itemize}
\section{Gruppen, Ringe, Körper}
\label{sec-3}
\subsection{Gruppe}
\label{sec-3-1}
\subsubsection{Verknüpfung}
\label{sec-3-1-1}
$M$ Menge, Eine Verknüpfung (inverse Verknüpfung) auf $M$ ist ein Abbildung \[*:M\times M \to M\]
Anstelle von $*(a,b)$ schreiben wir $a * b$
\paragraph{Beispiel}
\label{sec-3-1-1-1}
\begin{itemize}
\item $+: \mathbb{R} \times \mathbb{R} \to \mathbb{R},(a,b) \mapsto a + b$
\item $\cdot: \mathbb{R} \times \mathbb{R} \to \mathbb{R},(a,b) \mapsto a\cdot b$
\end{itemize}
sind Verknüpfungen
\subsubsection{Monoid}
\label{sec-3-1-2}
Ein Monoid ist ein Tupel $(M,*)$, bestehend aus einer Menge $M$ und einer Verküpfung \\
    $*:M\times M \to M$, welche folgende Bedingungen genügt:
\begin{itemize}
\item (M1) Die Verküpfung ist assoziativ, das heißt \[\Forall a,b,c\in M:(a*b)*c = a*(b*c) \]
\item (M2) Ex existiert ein neutrales Element $e$ in $M$, das heißt \[\exists e\in M:\Forall a\in M e*a = a = a*e\]
\end{itemize}
\paragraph{Beispiel}
\label{sec-3-1-2-1}
\begin{itemize}
\item $(\mathbb{N}_0,+), (\mathbb{Z},+)$ sind Monoide (neutrales Element: $0$)
\item $(\mathbb{N},+)$ ist kein Monoid (ex existiert kein neutrales Element)
\item $(\mathbb{N},\cdot),(\mathbb{Z},\cdot)$ sind Monoide (neutrales Element: $1$)
\end{itemize}
\paragraph{Bemerkung}
\label{sec-3-1-2-2}
$(M,*)$ Monoid. Dann gibt es in $M$ genau ein neutrales Element.
\subparagraph{Beweis}
\label{sec-3-1-2-2-1}
\begin{itemize}
\item Existenz: Es existert ein neutrales Element: folgt aus Definition eines Monoids
\item Eindeutigkeit: Seien $e,\tilde e \in M$ neutrale Element \[\Rightarrow e = e * \tilde e = \tilde e\]
\end{itemize}
\subsubsection{Inverses}
\label{sec-3-1-3}
$(M,*)$ Monoid mit neutralem Element $e$, $a\in M$
Ein Element $b\in M$ heit Inverses zu $a \xLeftrightarrow{\text{Def}} a * b = e = b * a$
\paragraph{Beispiel}
\label{sec-3-1-3-1}
\begin{itemize}
\item In $(\mathbb{Z},+)$ ist $-2$ ein Inverses zu $2$ denn $2 + (-2) = 0 = (-2) + 2$
\item In $(\mathbb{N}_0,+)$ existiert kein Inverses zu $2$, denn es existiert kein $n\in \mathbb{N}_0$ mit $n + n = 0 = n + 2$
\item \label{invex} In $(\mathbb{Z},\cdot)$ existiert kein Inverses zu $2$, denn es existiert kein $n\in\mathbb{Z}$ mit $2\cdot n = 1 = n \cdot 2$
\end{itemize}
\paragraph{Bemerkung}
\label{sec-3-1-3-2}
\label{beminv}
$(M,*)$ Monid, $a\in M$ Dann gilt: besitzt $a$ ein Inverses, dann ist dieses eindeutig bestimmt.
\subparagraph{Beweis}
\label{sec-3-1-3-2-1}
Seinen $b,\tilde b$ Inversen zu $a$, sein $e\in M$ das neutrale Element
\[\Rightarrow b = e * b = (\tilde b * a) * b = \tilde b * (a * b) = \tilde b\]
\subsubsection{Gruppe}
\label{sec-3-1-4}
Eine Gruppe ist ein Tupel $(G,*)$, bestehen aus einer Menge $G$ und einer Verknüpfung $*:G\times G \to G$, sodass gilt:
\begin{itemize}
\item (G1) $(G,*)$ ist ein Monoid
\item (G2) Jedes Element aus $G$ besitzt ein Inverses
\end{itemize}
In diesem Fall schreiben wir $a'$ für das nach \ref{beminv} eindeutig bestimmte Inverse eines Elements $a\in G$
\paragraph{Beispiel}
\label{sec-3-1-4-1}
\begin{itemize}
\item $(\mathbb{Z},+)$ ist eine Gruppe, denn $(\mathbb{Z},+)$ ist ein Monoid und für $a\in\mathbb{Z}$ ist $-a$ das inverse Element: $a + (-a) = 0 = (-a) + a$
\item $(\mathbb{Z},\cdot)$ ist keine Gruppe, denn das Element $2\in\mathbb{Z}$ hat kein Inverses (vergleiche \ref{invex}).
\item $(\mathbb{Q}\setminus \{0\},\cdot)$ ist eine Gruppe denn es ist ein Monoid mit neutralem Element $1$ und für jedes Element $a\in\mathbb{Q}\setminus\{0\}$ existiert ein $b\in \mathbb{Q}\setminus \{0\}$ mit $a\cdot b = 1 = b\cdot a$, nämlich $b = \frac{1}{a}$
\end{itemize}
\paragraph{Bemerkung 5.11}
\label{sec-3-1-4-2}
$(G,*)$ Gruppe mit neutralem Element $e,a,b,c \in G$. Dann gilt
\begin{enumerate}
\item (Kürzungsregel) \[a*b = a*c \Rightarrow b = c\] \[a*c = b * c \Rightarrow a = b\]
\item $a*b = e \Rightarrow b = a'$
\item $(a')' = a$
\item (Regel von Hemd und Jacke) $(a*b)' = b' * a'$
\end{enumerate}
\subparagraph{Beweis}
\label{sec-3-1-4-2-1}
\begin{enumerate}
\item Sei $a * b = a * c \Rightarrow a'*(a*b) = a'*(a*c) \Rightarrow (a'*a)*b=(a'*a)*c \Rightarrow e*b = e*c \Rightarrow b = c$
\item aus 1. $a*b = c = a*a' \Rightarrow b = a'$
\item Es ist $a*a' = e = a' * a$, das heißt $a$ ist Inverses zu $a' \Rightarrow (a')' = a$
\item Es ist $(a*b)*(b'*a') = a*(b*b')*a' = a*a' = e \Rightarrow b' * a' \xRightarrow{\text{2.}} (a *b)'$
\end{enumerate}
\subsubsection{Abelsche Gruppe}
\label{sec-3-1-5}
$(M,*)$ Monoid / Gruppe heißt kommutativ (abelsch)
\[\xLeftrightarrow{\text{Def}} \Forall a,b\in M: a*b = b*a\]
\paragraph{Beispiel}
\label{sec-3-1-5-1}
Alle bisher betrachteten Beispiele von Monoiden beziehungsweise Gruppen sind abelsch
\paragraph{Bemerkung 5.14}
\label{sec-3-1-5-2}
$M$ Menge, Wir setzten $S(M):= \{f:M\to M | f~\text{bijektiv}\}$
Dann ist $(S(M),\circ)$ eine Gruppen, die \textbf{symmetrische} Gruppe auf $M$
\subparagraph{Beweis}
\label{sec-3-1-5-2-1}
\begin{enumerate}
\item "\$\^{}" ist wohl definiert, das heißt für $f,g\in S(M)$ ist $f\circ g \in S(M)$ folgt aus \ref{sec-2-6-7-10}
\item "\$\^{}" ist assoziativ $f\circ(g\circ h) = (f\circ g) \circ h \Forall f,g\in S(M)$ nach \texttt{4.9}
\item $id_M$ ist neutral: $id_M \in S(M)$ und $id_M\circ f = f = f\circ id_M \Forall f\in S(M)$
\item Existenz von Inversen: $f\in S(M) \Rightarrow f$ bijektiv $\Rightarrow$ Es existiert Umkehrabbildung $f^{-1}\in S(M)$ zu $f$
         für diese gilt: $f\circ f^{-1} = id_M = f^{-1}\circ f$ das heißt $f^{-1}$ ist immer zu $f$ bezüglich "\$\^{}"
\end{enumerate}
\subsubsection{Permutationen}
\label{sec-3-1-6}
$n\in\mathbb{N}$
\[S_n := S(\{1,\ldots,n\}) = \{\pi \{1,\ldots,n\} \to \{1,\ldots,n\} \mid \pi~\text{ist bijektiv}\}\]
$(S_n,\circ)$ heißt die symmetrische Gruppe auf $n$ Ziffern, Elemente aus $S_n$ heißen Permutationen.
Wir schreiben Permutationen $\pi \in S_n$ in der Form:
\begin{equation}
\pi =
\begin{pmatrix}
1 & 2 & \ldots & n \\
\pi(1) & \pi(2) & \ldots & \pi(n)
\end{pmatrix}
\end{equation}
\paragraph{Beispiel}
\label{sec-3-1-6-1}
In $S_3$ ist
\begin{equation}
\begin{pmatrix}
1 & 2 & 3 \\
1 & 3 & 2 \\
\end{pmatrix}
\circ
\begin{pmatrix}
1 & 2 & 3 \\
3 & 2 & 1 \\
\end{pmatrix}
=
\begin{pmatrix}
1 & 2 & 3 \\
2 & 3 & 1 \\
\end{pmatrix}
\end{equation}

\begin{equation}
\begin{pmatrix}
1 & 2 & 3 \\
3 & 2 & 1 \\
\end{pmatrix}
\circ
\begin{pmatrix}
1 & 2 & 3 \\
1 & 3 & 2 \\
\end{pmatrix}
=
\begin{pmatrix}
1 & 2 & 3 \\
3 & 1 & 2 \\
\end{pmatrix}
\end{equation}
das heißt $(S_3,\circ)$ ist nicht abelsch.
\subsubsection{Restklassen}
\label{sec-3-1-7}
\paragraph{Motivation}
\label{sec-3-1-7-1}
Im täglischen Leben verwendet man zur Bestimmung von Uhrzeiten das Rechnen "modulo $24$", zum Beispiel 22Uhr + 7h = 5Uhr. Wir wollen dies mathematisch präzisieren und verallgemeinern
\paragraph{Bemerkung 5.17}
\label{sec-3-1-7-2}
$n\in\mathbb{N}$. Dann ist durch \[a \sim b \xLeftrightarrow{\text{Def}} \Exists q\in\mathbb{Z}:a - b = q n\]
eine Äquivalenzrelatioin auf $\mathbb{Z}$ gegeben.
Anstelle von $a\sim b$ schreiben wir auch $a\equiv b(\mod n)$ ("$n$ ist kongruent $b$ modulo $n$")
Die Äquivalenzklasse von $a\in \mathbb{Z}$ ist durch
\[\bar a := \{b\in\mathbb{Z}\mid b\equiv a(\mod n)\} = a + n\mathbb{Z} := \{a + n q \mid q\in \mathbb{Z}\}\]
gegeben und heißt die Restklasse von $a$ modulo $n$.
Die Menge aller Restklassen modulo $n$ wird $\frac{\mathbb{Z}}{n\mathbb{Z}}$ bezeichnet ("$\mathbb{Z}$ modulo $n\mathbb{Z}$")
Es ist: \[\frac{\mathbb{Z}}{n\mathbb{Z}} = \{\bar 0, \bar 1, \ldots, \overline{n - 1}\}\]
und die Restklassen $\bar 0, \ldots, \overline{n - 1}$ sind paarweise verschieden
\subparagraph{Beweis}
\label{sec-3-1-7-2-1}
\begin{enumerate}
\item "$\equiv$" ist eine Äquivalenzrelation, denn:
\begin{itemize}
\item "$\equiv$" ist reflexiv: Fpr $a\in\mathbb{Z}$ ist $a\equiv a(\mod n)$ denn $a - a = 0 = 0 n$
\item "$\equiv$" ist symmetrisch: Seien $a,b\in\mathbb{Z}$ mit $a\equiv b(\mod n) \exists q\in\mathbb{Z}:a - b = q n$ \\
           $\Rightarrow$ $b - a = (-q) n \Rightarrow b \equiv a(\mod n)$
\item "$\equiv$" ist transitiv: Seien $a,b,c\in\mathbb{Z}$ mit $a\equiv b(\mod n), b\equiv c(\mod n)$
\begin{itemize}
\item $\Rightarrow$ $\exists q_1,q_2 \in\mathbb{Z}$ mit $a - b = q_1 n, b - c = q_2 n$
\item $\Rightarrow$ $a - c = (a - b) + (b - c) = q_1 n + q_2 n = (q_1 + 1_2) n \Rightarrow a \equiv c(\mod n)$
\end{itemize}
\end{itemize}
\item Die Äquivalenzklasse von $n\in\mathbb{Z}$ ist gegeben durch
\[\{b\in\mathbb{Z} \mid b = a(\mod n)\}\]
\[= \{b\in\mathbb{Z} \mid \exists q\in\mathbb{Z}:b - a = qn\}\]
\[= \{b\in\mathbb{Z} \mid \exists q\in\mathbb{Z}:b = a + q n\}\]
\[= a + n\mathbb{Z} \]
\item \[\frac{\mathbb{Z}}{n\mathbb{Z}} = \{\bar 0, \bar 1, \ldots, \overline{n - 1}\}\]
         denn:
\begin{itemize}
\item Ist $a\in\mathbb{Z}$ beliebig, dann liefert Division mit Rest durch $n$: \\
           Es gibt $q,r\in\mathbb{Z}$ mit $a = q n + r,0\leq r < n$
           \[\Rightarrow a - r = q n \Rightarrow q \equiv r(\mod n) \Rightarrow \bar a = \bar r\]
           Das heißt: Jede Restklasse ist von der Form $\bar r$ mit $r\in \{0,\ldots,n - 1\}$ \\
\item Die Restklassen $\bar 0, \bar 1, \ldots, \overline{n - 1}$ sind paarweise verschieden denn: \\
           Seien $a,b\in\{0,\ldots,n - 1\}$ mit $\bar a = \bar b \Rightarrow a \equiv b(\mod n) \Rightarrow \exists q\in \mathbb{Z}: a - b = q n \Rightarrow \abs{a - b} = \abs{q} n$.
\begin{itemize}
\item Wäre $q\neq 0$, dann $\abs{q} \geq 1$ wegen $q\in\mathbb{Z} \Rightarrow \abs{a - b} \geq n$ \textbf{Wiederspruch} zu $a,b\in\{0,\ldots,n - 1\}$ \\
             Also: $q = 0$ das heißt $a = b$
\end{itemize}
\end{itemize}
\end{enumerate}
\paragraph{Beispiel}
\label{sec-3-1-7-3}
$n = 3: a\equiv b (\mod 3) \Leftrightarrow \exists q\in\mathbb{Z}: a - b = 3 q$ \\
     zum Beispiel: $11 \equiv 5(\mod 3)$, denn $11 - 5 = 6 = 2 \cdot 3$ \\
     zum Beispiel: $7 \not\equiv 2(\mod 3)$, denn $7 - 2 = 5$ und es gibt kein $q\in\mathbb{Z}$ mit $5 = 3 q$
\[\bar 0 = \{a\in\mathbb{Z}\mid a \equiv 0 (\mod 3)\} = \{a\in\mathbb{Z} \mid  \exists q\in\mathbb{Z}: a = 3q\} = 3\mathbb{Z} = \{\ldots,-6,-3,0,3,6,\ldots\}\]
\[\bar 1 = \{a\in\mathbb{Z}\mid a \equiv 1 (\mod 3)\} = \{a\in\mathbb{Z} \mid  \exists q\in\mathbb{Z}: a - 1 = 3q\} = 1 + 3\mathbb{Z} = \{\ldots,-5,-2,1,4,7,\ldots\}\]
\[\bar 2 = \{a\in\mathbb{Z}\mid a \equiv 2 (\mod 3)\} = \{a\in\mathbb{Z} \mid  \exists q\in\mathbb{Z}: a - 2 = 3q\} = 2 + 3\mathbb{Z} = \{\ldots,-4,-1,2,5,8,\ldots\}\]
\[\bar 3 = \{a\in\mathbb{Z}\mid a \equiv 3 (\mod 3)\} = \{a\in\mathbb{Z} \mid  \exists q\in\mathbb{Z}: a - 3 = 3q\} = \{a\in\mathbb{Z}\mid \exists q\in\mathbb{Z}:a=3(q + 1)\}3\mathbb{Z} = \bar 0\]
\[\bar 4 = \bar 1,\bar 5 = \bar 2,\overline{-1} = \bar 2\]
\paragraph{Bemerkung 5.18}
\label{sec-3-1-7-4}
$n\in\mathbb{N}$ wir definieren eine Verküpfung (Addition) auf $\frac{\mathbb{Z}}{n\mathbb{Z}}$ wie folgt: \\
     Für $\bar a,\bar b \in\frac{\mathbb{Z}}{n\mathbb{Z}}$ setzen wir $\bar a + \bar b = \overline{a + b}$
Dann gilt $(\frac{\mathbb{Z}}{n\mathbb{Z}},+)$ ist eine abelsche Gruppe
\subparagraph{Beweis}
\label{sec-3-1-7-4-1}
\begin{enumerate}
\item Die Verknüpfung ist wohldefiniert: \\ Problem: Die Addition verweendet Vertreter von Restklassen. Es ist zum Beispiel in $\frac{\mathbb{Z}}{n\mathbb{Z}}: \bar 3 + \bar 4 = \overline{3 + 4} = \bar 7 = \bar 2$, aber man könnte auch Rechnen:
$\bar 3 + \bar 4 = \bar 8 + \bar 9 = \overline{8 + 9} = \overline{17} = \bar 2$ \\
         Wir müssen nachweisen, dass die Wahl der Vertreter keinen Einfluss auf das Ergebnis hat, das heißt die Verknüfung ist "vertreter unahbhängig": \\
         Seien $a_1,a_2 ,b_1,b_2 \in\mathbb{Z},\overline{a_1} = \overline{a_2},\overline{b_1} = \overline{b_2}$
\begin{align}
&\Rightarrow a_1 \equiv a_2(\mod n), b_1 \equiv b_2(\mod n) \\
&\Rightarrow\exists q_1,q_2\in\mathbb{Z}: a_1 - a_2 = q_1 n, b_1 - b_2 = q_2 n n, b_1 - b_2 = q_2 n \\
&\Rightarrow (a_1 + b_1) - (a_2 + b_2) = (a_1 - a_2)+ (b_1 - b_2) = q_1 n + q_2 n = (q_1 + q_2) n \\
&\Rightarrow a_1 + b_ 1 \equiv a_2 + b_2 (\mod n) \\
&\Rightarrow \overline{a_1 + b_1} = \overline{a_2 + b_2}
\end{align}
\item $(\frac{\mathbb{Z}}{n\mathbb{Z}})$ ist eine abelsche Gruppe:
\begin{itemize}
\item Assoziativgesetz: Für alle $a,b,c\in\mathbb{Z}$ ist
\[(\bar a + \bar b) + \bar c = \overline{a + b} + \bar c = \overline{(a + b) + c} = \overline{a + (b + c)} = \bar a + \overline{b + c} = \bar a + (\bar b + \bar c)\]
\item $\bar 0$ ist neutrales Element, denn $\Forall a\in\mathbb{Z}:\bar 0 + \bar a = \overline{0 + a} = \bar a = \bar a + \bar 0$
\item Für $a\in\mathbb{Z}$ inst $\overline{-a}$ das inverse Element zu $\bar a$, denn $\bar a + \overline{-a} = \overline{a + (- a)} = \bar 0 = \overline{-a} + \bar a$
\item Kommutativgesetz: $\Forall a,b\in\mathbb{Z}:\bar a + \bar b = \overline{a + b} = \overline{b + a} = \bar b + \bar a$
\end{itemize}
\end{enumerate}
\subparagraph{Beispiel}
\label{sec-3-1-7-4-2}
Wir tragen die Ergebnisse der Verknüpfung "$+$" in einer Verknüpfungstafel zusamme:
n = 3
\begin{center}
\begin{tabular}{llll}
$+$ & $\bar 0$ & $\bar 1$ & $\bar 2$\\
\hline
$\bar 0$ & $\bar 0$ & $\bar 1$ & $\bar 2$\\
$\bar 1$ & $\bar 1$ & $\bar 2$ & $\bar 0$\\
$\bar 2$ & $\bar 2$ & $\bar 0$ & $\bar 1$\\
\end{tabular}
\end{center}

n = 4
\begin{center}
\begin{tabular}{lllll}
$+$ & $\bar 0$ & $\bar 1$ & $\bar 2$ & $\bar 3$\\
\hline
$\bar 0$ & $\bar 0$ & $\bar 1$ & $\bar 2$ & $\bar 3$\\
$\bar 1$ & $\bar 1$ & $\bar 2$ & $\bar 3$ & $\bar 0$\\
$\bar 2$ & $\bar 2$ & $\bar 3$ & $\bar 0$ & $\bar 1$\\
$\bar 3$ & $\bar 3$ & $\bar 0$ & $\bar 1$ & $\bar 2$\\
\end{tabular}
\end{center}
\subsubsection{Gruppenhomomorphismus}
\label{sec-3-1-8}
$(G,+),(H,\oast), \varphi : G \to H$ Abbildung \\
    $\varphi$ heißt ein Gruppenhomomorphismus $\xLeftrightarrow{\text{Def}} \Forall a,b,c\in G: \varphi(a*b) = \varphi(a) \oast \varphi(b)$ \\
    $\varphi$ heißt ein Gruppenisomorphismus $\xLeftrightarrow{\text{Def}} \varphi$ ist bijektiver Gruppenhomomorphismus
\paragraph{Beispiel}
\label{sec-3-1-8-1}
\begin{enumerate}
\item $\varphi:\mathbb{Z} \to \mathbb{Z}, a\mapsto 2 a$ ist Gruppenhomomorphismus von $(\mathbb{Z},+)$ nach $(\mathbb{Z},+)$ denn:
\[\varphi( a+ b) = 2(a + b) = 2 a + 2 b = \varphi(a) + \varphi(b) \Forall a,b\in\mathbb{Z}\]
$\varphi$ ist aber kein Gruppenisomorphismus, denn $\varphi$ ist nicht surjektiv $(1\not\in \varphi = \varphi{\mathbb{Z}})$
\item $n\in\mathbb{N}$. Dann gilt $\varphi:\mathbb{Z}\to\frac{\mathbb{Z}}{n\mathbb{Z}},a\mapsto\bar a$ ist ein Gruppenhomomorphismus von $(\mathbb{Z},+)$ nach $(\frac{\mathbb{Z}}{n\mathbb{Z}}, +)$, denn
\[\Forall a,b\in\mathbb{Z}:\varphi(a+b) = \overline{a + b} = \bar a + \bar b =\varphi(a) + \varphi(b)\]
$\varphi$ ist kein Gruppenisomorphismus, denn $\varphi$ ist nicht injektiv ($\varphi(0) = \bar 0 = \bar n = \varphi(n)$, aber $0\neq n$)
\item $\varphi:\mathbb{Z}\to\mathbb{Z},a\mapsto a + 1$ ist kein Gruppenhomomorphismus von $(\mathbb{Z},+)$ nach $(\mathbb{Z},+)$, denn
\[\varphi(2 + 6) = \varphi(8) = 9,~\text{aber}~\varphi(2)+\varphi(6) = 3 + 7 = 10\]
\item $\exp:\mathbb{R}\to\mathbb{R}_{\geq 0}, x\mapsto \exp{x} = e^x$ ist ein Gruppenisomorphismus von $(\mathbb{R},+)$ nach $(\mathbb{R}_{\geq 0},\cdot)$, denn:
\begin{itemize}
\item \[\exp(a + b) = \exp(a)\exp(b) \Forall a,b\in\mathbb{R}\]
\item $exp$ ist bijektiv (vgl. Ana1 - Vorlesung)
\end{itemize}
\end{enumerate}
\paragraph{Bemerkung 5.23}
\label{sec-3-1-8-2}
$(G,*),(H,\oast)$ Gruppen mit neutralen Elementen $e_G$ beziehungsweise $e_H,\varphi:G\to H$ Gruppenhomomorphismus. Dann gilt
\begin{enumerate}
\item $\varphi(e_G) = e_H$
\item $\Forall a\in G:\varphi(a') = \varphi(a)'$ (Hierbei ist $'$ das Inverse)
\item Ist $\varphi$ Gruppenisomorphismus, dann gilt $\varphi^{-1}:H\to G$ ebenfalls Gruppenisomorphismus
\end{enumerate}
$(G,*),(H,\oast)$ heißen isomorph $\xLeftrightarrow{\text{Def}}$ Ex existert ein Gruppenisomorphismus $\phi:G\to H$ Wir schreiben dann $(G,*) \cong (H,\oast)$
\subparagraph{Beweis}
\label{sec-3-1-8-2-1}
\begin{enumerate}
\item Es $e_H\oast \varphi(e_G) = \varphi(e_G) = \varphi(e_G*e_G) = \varphi(e_G) \oast(e_G) \Rightarrow e_H = \varphi(e_G)$
\item Sei $a\in G$ Dann ist $e_H = \varphi(e_G) = \varphi(a*a') = \varphi(a)\oast(a') \Rightarrow \varphi(a') = \varphi(a)'$
\item $\varphi^{-1}$ ist bijektiv, noch zu zeigen: $\varphi^{-1}$ ist ein Gruppehomomorphismus, das heißt
\[\varphi^{-1}(c\oast d) = \varphi^{-1}(c)*\varphi^{-1}(d) \Forall c,d\in H\]
Seien $c,d\in H$ Weil $\varphi$ bijektiv: $\exists a,b\in G:\varphi(a) = c,\varphi(b) =d$
\[\Rightarrow \varphi^{-1}(c\oast d) = \varphi^{-1}(\varphi(a)*\varphi(b)) = \varphi^{-1}(\varphi(a*b)) = a*b = \varphi^{-1}(c)*\varphi^{-1}(d)\hfill\square\]
\end{enumerate}
\subsection{Ring}
\label{sec-3-2}
Ein Ring ist ein Tupel $(R,+,\cdot)$, bestehend aus einer Menge R und 2 Verknüpfungen:
\begin{itemize}
\item $+:R\times R \to R,(a,b)\mapsto a + b$ \hfill genannt Addition
\item $\cdot:R\times R\to R, (a,b)\mapsto a\cdot b$ \hfill genannt Multiplikation
\end{itemize}
welche den folgenden Bedingungen genügen
\begin{itemize}
\item (R1) $(R,+)$ ist eine abelsche Gruppe
\item (R2) $(R,\cdot)$ ist ein Monoid
\item (R3) Es gelten die Distributivgeseze, das heisßt
\[\Forall a,b,c\in R: a\cdot(a + b) = a\cdot b + a\cdot c, (a+b)\cdot c = a\cdot c + b\cdot c\]
\end{itemize}
Ein Ring heißt \textbf{kommutativ} $\xLeftrightarrow{\text{Def}}$ die Multiplikation ist kommutativ, das heißt $\Forall a,b\in R: a\cdot b = b\cdot a$
\subsubsection{Anmerkung}
\label{sec-3-2-1}
\begin{itemize}
\item ohne Klammerung gilt die Konvention "$\cdot$" vor "$+$", "$\cdot$" wird häufig weggelassen
\item das neutrale Element bezüglich "$+$" bezeichnen wri mit $0_R$ (Nullelement), das neutrale Element bezüglich "$\cdot$" mit $1_R$ (Einselement). Das zu $a\in R$ bezüglich "$+$" inverse Element bezeichnen wir mit $-a$,
für $a + (-b)$ schreben wir $a - b$. Existiert zu $a\in R$ ein Inverses bezüglich "$\cdot$", so bezeichnen wir diesses mit $a^{-1}$
\item Wir schreiben häufig verkürzend "$R$ Ring" statt "$(R,+,\cdot)$ Ring"
\item In der Literatur wird gelegentlich die Forderung der Existenz eines neutralen Elements bezüglich "$\cdot$" weggelassen, "unser" Ringbegriff entspricht dort dem Begriff "Ring mit Eins"
\end{itemize}
\subsubsection{Beispiel}
\label{sec-3-2-2}
\begin{enumerate}
\item $(\mathbb{Z},+,\cdot)$ ist ein kommutativer Ring
\item Nullring $(\{0\},+,\cdot)$ mit $0 + 0 = 0, 0\cdot 0 = 0$ ist ein kummutativer Ring \\
       (hier ist Nullelement = Einselement = 0). Wir bezeichnen den Nullring kurz mit $0$.
\end{enumerate}
\subsubsection{Bemerkung 6.3}
\label{sec-3-2-3}
$R$ Ring. Dann gilt:
\begin{enumerate}
\item $0_R\cdot a = 0_R = a\cdot 0_R\Forall a\in R$
\item $a\cdot (-a) = - a b = (-a) \cdot b \Forall a,b\in R$
\item Ist $R\neq 0$, dann ist $1_R\neq 0_R$
\end{enumerate}
\paragraph{Beweis}
\label{sec-3-2-3-1}
\begin{enumerate}
\item $0_R + 0_R\cdot a = 0_R\cdot a = (0_R + 0_R)\cdot a = 0_R\cdot a + 0_R\cdot \xRightarrow{\text{"kürzen s. [[Bemerkung 5.11]]"}} 0_R = 0_R \cdot a$, $a\cdot 0_R = 0_R$ analog
\item $0_R = 0_R\cdot b = (a + (-a))\cdot b = a\cdot b + (-a) \cdot b \Rightarrow{\text{[[Bemerkung 5.11]]}} - a b = (-a)\cdot b$, $a\cdot(-b) 0 -a b$ analog
\item Beweis durch Kontraposition: Sei $1_R = 0_R$
        \[\Rightarrow \Forall a\in R: a = a\cdot 1_R = a\cdot 0_R = 0_R\]
        das heißt $R = 0\hfill\square$
\end{enumerate}
\subsubsection{Bemerkung 6.4}
\label{sec-3-2-4}
$n\in\mathbb{N}$ Für $\bar a, \bar b \in\frac{\mathbb{Z}}{n\mathbb{Z}}$ setzen wir $\bar a + \bar b := \overline{a + b}, \bar a\cdot \bar b := \overline{ab}$, dann ist $(\frac{\mathbb{Z}}{n\mathbb{Z}},+,\cdot)$ ein kommutativer Ring.

Wenn wir ab jetzt vom Ring $\frac{\mathbb{Z}}{n\mathbb{Z}}$ sprechen, dann meinen wir $(\frac{\mathbb{Z}}{n\mathbb{Z}},+,\cdot)$ mit den obigen Verknüpfungen
\paragraph{Beweis}
\label{sec-3-2-4-1}
\begin{enumerate}
\item Multiplikaiton ist wohldefiniert (das heißt "vertreterunabhängig", vergleiche \ref{sec-3-1-7-4}) \\
        Sei $a_1,a_2,b_1,b_2 \in\mathbb{Z}$ mit $\overline{a_1} = \overline{a_2},\overline{b_2} = \overline{b_2}$
\begin{align}
&\Rightarrow a_1 \equiv a_2 (\mod n), b_1\equiv b_2 (\mod n) \\
&\Rightarrow \exists q_1,q_2\in\mathbb{Z}:a_1 - a_2 = q_1 n, b_1 - b_2 = q_2 n \\
&\Rightarrow a_1 b_2 - a_2 b_2 = a_1(b_1 - b_2) + b_2 (a_1 - a_2) = a_q q_2 n + b_2 q_1 n = (a_1 q_2 + b_2 q_1) n \\
&\Rightarrow a_1 b_1 \equiv a_2 b_2 (\mod n) \\
&\Rightarrow \overline{a_1 b_1} = \overline{a_2 b_2}
\end{align}
\item Multiplikation ist assoziativ, Für $a,b,c\in\mathbb{Z}$ ist
\[\bar a\cdot (\bar b\cdot \bar c) = \bar a \cdot \overline{a\cdot c} = \overline{a\cdot(b\cdot c)} = \overline{(a\cdot b)\cdot c} = \overline{a\cdot b} \cdot \bar c = (\bar a\cdot \bar b) \cdot \bar c\]
\item Existenz eines Enselements: $\Forall a\in\mathbb{Z}:\bar 1 \cdot \bar a = \overline{1\cdot a} = \bar a = \bar a\cdot \bar 1$
\item Multiplikation ist kommutativ:
\[\Forall a,b\in\mathbb{Z}:\bar a\cdot \bar b = \overline{a\cdot b} = \overline{b\cdot a} = \bar b \cdot \bar a\]
\item $(\frac{\mathbb{Z}}{n\mathbb{Z}},+)$ ist abelsche Gruppe nach \ref{sec-3-1-7-4}
\item Distributivgesetz:
\begin{align}
\bar a\cdot (\bar b + \bar c) &= \bar a \cdot \overline{b + c} \\
&= \overline{a\cdot (b + c)} \\
&= \overline{a\cdot b + a\cdot c} \\
&= \overline{a\cdot b} + \overline{a\cdot c}
&= \bar a\cdot \bar b + \bar a \cdot\bar c
\end{align}
$(\bar a + \bar b)\cdot \bar c = \bar a\cdot \bar c + \bar b \cdot \bar c$ folgt wegen Kommutativität der Multiplikation
\end{enumerate}
\paragraph{Beispiel 6.5}
\label{sec-3-2-4-2}
Verknüpfungstafeln für $\frac{\mathbb{Z}}{n\mathbb{Z}}$
n = 3:
\begin{center}
\begin{tabular}{llll}
$+$ & $\bar 0$ & $\bar 1$ & $\bar 2$\\
\hline
$\bar 0$ & $\bar 0$ & $\bar 1$ & $\bar 2$\\
$\bar 1$ & $\bar 1$ & $\bar 2$ & $\bar 0$\\
$\bar 2$ & $\bar 2$ & $\bar 0$ & $\bar 1$\\
\end{tabular}
\end{center}

\begin{center}
\begin{tabular}{llll}
$\cdot$ & $\bar 0$ & $\bar 1$ & $\bar 2$\\
\hline
$\bar 0$ & $\bar 0$ & $\bar 0$ & $\bar 0$\\
$\bar 1$ & $\bar 0$ & $\bar 1$ & $\bar 2$\\
$\bar 2$ & $\bar 0$ & $\bar 2$ & $\bar 1$\\
\end{tabular}
\end{center}
n = 4:
\begin{center}
\begin{tabular}{lllll}
$+$ & $\bar 0$ & $\bar 1$ & $\bar 2$ & $\bar 3$\\
\hline
$\bar 0$ & $\bar 0$ & $\bar 1$ & $\bar 2$ & $\bar 3$\\
$\bar 1$ & $\bar 1$ & $\bar 2$ & $\bar 3$ & $\bar 0$\\
$\bar 2$ & $\bar 2$ & $\bar 3$ & $\bar 0$ & $\bar 1$\\
$\bar 3$ & $\bar 3$ & $\bar 0$ & $\bar 1$ & $\bar 2$\\
\end{tabular}
\end{center}

\begin{center}
\begin{tabular}{lllll}
$\cdot$ & $\bar 0$ & $\bar 1$ & $\bar 2$ & $\bar 3$\\
\hline
$\bar 0$ & $\bar 0$ & $\bar 0$ & $\bar 0$ & $\bar 0$\\
$\bar 1$ & $\bar 0$ & $\bar 1$ & $\bar 2$ & $\bar 3$\\
$\bar 2$ & $\bar 0$ & $\bar 2$ & $\bar 0$ & $\bar 2$\\
$\bar 3$ & $\bar 0$ & $\bar 3$ & $\bar 2$ & $\bar 1$\\
\end{tabular}
\end{center}

In $\frac{\mathbb{Z}}{n\mathbb{Z}}$ ist $\bar 2 \cdot \bar 2 = \bar 0$, aber $\bar 2\neq \bar 0$.
\subsubsection{Integritätsbereich}
\label{sec-3-2-5}
\label{Definition-6.6}
ist ein kommuativer Ring $(R,+,\cdot)$ mit $R\neq 0$, in dem gilt:
\[\Forall a,b\in R: a\cdot b = 0_R \Rightarrow a = 0_R\vee b = 0_R\]
beziehungsweise äquivalent dazu:
\[a\neq 0_R \wedge b\neq 0_R \Rightarrow a\cdot b \neq 0_R\]
\paragraph{Beispiel 6.7}
\label{sec-3-2-5-1}
\begin{itemize}
\item $\frac{\mathbb{Z}}{3\mathbb{Z}}$ ist ein Integritätsbereich, $\frac{\mathbb{Z}}{4\mathbb{Z}}$ ist kein Integritätsbereich, denn $\bar 2\cdot \bar 2 = \bar 0$, aber $\bar 2 \neq \bar 0$
\end{itemize}
\paragraph{Bemerkung 6.8}
\label{sec-3-2-5-2}
$n\in\mathbb{N}$ Dann sind äquivalent
\begin{enumerate}
\item \label{6.8.1} $\frac{\mathbb{Z}}{n\mathbb{Z}}$ ist ein Integritätsbereich
\item \label{6.8.2} $n$ ist eine Primzahl
\end{enumerate}
\subparagraph{Beweis}
\label{sec-3-2-5-2-1}
\ref{6.8.1} $\Rightarrow$ \ref{6.8.2} zeigen wir durch Kontraposition, das heißt \$$\neq$\$\ref{6.8.2} $\Rightarrow$ $\neq$ \ref{6.8.1} \\
      Sei $n\in\mathbb{N}$ keine Primzahl. Falls $n = 1$ dann ist $\frac{\mathbb{Z}}{n\mathbb{Z}} = \{\bar 0\}$ (Nullring), das heißt $\frac{\mathbb{Z}}{n\mathbb{Z}}$ ist kein Integritätsbereich. Seim im Folgenden $n > 1$ und keine Primzahl.
\begin{align}
&\Rightarrow \exists a,b\in\mathbb{N}:1\ <a,b<n \wedge n = a\cdot b \\
&\Rightarrow \bar 0 = \bar n = \overline{a b} = \bar a \cdot \bar b
\end{align}
und es ist $bar a,\bar b\neq \bar 0$ $\Rightarrow$ $\frac{\mathbb{Z}}{n\mathbb{Z}}$ kein Integrationsbereich. \\
      \ref{6.8.2} $\Rightarrow$ \ref{6.8.1}: Sein $n$ eine Primzahl $\Rightarrow$ $n > 1$, insbesondere $\frac{\mathbb{Z}}{n\mathbb{Z}} \neq 0$. Seien $\bar a, \bar b \in \frac{\mathbb{Z}}{n\mathbb{Z}}$ mit $\bar a\cdot \bar b = \bar 0$
\[\Rightarrow \exists q\in\mathbb{Z}:a b = q n\]
Da $n$ Primzahl, kommt $n$ n der Primfaktorzerlengung von $a b$ als Primfaktor vor \\
      $\Rightarrow$ $n$ kommt in der Primfaktorzerlegung von $a$ oder $b$ als Primfaktor vor \[\Rightarrow n\mid a \vee n\mid b \Rightarrow \bar a = \bar 0 \vee \bar b = \bar 0\]
\subsection{Körper}
\label{sec-3-3}
\label{Definition-6.9}
Ein Körper ist ein kommutativer Ring $(K,+,\cdot)$, in dem gilt $K\neq 0$ und jedes Element $a\in K, a\neq 0$ besitzt ein Inverses in $K$ bezüglich "$\cdot$", das heißt: $\exists b\in K:a\cdot b = 1_K$. Wir setzen \$K$^{\ast}$ := K$\setminus$\{0\}
\subsubsection{Beispiel}
\label{sec-3-3-1}
\begin{enumerate}
\item $(\mathbb{R},+,\cdot),(\mathbb{Q},+,\cdot)$ sind Körper (mit den üblichen $+,\cdot$)
\item $\frac{\mathbb{Z}}{3\mathbb{Z}}$ ist ein Körper (betrachte Verknüpfungstafel)
\item $\frac{\mathbb{Z}}{4\mathbb{Z}}$ ist ein kein Körper: Das Element $\bar 2$ besitzt kein Inverses bezüglich "$\cdot$"
\end{enumerate}
\subsubsection{Bemerkung 6.11}
\label{sec-3-3-2}
$K$ Körper, Dann gilt:
\begin{enumerate}
\item $0_K \neq 1_K$
\item \label{6.11.2} $K$ ist ein Integritätsbereich
\item $(K^\ast,\cdot)$ ist eine abelsche Gruppe mit neutralem Element $1_K$
\end{enumerate}
\paragraph{Beweis}
\label{sec-3-3-2-1}
\begin{enumerate}
\item folgt aus \ref{sec-3-2-3}
\item $K\neq 0$ nach Definition. Seien $a,b\in K$ mit $a b = 0_K$. Falls $a\neq 0_K$ dann
\[b = 1_K \cdot b = (a^{-1} a)\cdot b = a^{-a}(a b) = a^{-1}\cdot 0_K = 0_K\]
Insebesondere gilt: $a = 0\vee b = 0$
\item $K^\ast\times K^\ast \to K^\ast$ ist wohldefiniert nach \ref{6.11.2} (aus $a,b\in K^\ast$ folgt $a b\in K^\ast$) \\
        Da $(K,\cdot)$ abelscher Monoid mit neutralem Element $1_K$ ist auch $(K^\ast,\cdot)$ abelscher Monid mit neutralem Element $1_K$.
Nach \ref{Definition-6.9} besitzt jedes Element $a\in K^\ast$ ein Inverses $b\in K$ mit $a b = 1_K$ Wegen $0_K \neq 1_K$ ist $b\neq 0_K$ (sonst $a b = a\cdot 0_K = 0_K \neq 1_K$), das heißt $b\in K^\ast\hfill\square$
\end{enumerate}
\subsubsection{Bemerkung 6.12}
\label{sec-3-3-3}
$R$ Integritätsbereich, der nur endlich viele Elemente hat. Dann ist $R$ ein Körper.
\paragraph{Beweis}
\label{sec-3-3-3-1}
$R$ Integritätsbereich $\Rightarrow$ $R\neq 0$ \\
     Noch zu zeigen: $a\in R\setminus\{0_R\} \Rightarrow \exists b\in R: a b = 1_R$
Sei $a\in R\setminus\{0_R\}$. Wir betrachten die Abbildung $\varphi_a: R\to R,x\mapsto a x$
\begin{enumerate}
\item Behauptung: $\varphi_a$ ist injektiv, denn:
\begin{align}
\intertext{Seien $x,y\in R$ mit}
\varphi_a(x) =\varphi_a(y) \Rightarrow a x = a y \Rightarrow a x + (-(a y)) = 0_R \\
\intertext{Mit [[Bemerkung 6.3]] folgt:}
\Rightarrow a x + a(-a) = -R \Rightarrow a(x - y) = 0_R  \\
\intertext{Aus $R$ Integrationsbereich und $a\neq 0$ folgt:}
x - y = 0 \Rightarrow x = y
\end{align}
\item Da $R$ endlich ist und $\varphi_a$ inejktiv ist, ist $\varphi_a$ nach \ref{sec-2-6-7-10} surjektiv
\[\Rightarrow  \exists b\in R: \varphi_a(b) = 1_R \Rightarrow a b = 1_R\]
\end{enumerate}
\subsubsection{Folgerung 6.13}
\label{sec-3-3-4}
$n\in\mathbb{N}$ Dann sind äquivalent
\begin{enumerate}
\item \label{6.13.1} $\frac{\mathbb{Z}}{n\mathbb{Z}}$ ist ein Körper
\item \label{6.13.2} $n$ ist eine Primzahl
\end{enumerate}
\paragraph{Beweis}
\label{sec-3-3-4-1}
\ref{6.13.1} $\Rightarrow$ \ref{6.13.2} durch Kontraposition: $\neq$ \ref{6.13.2} $\Rightarrow$ \ref{6.13.1} \\
     Sei $n$ keine Primzahl $\Rightarrow$ $\frac{\mathbb{Z}}{n\mathbb{Z}}$ kein Integritätsbereich $\Rightarrow$ $\frac{\mathbb{Z}}{n\mathbb{Z}}$ kein Körper \\
     \ref{6.13.2} $\Rightarrow$ \ref{6.13.1} Sei $n$ eine Primzahl $\Rightarrow$ $\frac{\mathbb{Z}}{n\mathbb{Z}}$ Integritätsbereich, der nur endlich viele Elemente hat $\Rightarrow$ $\frac{\mathbb{Z}}{n\mathbb{Z}}$ Körper
\paragraph{Notation}
\label{sec-3-3-4-2}
$p$ Primzahl. Man nennt $\mathbb{F}_P := \frac{\mathbb{Z}}{p\mathbb{Z}}$ auch den endlichen Körper mit $p$ Elemente
% Emacs 25.1.1 (Org mode 8.2.10)
\end{document}