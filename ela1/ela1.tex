% Created 2016-10-28 Fr 11:12
\documentclass[11pt]{article}
\usepackage[utf8]{inputenc}
\usepackage[T1]{fontenc}
\usepackage{fixltx2e}
\usepackage{graphicx}
\usepackage{longtable}
\usepackage{float}
\usepackage{wrapfig}
\usepackage{rotating}
\usepackage[normalem]{ulem}
\usepackage{amsmath}
\usepackage{textcomp}
\usepackage{marvosym}
\usepackage{wasysym}
\usepackage{amssymb}
\usepackage{hyperref}
\tolerance=1000
\usepackage{siunitx}
\usepackage{fontspec}
\sisetup{load-configurations = abbrevations}
\newcommand{\estimates}{\overset{\scriptscriptstyle\wedge}{=}}
\usepackage{mathtools}
\DeclarePairedDelimiter\abs{\lvert}{\rvert}%
\DeclarePairedDelimiter\norm{\lVert}{\rVert}%
\DeclareMathOperator{\Exists}{\exists}
\DeclareMathOperator{\Forall}{\forall}
\def\colvec#1{\left(\vcenter{\halign{\hfil$##$\hfil\cr \colvecA#1;;}}\right)}
\def\colvecA#1;{\if;#1;\else #1\cr \expandafter \colvecA \fi}
\author{Robin Heinemann}
\date{\today}
\title{Lineare Algebra (Vogel)}
\hypersetup{
  pdfkeywords={},
  pdfsubject={},
  pdfcreator={Emacs 25.1.1 (Org mode 8.2.10)}}
\begin{document}

\maketitle
\tableofcontents


\section{Einleitung}
\label{sec-1}
Übungsblätter/Lösungen:
jew. Donnerstag / folgender Donnerstag
Abgabe Donnerstag 9:30
50\% der Übungsblätter
\subsection{Plenarübung}
\label{sec-1-1}
Aufgeteilt
\subsection{Moodle}
\label{sec-1-2}
Passwort: vektorraumhomomorphismus
\subsection{Klausur}
\label{sec-1-3}
24.02.2017
\section{Grundlagen}
\label{sec-2}
\subsection{Naive Aussagenlogik}
\label{sec-2-1}
naive Logik: wir vewenden die sprachliche Vorstellung ($\neq$ mathematische Logik: eigne Vorlesung)
Eine Aussage ist ein festehender Satz, dem genau einer der Wahrheitswerte "wahr" oder "falsch" zugeordnet werden kann.
Aus einfachen Aussagen kann man durch logische Verknüpfungen kompliziertere Aussagen bilden.
Angabe der Wahrheitswertes der zusammengesetzten Aussage erfolgt duch Wahrheitstafeln (liefern den Wahrheitswert der zusammengesetzten Aussage, aus dem Wahrheitswert der einzelnen Aussagen).
Im folgenden seien $A$ und $B$ Aussagen.
\begin{itemize}
\item Negation (NICHT-Verknüpfung)
\begin{itemize}
\item Symbol: \$$\neg{}$
\item Wahrheitstafel:
\begin{center}
\begin{tabular}{ll}
$A$ & $\neg A$\\
\hline
w & f\\
f & w\\
\end{tabular}
\end{center}
\item Beispiel: $A$: 7 ist eine Primzahl (w)
$\neg A$: 7 ist keine Primzahl (f)
\end{itemize}

\item Konjunktion (UND-Verknüpfung)
\begin{itemize}
\item Symbol $\wedge$
\item Wahrheitstafel:
\begin{center}
\begin{tabular}{lll}
$A$ & $B$ & $A\wedge B$\\
\hline
w & w & w\\
w & f & f\\
f & w & f\\
f & f & f\\
\end{tabular}
\end{center}
\end{itemize}

\item Disjunktion (ODER-Verknüpfung)
\begin{itemize}
\item Symbol: $\vee$
\item Wahrheitstafel:
\begin{center}
\begin{tabular}{lll}
$A$ & $B$ & $A\vee B$\\
\hline
w & w & w\\
w & f & w\\
f & w & w\\
f & f & f\\
\end{tabular}
\end{center}
\item exklusives oder: $(A\vee B) \wedge (\neg(A\wedge B))$
\end{itemize}
\item Beispiel $A$: 7 ist eine Primzahl (w), $B$: 5 ist gerade (f)
\begin{itemize}
\item $A\wedge B$ 7 ist eine Primzahl und 5 ist gerade (f)
\item $A\vee B$ 7 ist eine Primzahl oder 5 ist gerade (w)
\end{itemize}

\item Implikation (WENN-DANN-Verknüpfung)
\begin{itemize}
\item Symbol: $\Rightarrow$
\item Wahrheitstafel:
\begin{center}
\begin{tabular}{lll}
$A$ & $B$ & $A\Rightarrow B$\\
w & w & w\\
w & f & f\\
f & w & w\\
f & f & w\\
\end{tabular}
\end{center}
\item Sprechweise: $A$ impliziert $B$, aus $A$ folgt $B$, $A$ ist eine hinreichende Bedingung für $B$ (ist $A\Rightarrow B$ wahr, dann folgt aus $A$ wahr, $B$ ist wahr), $B$ ist eine notwendige Bedingung für $A$ (ist $A\Rightarrow B$ wahr, dann kann $A$ nur dann wahr sein, wenn Aussage $B$ wahr ist)
\item Beispiel Es seinen $m,n\in\mathbb{N}$
\begin{itemize}
\item $A$: m ist gerade
\item $B$: $mn$ ist gerade
\item Dann gilt $\Forall m,n \in\mathbb{N}~A\Rightarrow B~\text{wahr}$ \\
         Fallunterscheidung:
\begin{itemize}
\item $m$ gerade, $n$ gerade, dann ist $A$ wahr, $B$ wahr, d.h. $A\Rightarrow B$ wahr
\item $m$ gerade, $n$ ungerade, dann ist $A$ wahr, $B$ wahr, d.h. $A\Rightarrow B$ wahr
\item $m$ ungerade, $n$ gerade, dann ist $A$ falsch, $B$ wahr, d.h. $A\Rightarrow B$ wahr
\item $m$ ungerade, $n$ ungerade, dann ist $A$ falsch, $B$ falsh, d.h. $A\Rightarrow B$ wahr
\end{itemize}
\end{itemize}
\end{itemize}
\item Äquivalenz (GENAU-DANN-WENN-Verknüpfung)
\begin{itemize}
\item Symbol $\Leftrightarrow$
\item Wahrheitstafel:
\begin{center}
\begin{tabular}{lll}
$A$ & $B$ & $A\Leftrightarrow$ B\\
\hline
w & w & w\\
w & f & f\\
f & w & f\\
f & f & w\\
\end{tabular}
\end{center}
\item Sprechweise: $A$ gilt genau dann, wenn $B$ gilt, $A$ ist hinreichend und notwendig für $B$ \\
       Die Aussagen $A\Leftrightarrow B$ und $(A\Rightarrow B)\wedge (B\Rightarrow A)$ sind gleichbedeutend:
\begin{center}
\begin{tabular}{llllll}
$A$ & $B$ & $A\Leftrightarrow B$ & $A\Rightarrow B$ & $B\Rightarrow A$ & $(A\Rightarrow B)\wedge (B\Rightarrow A)$\\
\hline
w & w & w & w & w & w\\
w & f & f & f & w & f\\
f & w & f & w & f & f\\
f & w & f & w & f & f\\
f & f & w & w & w & w\\
\end{tabular}
\end{center}
\item Beispiel: Es sei $n$ eine ganze Zahl \\
       $A:~n-2>1$ \\
       $B:~n>3$ \\
       $\Forall n\in\mathbb{N}~\text{gilt}~A\Leftrightarrow B$
       $C:~n>0$ \\
       $D:~n^2>0$ \\
       Für $n=-1$ ist die Äquivalenz $C\Leftrightarrow$ falsch ($C$ falsch, $D$ wahr) \\
       Für alle ganzen Zahlen $n$ gilt zumindest die Implikation $C\Rightarrow D$
\end{itemize}
\end{itemize}
\subsection{Beweis}
\label{sec-2-2}
Mathematische Sätze, Bemerkungen, Folgerungen, etc. sind meistens in Form wahrer Implikationen formuliert
\subsubsection{beweisen}
\label{sec-2-2-1}
Begründen warum diese Implikation wahr ist
\subsubsection{Beweismethoden for diese Implikation $A\Rightarrow B$}
\label{sec-2-2-2}
\begin{itemize}
\item direkter Beweis ($A\Rightarrow B$)
\item Beweis durch Kontraposition ($\neq B \Rightarrow \neg A$)
\item Widerspruchbeweis ($\neg (A\wedge \neg B)$)
\end{itemize}
Diese sind äquivalent zueinander
\begin{center}
\begin{tabular}{lllllll}
$A$ & $B$ & $\neg A$ & $\neg B$ & $A\Rightarrow B$ & $\neg B \Rightarrow \neg A$ & $\neg (A \wedge \neg B)$\\
\hline
w & w & f & f & w & w & w\\
w & f & f & w & f & f & f\\
f & w & w & f & w & w & w\\
f & f & w & w & w & w & w\\
\end{tabular}
\end{center}
\begin{enumerate}
\item Beispiel
\label{sec-2-2-2-1}
$m,n$ natürliche Zahlen \\
     \[A:~m^2 < n^2\]
\[B:~m < n\]
Wir wollen zeigen, dass $A\Rightarrow B$ für alle natürlichen Zahlen $m,n$ wahr ist
\begin{itemize}
\item direkter Beweis: \\
       \[A:~m^2 < n^2 \Rightarrow 0 < n^2 - m^2 \Rightarrow 0 < (n-m)\underbrace{(n+m)}_{>0} \Rightarrow 0 < n-m \Rightarrow m<n\]
\item Beweis durch Kontraposition: \\
       \[\neg B:~m \geq n \Rightarrow m^2\geq n m \wedge m n \geq n^2 \Rightarrow m^2 \geq n^2 \Rightarrow \neg A\]
\item Beweis durch Widerspruch: \\
       \[A\wedge \neg B \Rightarrow m^2 < n^2 \wedge n\leq m \Rightarrow m^2 < n^2 \wedge m n \leq m^2 \wedge n^2 \leq m n \Rightarrow m n \leq m^2 < n^2 \leq m n\]
       Wiederspruch
\end{itemize}
\end{enumerate}
\subsection{Existenz- und Allquantor}
\label{sec-2-3}
\subsubsection{Existenzquantor}
\label{sec-2-3-1}
\$A(x) Aussage, die von Variable x abhängt \\
    $\exists x:~A(x)$ ist gleichbedeutend mit "Es existiert ein x, für das $A(x)$ wahr ist" (hierbei ist "existiert ein x" im Sinne von "existiert mindestens ein x" zu verstehen) \\
    Beispiel:
\[\exists n\in\mathbb{N}:~n>5\quad\text{(w)}\]
$\exists !x:~A(x)$ ist gleichbedeutend mit "Es existiert genau ein x, für dass $A(x)$ wahr ist" $\backslash$
\subsubsection{Allquantor}
\label{sec-2-3-2}
$\Forall x:~A(x)$ ist gleichbedeutend mit "Für alle x ist A(x) wahr"
Beispiel:
\[\Forall n\in\mathbb{N}: 4n~\text{ist gerade}\]
\subsubsection{Negation von Existenz- und Allquantor}
\label{sec-2-3-3}
\[\neg(\exists x:~A(x)) \Leftrightarrow \Forall x:~\neg A(x)\]
\[\neg(\Forall x:~A(x)) \Leftrightarrow \exists x:~\neg A(x)\]
\subsubsection{Spezielle Beweistechniken für Existenz und Allaussagen}
\label{sec-2-3-4}
\begin{itemize}
\item Angabe eines Beispiel, um zu zeigen, dass deine Existenzaussage wahr ist. \\
      Beispiel:
\[\exists n\in\mathbb{N}:~n>5 \text{ist wahr, denn für $n = 7$ ist die Aussage $n > 5$ wahr}\]
\item Angabe eines Gegenbeispiel, um zu zeigen, dass eine Allausage falsch ist. \\
      Beispiel:
\[\Forall n\in\mathbb{N}:~n\leq 5 \text{ist flasch, dann für $n=7$ ist die Aussage $n\leq 5$ falsch}\]
\end{itemize}
\subsection{Naive Mengenlehre}
\label{sec-2-4}
Mengenbegriff nach Cantor: \\
   Eine Menge ist eine Zusammenfassung von bestimmten, wohlunterschiedenen Objekten userer Anschauung oder useres Denkens (die Elemente genannt werden) zu einem Ganzen

\subsubsection{Schreibweise}
\label{sec-2-4-1}
\begin{itemize}
\item $x\in M$, falls $x$ ein Element von $M$ ist
\item $x\not\in M$, falls $x$ kein Element von $M$ ist
\item $M=N$, falls $M$ und $N$ die gleichen Elemente besitzen, $M\subseteq N \wedge N\subseteq M$
\end{itemize}

\subsubsection{Angabe von Mengen}
\label{sec-2-4-2}
\begin{itemize}
\item Reihenfolge ist unrelevant (\$\{1,2,3\}=\{1,3,2\})
\item Elemente sind wohlunterschieden $\{1,2,2\} = \{1,2\}$
\item Auflisten der Elemente $M=\{a,b,c,\ldots\}$
\item Beschreibung der Elemente durch Eigenschaften: $M=\{x\mid E(x)\}$ \\
     (Elemente x, für die E(x) wahr)
\begin{itemize}
\item Beispiel:
\[\{2,4,6,8\} = \{x\mid x\in\mathbb{N}, x~\text{gerade}, 1 < x < 10\}\]
\end{itemize}
\end{itemize}
\subsubsection{leere Menge}
\label{sec-2-4-3}
Die leere Menge $\emptyset$ enthält keine Elemente
\begin{enumerate}
\item Beispiel
\label{sec-2-4-3-1}
\[\{x\mid x\in\mathbb{N}, x < -5\} = \emptyset\]
\end{enumerate}

\subsubsection{Zahlenbereiche}
\label{sec-2-4-4}
Menge der natürlichen Zahlen:
\[\mathbb{N} := \{1,2,3,\ldots\}\]

Menge der natürlichen Zahlen mit Null:
\[\mathbb{N}_0 := \{0, 1,2,3,\ldots\}\]

Menge der Ganzen Zahlen:
\[\mathbb{Z} := \{0,1,-1,2,-2\}\]

Menge der rationalen Zahlen:
\[\mathbb{Q} := \{\frac{m}{n} \mid m\in\mathbb{Z}, n\in\mathbb{N}\}\]

Menge der reellen Zahlen: $\mathbb{R}$

\subsubsection{Teilmenge}
\label{sec-2-4-5}
$A,B$ seien Mengen. \\
    $A$ heißt Teilmenge von $B~(A\subseteq B) \xLeftrightarrow{\text{Def.}} \Forall x\in A: x\in B$
$A$ heißt echte Teilmenge von $B~(A\subset B) \xLeftrightarrow{\text{Def.}} A\subseteq B \wedge A\neq B$
\begin{enumerate}
\item Anmerkung
\label{sec-2-4-5-1}
Offenbar gilt für Mengen $A,B$:
\[A=B \Leftrightarrow A\subseteq B \wedge B\subseteq A\]
$\emptyset$ ist Teilmenge jeder Menge

\item Beipspiel
\label{sec-2-4-5-2}
\[\mathbb{N}\subset\mathbb{N}_0\subset\mathbb{Z}\subset\mathbb{Q}\]
\end{enumerate}

\subsubsection{Durschnitt}
\label{sec-2-4-6}
\[A \cap B := \{x\mid x\in A \wedge x\in B\}\]
\begin{enumerate}
\item Beispiel
\label{sec-2-4-6-1}
\[A=\{2,3,5,7\}, B=\{3,4,6,7\}, A\cap B = \{3,7\}\]
\end{enumerate}

\subsubsection{Vereinigung}
\label{sec-2-4-7}
\[A\cup B := \{x\mid x\in A \vee x\in B\}\]
\begin{enumerate}
\item Beispiel
\label{sec-2-4-7-1}
\[A=\{2,3,5,7\}, B=\{3,4,6,7\}, A\cup B = \{2,3,4,5,6,7\}\]
\end{enumerate}

\subsubsection{Differenz}
\label{sec-2-4-8}
\[A\setminus B := \{x\mid x\in A \wedge x\not\in B\}\]
Im Fall $B\subseteq A$ nennt man $A\setminus B$ auch das Komplement von $B$ in $A$ und schreibt $\mathcal{c}_A(B) = A\setminus B$
\begin{enumerate}
\item Beispiel
\label{sec-2-4-8-1}
\[A=\{2,3,5,7\}, B=\{3,4,6,7\}, A\setminus B = \{2,5\}\]
\end{enumerate}
\subsubsection{Bemerkung zu Vereinigung und Durschnitt}
\label{sec-2-4-9}
$A,B$ seien zwei Mengen. Dann gilt \[A\cap (B\cup C) = (A\cap B) \cup (A\cap C)\]
\begin{enumerate}
\item Beweis
\label{sec-2-4-9-1}
\[A\cap(B\cup C) \subseteq (A\cap B) \cup (A\cap C)\]
\[A\cap(B\cup C) \supseteq (A\cap B) \cup (A\cap C)\]
"$\subseteq$" Sei $x\in A \cap (B\cup C)$. Dann ist $x\in A \wedge x\in B\cup C$
\begin{itemize}
\item 1. Fall: $x\in A \wedge x\in B$
       \[\Rightarrow x\in A\cap B \Rightarrow x \in (A\cap B) \cup (A\cap C)\]
\item 2. Fall $x\in A \wedge x\in C$
       \[\Rightarrow x\in A\cap C \Rightarrow x\in (A\cap B)\cup(A\cap C)\]
\end{itemize}
Damit ist "$\subseteq$" gezeigt.
"$\supseteq$" Sei \$x$\in$ (A$\cap$ B) $\cup$ (A$\cap$ C)
\[\Rightarrow x\in A\cap B \vee x\in A\cap C \\ \Rightarrow (x\in A \wedge x\in B) \vee (x\in A \wedge x\in C) \\ \Rightarrow x\in A \wedge (x\in B\vee x\in C) \\ \Rightarrow x\in A \wedge x\in B\cup C \\ \Rightarrow x\in A\cap (B\cup C)\]
Damit ist "$\supseteq$" gezeigt.
\end{enumerate}
\subsubsection{Bemerkung zu Äquivalenz von Mengen}
\label{sec-2-4-10}
Seien $A,B$ Mengen, dann sind äquivalent:
\begin{enumerate}
\item $A\cup B = B$
\item $A\subseteq B$
\end{enumerate}
\begin{enumerate}
\item Beweis
\label{sec-2-4-10-1}
Wir zeigen 1) $\Rightarrow$ 2) und 2) $\Rightarrow$ 1.
\[1) \Rightarrow 2):~\text{Es gelte}~A\cup B = B,~\text{zu zeigen ist}~A\subseteq B \\ \text{Sei}~x\in A \Rightarrow x\in A \wedge x \in B \Rightarrow x\in A\cup B = B\]
\[2) \Rightarrow 1):~\text{Es gelte}~A\subseteq B\text{, zu zeigen ist}~A\cup B = B \]
"$\subseteq$": Sei $x\in A\cup B \Rightarrow x\in A \vee x\in B \xRightarrow{A\subseteq B} x\in B$
"$\supseteq$": $B\subseteq A\cup B$ klar
\end{enumerate}
\subsubsection{Kartesisches Produkt}
\label{sec-2-4-11}
Seien $A,B$ Mengen
\[A\times B := \{(a,b)\mid a\in A, b\in B\}\]
heipt das kartesische Produkt von $A$ und $B$. Hierbei ist $(a,b) = (a',b') \xLeftrightarrow{\text{Def}} a = a' \wedge b = b'$ a = a' $\wedge$ b = b'\$

\begin{enumerate}
\item Beispiel
\label{sec-2-4-11-1}
\begin{itemize}
\item \[\{1,2\}\times \{1,3,4\} = \{(1,1),(1,3),(1,4),(2,1),(2,3),(2,4)\}\]
\item \[\mathbb{R}\times\mathbb{R}=\{(x,y)|mid x,y \in \mathbb{R}\} = \mathbb{R}^2\]
\end{itemize}
\end{enumerate}
\subsubsection{Potenzmenge}
\label{sec-2-4-12}
$A$ sei eine Menge
\[\mathcal{P} (A) := \{M\mid M\subseteq A\}\]
heißt die Potenzmenge von $A$
\begin{enumerate}
\item Beispiel
\label{sec-2-4-12-1}
\[\mathcal{P} (\{1,2,3\}) =  \{\emptyset, \{1\}, \{2\},\{3\},\{1,2\},\{1,3\},\{2,3\}\{1,2,3,4\}\}\]
\end{enumerate}
\subsubsection{Kardinalität}
\label{sec-2-4-13}
$M$ sei eine Menge. Wir setzen
\[\abs{M} := \begin{cases} n & \text{falls $M$ eine endliche Menge ist und $n$ Elemente enthält} \\ \infty & \text{falls $M$ nicht endlich ist} \end{cases}\]
$\abs{M}$ heißt Kardinalität von A
\begin{enumerate}
\item Beispiel
\label{sec-2-4-13-1}
\begin{itemize}
\item $\abs{\{7,11,16\}} = 3$
\item $\abs{\mathbb{N}} = \infty$
\end{itemize}
\end{enumerate}
\subsubsection{Bemerkung zu natürlichen Zahlen}
\label{sec-2-4-14}
Für die natürlichen Zahlen gilt das Induktionsaxiom
Ist $M\subseteq N$ eine Teilmenge, für die gilt:
\[1\in M \wedge \Forall n\in M : n\in M \Rightarrow n+1 \in M\]
dann ist $M = \mathbb{N}$
\subsubsection{Prinzip der vollständigen Induktion}
\label{sec-2-4-15}
Für jedes $n\in \mathbb{N}$ sei eine Aussage $A(n)$ gegeben. Die Aussagen $A(N)$ gelten für alle $n\in\mathbb{N}$, wenn man folgendes zeigen kann: \\
\begin{itemize}
\item (IA) $A(1)$ ist wahr
\item (IS) Für jedes $n\in\mathbb{N}$ gilt: $A(n) \Rightarrow A(n+1)$
\end{itemize}
Der Schritt (IA) heißt Induktionsanfang, die Implikation $A(n) \Rightarrow A(n+1)$ heißt Induktionsschritt
\begin{enumerate}
\item Beweis
\label{sec-2-4-15-1}
Setze $M := \{n\in \mathbb{N}\mid A(n)~\text{ist wahr}\}$
Wegen (IA) ist $1\in M$, wegen (IS) gilt: $n\in M \Rightarrow n+1 \in M$ \\
     Nach Induktionsaxiom folgt $M = \mathbb{N}$, das heißt $A(n)$ ist wahr für alle $n\in \mathbb{N}$
\item Beispiel
\label{sec-2-4-15-2}
Für $n\in\mathbb{N}$ sei $A(n)$ die Aussage: $1+\ldots + n = \frac{n(n+1)}{2}$
Wir zeigen: $A(n)$ ist wahr für alle $n\in \mathbb{N}$, und zwar durch vollständige Induktion
\begin{itemize}
\item (IA) $A(1)$ ist wahr, denn $1 = \frac{1(1+1)}{2}$
\item (IS) zu zeigen: $A(n) \Rightarrow A(n+1)$ \\
       Es gelte $A(n)$, das heißt $1+\ldots+n = \frac{n(n+1)}{2}$ ist wahr \[\Rightarrow 1 + \ldots + n + (n + 1) = \frac{n(n+1)}{2} + (n+1) =  \frac{n(n+1) + 2(n+1)}{2} = \frac{(n+1)(n+2)}{2} \square\]
\end{itemize}
\end{enumerate}
\subsection{Relationen}
\label{sec-2-5}
\subsubsection{Definiton}
\label{sec-2-5-1}
Eine Relation auf $M$ ist eine Teilmenge $R\subseteq M\times M$
Wir schreiben $a\sim b \xLeftrightarrow{\text{Def}} (a,b) \in R$ ("a steht in Relation zu b")

\begin{itemize}
\item anschaulich: eine Relation auf $M$ stellt eine "Beziehung" zwischen den Elementen von $M$ her.
\item Für $a,b \in M$ gilt entweder $a\sim b$ oder $a\not\sim b$, denn: entweder ist $(a,b) \in R$ oder $(a,b)\not\in R$
\end{itemize}
\begin{enumerate}
\item Anmerkung
\label{sec-2-5-1-1}
Aufgrund der obigen Notation spricht man in der Regel von Relation "\$$\sim$" auf $M$ als von der Relation $R \subseteq M\times M$
\item Beispiel
\label{sec-2-5-1-2}
\$M = \{1,2,3\}. Durch $R = \{(1,1), (1,2), (3,3) \subseteq M\times M\}$ ist eine Relation auf $M$ gegeben. Es gilt dann: $1\sim 1, 1\sim 2, 3\sim 3$ (aber zum Beispiel: $1\not\sim 3, 2\not\sim 1, 2\not\sim 2$)
\end{enumerate}

\subsubsection{Eigenschaften von Relationen}
\label{sec-2-5-2}
$M$ Menge, $\sim$ Relation auf $M$ \\
    $\sim$ heißt:
\begin{itemize}
\item reflexiv $\xLeftrightarrow{\text{Def}}$ für alle $a\in M$ gilt $a\sim a$
\item symmetrisch $\xLeftrightarrow{\text{Def}}$ für alle $a,b\in M$ gilt: $a\sim b \Rightarrow b\sim a$
\item antisymmetrisch $\xLeftrightarrow{\text{Def}}$ für alle $a,b \in M$ gilt: $a\sim b \wedge b\sim a \Rightarrow a = b$
\item transitiv $\xLeftrightarrow{\text{Def}}$ für alle $a,b,c\in M$ gilt: $a\sim b \wedge b\sim v \Rightarrow a\sim c$
\item total $\xLeftrightarrow{\text{Def}}$ für alle $a,b\in M$ gilt: $a\sim b \vee b\sim a$
\end{itemize}
\begin{enumerate}
\item Beispiel
\label{sec-2-5-2-1}
Sei $M$ die Menge der Studierenden in der LA1-Vorlesung
\begin{enumerate}
\item Für $a,b \in M$ sei $a\sim b \xLeftrightarrow{\text{Def}}$ $a$ hat den selben Vornamen wie $b$ \\
        $\sim$ reflexiv, symmetrisch, nicht antisymmetrisch, transitiv, nicht total
\item Für $a,b \in M$ sei $a\sim b \xLeftrightarrow{\text{Def}}$ Martrikelnummer von $a$ ist kleiner gleich als die Martrikelnummer von $b$ \\
        $\sim$ ist reflexiv, nicht symmetrisch, antisymmetrisch, transitiv, total
\item Für $a,b \in M$ sei $a\sim b \xLeftrightarrow{\text{Def}}$ $a$ sitzt auf dem Platz recht von $b$ \\
        $\sim$ ist nicht reflexiv, nicht symmetrisch, nicht antisymmetrisch, nicht transitiv, nicht total
\end{enumerate}
\end{enumerate}
\subsubsection{Halbordnung / Totalordung}
\label{sec-2-5-3}
$\sim$ heißt
\begin{itemize}
\item Halbordnung auf $M\xLeftrightarrow{\text{Def}}~\sim$ ist reflexiv, antisymmetrisch und transitiv
\item Totalordung auf $M\xLeftrightarrow{\text{Def}}~\sim$ ist eine Halbordnung und $\sim$ ist total
\end{itemize}
In diesen Fällen sagt man auch: Das Tupel $(M,\sim)$ ist eine halbgeordnete, beziehungsweise totalgeordnete Menge.
\begin{enumerate}
\item Beispiel
\label{sec-2-5-3-1}
\begin{enumerate}
\item $\leq$ auf $\mathbb{N}$ ist eine Totalordung
\item Sei $M = \mathcal{P}(\{1,2,3\})$. $\subseteq$ ist auf $M$ eine Halbordung, aber keine Totalordung (es ist zum Beispiel weder $\{1\} \subseteq \{3\}$ noch $\{3\}\subseteq \{\}$)
\end{enumerate}
\item Anmerkung
\label{sec-2-5-3-2}
Wegen der Analogie zur $\leq$ auf $\mathbb{N}$ bezeichnen wir Halbordnungen in der Regel mit $\leq$
\end{enumerate}
\subsubsection{Größtes / kleistes Element}
\label{sec-2-5-4}
$(M, \leq)$ halbgeordnete Menge, $a\in M$ \\
    $a$ heißt ein
\begin{itemize}
\item größtes Element von $M\xLeftrightarrow{\text{Def}}$ Für alle $x\in M$ gilt $x\leq a$
\item kleinstes Element von $M\xLeftrightarrow{\text{Def}}$ Für alle $x\in M$ gilt $a\leq x$
\end{itemize}
\begin{enumerate}
\item Bemerkung
\label{sec-2-5-4-1}
$(M,\leq)$ halbgeordnete Menge \\
     Dann gilt: Existiert in $M$ ein größtes (beziehungsweise kleinstes) Element, so ist dieses eindeutig bestimmt
\begin{enumerate}
\item Beweis
\label{sec-2-5-4-1-1}
Es seien $a,b\in M$ größte Elemente von $M$ \\
      $\Rightarrow x\leq a$ für alle $x\in M$, also auch $b\leq a$ \\
      Außerdem: \$x $\le$ b für alle $x\in M$, also auch $a\leq b$ \\
      $\xRightarrow{\text{Antisymmetrie}} a = b$ \\
      Analog für kleinstes Element
\item Anmerkung
\label{sec-2-5-4-1-2}
Dies sagt nichts darüber aus, ob ein größtes (beziehungsweise kleinstes) Element in $M$ überhaupt existiert.
\end{enumerate}
\item Beispiel
\label{sec-2-5-4-2}
\begin{enumerate}
\item In $(\mathbb{N},\leq)$ ist 1 das kleinste Element, ein größtes Element gibt es nicht
\item $(\{\{1\},\{2\},\{3\},\{1,2\},\{1,3\},\{2,3\}\}, \subseteq)$ ist eine halbgeordnete Menge ohne kleinstes beziehungsweise größtes Element
\end{enumerate}
\end{enumerate}
\subsubsection{maximales / minimales Element}
\label{sec-2-5-5}
$(M,\leq)$ halbgeordnete Menge, $a\in M$ \\
    $a$ heißt ein
\begin{itemize}
\item maximales Element von $M \xLeftrightarrow{\text{Def}}$ für alle $x\in M$ gilt: $a\leq x \Rightarrow a = x$
\item minmales Element von $M \xLeftrightarrow{\text{Def}}$ für alle $x\in M$ gilt: $x\leq a \Rightarrow a = x$
\end{itemize}
\begin{enumerate}
\item Beispiel
\label{sec-2-5-5-1}
In $(\{\{1\},\{2\},\{3\},\{1,2\},\{1,3\},\{2,3\}\}, \subseteq)$ sind $\{1,2\},\{1,3\},\{2,3\}$ maximale Elemente und $\{1\},\{2\},\{3\}$ sind minimale Elemente.
\item Bemerkung
\label{sec-2-5-5-2}
$(M,\leq)$ halbgeordnete Menge, $a\in M$ \\
     Dann gilt: Ist $a$ ein größtes (beziehungsweise kleinstes) Element von $M$, dann ist $a$ ein maximales (beziehungsweise minimales) Element von $M$.
\begin{enumerate}
\item Beweis
\label{sec-2-5-5-2-1}
Sei $a$ ein größtes Element von $M$. \\
      zu zeigen ist: Für alle $x\in M$ gilt $a\leq x \Rightarrow a = x$
Sei $x\in M$ mit $a\leq x$. Da $a$ größtes Element von $M$ ist, gilt auch $x\leq a$ \\
      $\xLeftrightarrow{\text{Antisymmetrie}} a = x$ \\
      Analog für kleinstes Element.
\end{enumerate}
\end{enumerate}
\subsubsection{Äquivalenzrelation}
\label{sec-2-5-6}
$M$ Menge, $\sim$ auf $M$ \\
    $\sim$ heißt Äquivalenzrelation $\xLeftrightarrow{\text{Def}}~\sim$ ist reflexiv, symmetrisch und transitiv.
In dem Fll sagen wir für $a\sim b$ auch $a$ ist äquivalent zu $b$. Für $a\in M$ heißt $[a]:=\{b\in M \mid b\sim a\}$ heißt die Äquivalentklasse von $a$.
Elemente aus $[a]$ nennt man Vertreter oder Repräsentanten von $a$
\begin{enumerate}
\item Beispiel
\label{sec-2-5-6-1}
$M$ Menge aller Bürgerinnen und Bürger Deutschlands. \\
     Wir definieren für $a,b\in M$ $a\sim b \xLeftrightarrow{\text{Def}} a$ und $b$ sind im selben Jahr geboren. \\
     $\sim$ ist ein Äquivalenzrelation. \\
     Jerôme Boateng wrude 1988 geboren. $[\text{Jerôme Boateng}] = \{b\in M\mid b~\text{ist im selben Jarh geboren wei Jerôme Boateng}\} = \{b\in M\mid b~\text{wurde 1988 geboren}\}$
Weitere Vertreter von $[\text{Jerôme Boateng}]$ sind zum Beispiel Mesut Özil, Mats Hummels.
Es ist $[\text{Jerôme Boateng}] = [\text{Mesut Özil}] = [\text{Mats Hummels}]$.
Man sieht in diesem Beispiel: Die Menge $M$ zerfällt komplett in verschiedene Äquivalentzklassen:
\begin{itemize}
\item Jeder Bürger / jede Bürgerinn Detuschalnds ist in genau einer Äquivalenzklasse enthalten
\item Jede zwei Äquivalentklasse sind endweder gleich oder disjunkt (haben leeren Durchschnitt)
\end{itemize}
\item Bemerkung
\label{sec-2-5-6-2}
$M$ Menge, $\sim$ Äquivalenzrelation auf $M$ \\
     Dann gilt:
\begin{enumerate}
\item Jedes Element von $M$ liegt in genau einer Äquivalenzklasse
\item Je zwei Äquivalenzklassen sind entweder gleich oder disjunkt
\end{enumerate}
Man sagt auch: Die Äquivalenzklassen bezüglich "$\sim$" bilden eine \textbf{Partition} von $M$.
\begin{enumerate}
\item Beweis
\label{sec-2-5-6-2-1}
\begin{enumerate}
\item Sei $a\in M$ \\
         zu zeigen: Es gibt genau eine Äquivalenzklassen, in der $a$ liegt
\begin{enumerate}
\item Es gibt eine Äquivalenzklasse, in der $a$ liegt, denn \$a$\in$ [a], denn $a\sim a$
\item Ist \$a$\in$[b] und a$\in$[c], dann ist [b]=[c] (d.h. $a$ liegt in höchstens einer Äquivalenzklasse) \\
            denn: Seien $b,c\in M$ mit $a\in[b]$ und $a\in[c]$
            $\Rightarrow a\sim b$ und $a\sim c \xRightarrow{\text{Symmetrie}} b\sim a und a\sim c \xRightarrow{\text{Transitivität}} b\sim c$
            Behautptung $[b] =[c]$
            denn: "$\subseteq$" Sei $x\in [b] \Rightarrow x\sim b \xRightarrow{Transitivität}^{b\sim c} x\sim c \Rightarrow x\in [c]$
            denn: "$\supseteq$" Sei $x\in [c] \Rightarrow x\sim c \xRightarrow{Transitivität}^{c\sim b} x\sim b \Rightarrow x\in [b]$
\end{enumerate}
\item Sind $b,c\in M$ mit $[b] \cap [c] \neq \emptyset$, dann existiert ein \$a$\in$ [b]$\cap$ [c], und es folgt wie in 2.: \\
         $[b] = [c]$
         Für $b,c\in M$ gilt also entweder $[b]\cap[c] =\emptyset$ oder $[b] = [c]\hfill\square$
\end{enumerate}
\end{enumerate}
\item Faktormenge
\label{sec-2-5-6-3}
$M$ Menge, $\sim$ Äquivalenzrelation auf $M$
$M/\sim := \{[a]|a\in M\}$ (Menge der Äquivalenzklassen) heißt die Faktormenge (Quotientenmenge) von $M$ nach $\sim$
\begin{enumerate}
\item Beispiel
\label{sec-2-5-6-3-1}
\[M= \{1,2,3,-1,-2,-3\}\]
Für $a,b,c \in M$ setzen wir $a\sim b \xLeftrightarrow{\text{Def.}} \abs{x} = \abs{b}$
Das ist eine Äquivalenzrelation auf $M$
Es ist \$\footnote{DEFINITION NOT FOUND.} = \{1,-1\},\footnote{DEFINITION NOT FOUND.}=\{2,-2\},\footnote{DEFINITION NOT FOUND.}=\{3,-3\}
Somit: \$M/sim := \{\footnotemark[1]{},\footnotemark[2]{},\footnotemark[3]{}\} = \{\{1,-1\},\{2,-2\},\{3,-3\}\}
\item Anmerkung
\label{sec-2-5-6-3-2}
Der Übergang zur Äquivalenzklassen soll (für eine jeweils gegebene Relation) irrelevante Informationen abstreifen.
\end{enumerate}
\end{enumerate}
\subsection{Abbildungen}
\label{sec-2-6}
\textbf{naive Definition}: \\
    Eine Abbildung $f$ von $M$ nach $N$ ist eine Vorschrift, die jedem $n\in M$ genau ein Element aus $N$ zuordnet, dieses wird mit $f(n)$ bezeichnet.
\textbf{Notation}: \\
    \[f:M\to N,m\mapsto f(m)\]

Zwei Abbildungen $f,g:M\to N$ sind gleich, wenn gilt $\Forall n\in M:f(n) = g(n)$
$M$ heißt die Definitionsmenge von $f$, $N$ heißt die Zielmenge von $f$
\subsubsection{Definition}
\label{sec-2-6-1}
Eine Abbildung $f$ von $M$ nach $N$ ist ein Tupel $(M,N,G_f)$, wobei $G_f$ eine Teilmenge von $M\times N$ mit der Eigenschaft ist, dass für jedes Element $m\in M$ genau ein Element $n\in N$ mit $(m,n) \in G_f$ existiert.
(für dieses Element $n$ schreiben wir auch $f(m)$). $G_f$ heißt der Graph von $f$.
\subsubsection{Beispiel}
\label{sec-2-6-2}
\begin{enumerate}
\item $f:\mathbb{R}\to\mathbb{R}, x\mapsto x^2$
\item $f:\mathbb{R}\to\mathbb{R}^2,x\mapsto (x,x+1)$
\item $M$ Menge, $id_M: M\to M,m\mapsto m$ heißt Identität (identische Abbildung) auf $M$
\item $I$,$M$ Mengen: Eine über $I$ indizierte Familie von Elementen von $M$ ist eine Abbildung: \\
       $m:I\to M,i\mapsto m(i) =: m_i$. Wir schreiben für die Familie auch kurz $(m_i)_{i\in I}$. $I$ heißt Indexmenge der Familie.
\item Spezialfall von 4.: $I = \mathbb{N},M = \mathbb{R}:~((m_i)_{i\in\mathbb{N}})$ nennt man auch Folge reeler Zahlen.
\end{enumerate}
\subsubsection{Anmerkung über den Begriff der Familie}
\label{sec-2-6-3}
Über den Begriff der Familie lassen sich diverse Konstruktionen aus der naiven Mengenlehre verallgemeinern.
Ist $(M_i)_{i\in I}$ eine Familie von Mengen, dann ist:
\[\cup_{i\in I} M_i:=\{x\mid\exists i\in I: x\in M_i\}\]
\[\cap_{i\in I}M_i := \{x\mid\Forall i\in I: x\in M_i\}\]
\[\prod_{i\in I}M_i := \{(x_i)_{i\in I}\mid \Forall i\in I: x_i \in M\}\]
\subsubsection{Bild}
\label{sec-2-6-4}
$m,N$ Mengen, $f:M\to n$ Abbildung. \\
    Sind $m\in M,n\in N$ mit $n = f(m)$ dann nennen wir $n$ ein \textbf{Bild} von $m$ unter $f$ und wir nennen $m$ ein \textbf{Urbild} von $n$ unter $f$.
\begin{enumerate}
\item Anmerkung
\label{sec-2-6-4-1}
In obiger Situation ist das Bild von $m$ unter $f$ eindeutig bestimmt (nach der Definition einer Abbildung)
Urbilder sind im allgemeinen nicht eindeutig bestimmt, und im Allgemeinen besitzt nicht jedes Element aus $N$ ein Urbild.
\item Beispiel
\label{sec-2-6-4-2}
$f:\mathbb{R}\to\mathbb{R},x\mapsto x^2$, dann ist $4=f(2) = f(-2)$, das heißt $2$ und $-2$ sind Urbilder von $4$, das Element $-5$ hat kein Urbild unter $f$, denn es existiert kein $x\in\mathbb{R}$ mit $x^2 = -5$
\item Definition
\label{sec-2-6-4-3}
$M, N$ Mengen, $f:M\to N$ Abbildung, $A\subseteq M, B\subseteq N$ \\
     $f(A) := \{f(a)\mid a\in A\} \subseteq N$ heißt das Bild von $A$ unter $f$. \\
     $f^-1(B) := \{m\in M\mid f(m) \in B\} \subseteq M$ heißt das Urbild von $B$ unter $f$
\begin{enumerate}
\item Beispiel
\label{sec-2-6-4-3-1}
\[f:\mathbb{R}\to\mathbb{R},x\mapsto x^2\]
\[f(\{1,2,3\}) = \{1,4,9\}\]
\[f^-1(\{4,-5\}) = \{2,-2\}\]
\[f^-1(\{4\}) = \{2,-2\}\]
\[f^-1(\{-5\}) = \emptyset\]
\[f(\mathbb{R}) = {x^2\mid x\in \mathbb{R}} = \{x\in\mathbb{R}\mid x\geq 0\} =:\mathbb{R}_{\geq 0}\]
\end{enumerate}
\end{enumerate}
\subsubsection{Restriktion}
\label{sec-2-6-5}
$M,N$ Mengen, $f:M\to N$ Abbildung, $A\subseteq M$
\[f\mid_A:A\to N, m\mapsto f(m)\]
heißt die Restriktion von $f$ auf $A$.
Ist $B\subseteq N$ mit $f(A) \subseteq B$, dann setzen wir
\[f\mid_A^B: A\to B,m\mapsto f(m)\]
Ist $f(M) \subseteq B$ dann setzen wir:
\[f\mid^B := f\mid_M^B,M\to B, m\mapsto f(m)\]
\subsubsection{Komposition}
\label{sec-2-6-6}
$L,M,N$ Mengen, $f:L\to M,g:M\to N$ Abbildung \\
    \[g\circ f: L\to N, x\mapsto(g\circ f)(x):=g(f(x))\]
heißt die Komposition (Hintereinanderausführung) von $f$ und $g$
\begin{enumerate}
\item Beispiel
\label{sec-2-6-6-1}
\[f:\mathbb{R}\to\mathbb{R},x\mapsto x^2, g:\mathbb{R}\to\mathbb{R}:x\mapsto x + 1\]
\[\Rightarrow g\circ f:\mathbb{R}\to\mathbb{R},x\mapsto g(f(x)) = g(x^2) = x^2 + 1\]
\item Assoziativität
\label{sec-2-6-6-2}
$L,M,N,P$ Mengen, $f:L\to M, g:M\to N,h:n\to p$ \\
     Dann gilt
\[h\circ (g\circ f) = (h\circ g)\circ f\]
das heißt die Verknüpfung von Abbildungen ist assoziativ.
\begin{enumerate}
\item Beweis
\label{sec-2-6-6-2-1}
Für $x\in L ist$ \\
      \[(h\circ (g\circ f)) = h((g\circ f)(x)) = h(g(f(x))) = (h\circ g)(f(x)) = ((h\circ g)\circ f)(x)\hfill\square\]
\end{enumerate}
\end{enumerate}
\subsubsection{Eigenschaften von Abbildungen}
\label{sec-2-6-7}
$M,N$ Mengen, $f:M\to N$ Abbildung
\begin{enumerate}
\item Injektivität
\label{sec-2-6-7-1}
$f$ heißt injektiv: \[\xLeftrightarrow{\text{Def}} \Forall m_1,m_2\in M: f(m_1) = f(m_2) \Rightarrow m_1 = m_2 \Leftrightarrow \Forall m_1,m_2\in M: m_1\neq m_2 \Rightarrow f(m_1)\neq f(m_2)\]
\item Surjektivität
\label{sec-2-6-7-2}
$f$ heißt sujektiv:
\[\xLeftrightarrow{\text{Def}} \Forall n\in M :\exists m\in M: f(m) = n \Leftrightarrow f(M) = N\]
\item Bijektivität
\label{sec-2-6-7-3}
$f$ heißt bijektiv: $\xLeftrightarrow{\text{Def}}$ $f$ ist injektiv und surjektiv
\item Beispiel
\label{sec-2-6-7-4}
\begin{enumerate}
\item $f:\mathbb{R}\to\mathbb{R},x\mapsto x^2$ ist:
\begin{itemize}
\item nicht injektiv, denn $f(2) = f(-2)$, aber $2\neq -2$
\item nicht surjektiv, denn es existier kein $m\in\mathbb{R}$ mit $f(m) = -1$
\item nicht bijektiv
\end{itemize}
\item $f:\mathbb{R}_{\geq 0} \to \mathbb{R}, x\mapsto x^2$ ist:
\begin{itemize}
\item injektiv, denn für $m_1,m_2 \in\mathbb{R}_{\geq 0}$ gilt: $f(m_1) = f(m_2) \Rightarrow m_1^2 = m_2^2 \xRightarrow{m_1,m_2 > 0} m_1 = m_2$
\item nicht surjektiv, denn es existier kein $m\in\mathbb{R}_{\geq 0}$ mit $f(m) = -1$
\item nicht bijektiv
\end{itemize}
\item $f:\mathbb{R}_{\geq 0} \to \mathbb{R}_{\geq 0}, x\mapsto x^2$ ist:
\begin{itemize}
\item injektiv, denn für $m_1,m_2 \in\mathbb{R}_{\geq 0}$ gilt: $f(m_1) = f(m_2) \Rightarrow m_1^2 = m_2^2 \xRightarrow{m_1,m_2 > 0} m_1 = m_2$
\item surjektiv, denn für $m\in\mathbb{R}_{\geq 0}$ ist $f(\sqrt{m}) = (\sqrt{m})^2 = m$
\item bijektiv
\end{itemize}
\end{enumerate}
\item Bemerkung
\label{sec-2-6-7-5}
$M,N$ Mengen, $f:M\to N, g:n\to M$ mit $g\circ f = id_M$
Dann ist $f$ injektiv und $g$ surjektiv.
\begin{enumerate}
\item Beweis
\label{sec-2-6-7-5-1}
\begin{enumerate}
\item $f$ ist injektiv, denn: \\
         Seien $m_1, m_2 \in M$ mit $f(m_1) = f(m_2) \Rightarrow g(f(m_1)) = g(f(m_2)) \Rightarrow (g\circ f)(m_1) = (g\circ f)(m_2) \Rightarrow id_m(m_1) = id_M(m_2)\Rightarrow m_1 = m_2$
\item $g$ ist surjektiv, denn: \\
         Sei $m\in M$ Dann ist $m=id_M(m) = (g\circ f)(m) = g(f(m))$
\end{enumerate}
\end{enumerate}
\end{enumerate}
% Emacs 25.1.1 (Org mode 8.2.10)
\end{document}